%%%%%%%%%%%%%%%%%%%%%%%%%%%%%%%%%%%%%%%%%%%%%%%%%%%%%%%%%%%%%%%%%%%%%%%%%
%
% Complete Book Template - Yolah Game AI Engine
%
% A comprehensive single-file template for writing a technical book
% This template includes all necessary packages, custom environments,
% and example content to get you started quickly.
%
% USAGE:
% 1. Copy this file: cp book_template.tex book.tex
% 2. Edit book.tex with your content
% 3. Compile: make pdf (or pdflatex --shell-escape book.tex)
%
%%%%%%%%%%%%%%%%%%%%%%%%%%%%%%%%%%%%%%%%%%%%%%%%%%%%%%%%%%%%%%%%%%%%%%%%%

% you were helping me translating the book from french to english. I write in french because it is easier for me and you correct my french, put it in comments and translate into english


\documentclass[12pt,a4paper,twoside,openright]{book}

%%%%%%%%%%%%%%%%%%%%%%%%%%%%%%%%%%%%%%%%%%%%%%%%%%%%%%%%%%%%%%%%%%%%%%%%%
% PACKAGES
%%%%%%%%%%%%%%%%%%%%%%%%%%%%%%%%%%%%%%%%%%%%%%%%%%%%%%%%%%%%%%%%%%%%%%%%%

% Encoding and Language
\usepackage[utf8]{inputenc}
\usepackage[T1]{fontenc}
\usepackage[english]{babel}

% Page Layout
\usepackage[
    top=1in,
    bottom=1in,
    left=1.25in,
    right=1in,
    headheight=15pt
]{geometry}

% Math
\usepackage{amsmath}
\usepackage{amssymb}
\usepackage{amsthm}
\usepackage{mathtools}

% Graphics and Colors
\usepackage{graphicx}
\graphicspath{{figures/}}
\usepackage{xcolor}
\usepackage{tikz}
\usetikzlibrary{arrows.meta, positioning, shapes.geometric, calc}

% Tables
\usepackage{booktabs}
\usepackage{array}
\usepackage{multirow}
\usepackage{longtable}

% Code Listings
\usepackage{minted}
\usemintedstyle{friendly}
\setminted{
    breaklines,
    fontsize=\small,
    frame=lines,
    framesep=2mm,
    baselinestretch=1.2,
    tabsize=4
}

% Define code background color
\definecolor{codebg}{RGB}{248,248,248}

% Algorithms
\usepackage{algorithm}
\usepackage{algpseudocode}

% Bibliography
\usepackage[
    backend=biber,
    style=numeric,
    sorting=none,
    maxbibnames=99
]{biblatex}
\addbibresource{references.bib}

% Hyperlinks
\usepackage{hyperref}
\hypersetup{
    colorlinks=true,
    linkcolor=blue!50!black,
    citecolor=green!50!black,
    urlcolor=blue!50!black,
    bookmarksdepth=3,
    pdfstartview=FitH
}

% Headers and Footers
\usepackage{fancyhdr}
\pagestyle{fancy}
\fancyhf{}
\fancyhead[LE]{\leftmark}
\fancyhead[RO]{\rightmark}
\fancyfoot[C]{\thepage}
\renewcommand{\headrulewidth}{0.4pt}

% Other Useful Packages
\usepackage{float}              % Better float control
\usepackage{caption}            % Better captions
\usepackage{subcaption}         % Subfigures
\usepackage{enumitem}           % Better lists
\usepackage{footnote}           % Better footnotes
\usepackage{csquotes}           % Quotations

%%%%%%%%%%%%%%%%%%%%%%%%%%%%%%%%%%%%%%%%%%%%%%%%%%%%%%%%%%%%%%%%%%%%%%%%%
% CUSTOM COLORS
%%%%%%%%%%%%%%%%%%%%%%%%%%%%%%%%%%%%%%%%%%%%%%%%%%%%%%%%%%%%%%%%%%%%%%%%%

\definecolor{boxborder}{RGB}{52,101,164}
\definecolor{boxbg}{RGB}{232,241,250}
\definecolor{algoborder}{RGB}{46,125,50}
\definecolor{algobg}{RGB}{232,245,233}
\definecolor{resultborder}{RGB}{183,28,28}
\definecolor{resultbg}{RGB}{255,235,238}

%%%%%%%%%%%%%%%%%%%%%%%%%%%%%%%%%%%%%%%%%%%%%%%%%%%%%%%%%%%%%%%%%%%%%%%%%
% CUSTOM ENVIRONMENTS
%%%%%%%%%%%%%%%%%%%%%%%%%%%%%%%%%%%%%%%%%%%%%%%%%%%%%%%%%%%%%%%%%%%%%%%%%

% Important Box
\usepackage{tcolorbox}
\tcbuselibrary{most}

\newtcolorbox{importantbox}{
    colback=boxbg,
    colframe=boxborder,
    fonttitle=\bfseries,
    title=Important,
    boxrule=1.5pt,
    arc=3pt
}

% Algorithm Box
\newtcolorbox{algorithmbox}{
    colback=algobg,
    colframe=algoborder,
    fonttitle=\bfseries,
    title=Algorithm,
    boxrule=1.5pt,
    arc=3pt
}

% Result Box
\newtcolorbox{resultbox}{
    colback=resultbg,
    colframe=resultborder,
    fonttitle=\bfseries,
    title=Key Result,
    boxrule=1.5pt,
    arc=3pt
}

\usepackage{wasysym}
\usepackage{fontawesome}

%%%%%%%%%%%%%%%%%%%%%%%%%%%%%%%%%%%%%%%%%%%%%%%%%%%%%%%%%%%%%%%%%%%%%%%%%
% DOCUMENT INFORMATION
%%%%%%%%%%%%%%%%%%%%%%%%%%%%%%%%%%%%%%%%%%%%%%%%%%%%%%%%%%%%%%%%%%%%%%%%%

\title{
    \Huge\textbf{Yolah Board Game} \\
    \vspace{0.5cm}
    \Large Building a Two-Player Perfect-Information Game with AI Players\\ 
}

\author{
    \Large Pascal Garcia \\
    % \vspace{0.3cm}
    % \normalsize INSA Rennes
}

\date{\today}

%%%%%%%%%%%%%%%%%%%%%%%%%%%%%%%%%%%%%%%%%%%%%%%%%%%%%%%%%%%%%%%%%%%%%%%%%
% BEGIN DOCUMENT
%%%%%%%%%%%%%%%%%%%%%%%%%%%%%%%%%%%%%%%%%%%%%%%%%%%%%%%%%%%%%%%%%%%%%%%%%

\begin{document}

%%%%%%%%%%%%%%%%%%%%%%%%%%%%%%%%%%%%%%%%%%%%%%%%%%%%%%%%%%%%%%%%%%%%%%%%%
% FRONT MATTER
%%%%%%%%%%%%%%%%%%%%%%%%%%%%%%%%%%%%%%%%%%%%%%%%%%%%%%%%%%%%%%%%%%%%%%%%%

\frontmatter

% Title Page
\maketitle

\thispagestyle{empty}
\vspace*{\fill}
% \begin{center}
% \reflectbox{\copyright} \the\year{} Pascal Garcia
% \end{center}
\begin{center}
    \includegraphics[width=10cm]{logo.png}
\end{center}
\vspace*{\fill}
\clearpage


% Copyright Page
\thispagestyle{empty}
\vspace*{\fill}
% \begin{center}
% \reflectbox{\copyright} \the\year{} Pascal Garcia
% \end{center}
\begin{center}
{\ttfamily\Huge\textcolor{black}{C'est en forgeant qu'on devient forgeron}}
\end{center}
\vspace*{\fill}
\clearpage

% Dedication (Optional)
\thispagestyle{empty}
\vspace*{\fill}
\begin{center}
\textit{À Sarah, Hugo et C\'elya \textcolor{red}{\faHeart}}
\end{center}
\vspace*{\fill}
\clearpage

% Table of Contents
\tableofcontents

% List of Figures
\listoffigures

% List of Tables
\listoftables

% Preface (Optional)
%\chapter{Preface}


%%%%%%%%%%%%%%%%%%%%%%%%%%%%%%%%%%%%%%%%%%%%%%%%%%%%%%%%%%%%%%%%%%%%%%%%%
% MAIN MATTER
%%%%%%%%%%%%%%%%%%%%%%%%%%%%%%%%%%%%%%%%%%%%%%%%%%%%%%%%%%%%%%%%%%%%%%%%%

\mainmatter

%%%%%%%%%%%%%%%%%%%%%%%%%%%%%%%%%%%%%%%%%%%%%%%%%%%%%%%%%%%%%%%%%%%%%%%%%
% CHAPTER 1: INTRODUCTION
%%%%%%%%%%%%%%%%%%%%%%%%%%%%%%%%%%%%%%%%%%%%%%%%%%%%%%%%%%%%%%%%%%%%%%%%%

\chapter{Introduction}
\label{ch:introduction}

\section{The Yolah Game}

% J'ai créé le jeu Yolah dans le but d'illustrer des techniques efficaces d'implémentation de jeux de plateaux
% et d'intelligences artificielles pour mes étudiants. Je me suis inspiré du jeu des pingouins dont vous pouvez voir la boîte dans la figure \ref{fig:pingouins} (je le conseille \smiley)

I created the Yolah game to illustrate effective techniques for implementing board games
and artificial intelligences for my students. I was inspired by the penguin game, whose box you can see in Figure~\ref{fig:pingouins} (I highly recommend it \smiley)

\begin{figure}[htpb]
\centering
\includegraphics[width=0.4\textwidth]{pingouins.png}
\caption{The Pingouins game box}
% \caption{La boîte du jeu des pingouins}
\label{fig:pingouins}
\end{figure}

\begin{importantbox}
% J'ai fait au mieux de mes connaissances actuelles (\emph{ars longa, vita brevis}) pour implémenter mon jeu et les IA associées. Mais comme tout bon scientifique, il faudra regarder mon travail avec un oeil critique.
I have done my best with my current knowledge (\emph{ars longa, vita brevis}) to implement my game and the associated AIs. But like any good scientist, you should look at my work with a critical eye.
% J'ai écrit le livre en français (plus facile pour moi) et j'ai demandé à un assistant (Claude) de le traduire pour moi.
I wrote the book in French (easier for me) and asked an AI assistant (Claude~\cite{anthropic2025claude}) to translate it for me.
\end{importantbox}

% Je vais maintenant décrire les règles du jeu, puis j'expliquerai pourquoi j'ai choisi ces règles, je donnerai un exemple de partie entre deux IA et je présenterai la suite du livre.
I will now describe the rules of the game, then I will explain why I chose these rules, I will give an example of a game between two AIs and then I will present the rest of the book.

\subsection{Game Rules}

\begin{figure}[htpb]
\centering
\includegraphics[width=0.7\textwidth]{yolah_initial_configuration.png}
\caption{The initial configuration of the Yolah game}
% \caption{La configuration initiale du jeu Yolah}
\label{fig:yolah_initial_configuration}
\end{figure}

% Le plateau du jeu Yolah est représenté dans la figure \ref{fig:yolah_initial_configuration}.
% On peut y voir quatre pions noirs et quatre pions blancs placés de manière symétrique. Les noirs commencent en choisissant une de leurs pièces parmi
% les quatre. Une pièce ne pourra jamais disparaître du plateau car Yolah est un jeu sans possibilité de captures. Une pièce se déplace dans les huit directions
% aussi loin qu'elle le souhaite tant qu'elle n'est pas bloquée par une autre pièce ou un trou (notion que nous allons bientôt aborder). Par exemple, si les noirs
% choisissent de déplacer leur pièce qui se situe en {\ttfamily d5}, les cases où elle peut atterrir sont indiquées par de petites croix noires dans la figure \ref{fig:yolah_moves1}.

The Yolah game board is shown in Figure~\ref{fig:yolah_initial_configuration}.
You can see four black pieces and four white pieces placed symmetrically. Black starts by choosing one of their four pieces.
A piece can never disappear from the board because Yolah is a game without captures. A piece moves in all eight directions
as far as it wishes as long as it is not blocked by another piece or a hole (a concept we will soon discuss). For example, if black
chooses to move their piece located at {\ttfamily d5}, the squares where it can land are indicated by small black crosses in Figure~\ref{fig:yolah_moves1}.\\

\begin{figure}[htpb]
\centering
\includegraphics[width=0.67\textwidth]{yolah_moves1.pdf}
\caption{Possible moves (small black crosses) for the black piece identified by a smiley (square {\ttfamily d5})}
% \caption{Mouvements possibles (petites croix noires) pour le pion noir identifié par un smiley (case {\ttfamily d5})}
\label{fig:yolah_moves1}
\end{figure}

% Maintenant, si le pion noir en {\ttfamily d5} se déplace en {\ttfamily b7}, ce que nous noterons par {\ttfamily d5:b7}, nous obtenons la configuration représentée dans la figure \ref{fig:yolah_moves2}. On peut remarquer que la case de départ du pion noir disparaît et devient un trou~! Cette case (ce trou) devient inaccessible et infranchissable pour le reste de la partie~! Ceci va créer des opportunités pour bloquer l'adversaire et essayer de créer des zones où l'adversaire ne pourra pas aller.

Now, if the black piece at {\ttfamily d5} moves to {\ttfamily b7}, which we will denote as {\ttfamily d5:b7}, we get the configuration shown in Figure~\ref{fig:yolah_moves2}. Notice that the starting square of the black piece disappears and becomes a hole! This square (this hole) becomes inaccessible and impassable for the rest of the game! This will create opportunities to block the opponent and try to create areas where the opponent cannot go.\\

\begin{figure}[htpb]
\centering
\includegraphics[width=0.6\textwidth]{yolah_moves2.png}
\caption{Black just moved from {\ttfamily d5} to {\ttfamily b7}. The starting square {\ttfamily d5} becomes inaccessible and impassable for the rest of the game}
% \caption{Noir vient de se déplacer de {\ttfamily d5} en {\ttfamily b7}. La case de départ {\ttfamily d5} devient inaccessible et infranchissable pour le reste de la partie}
\label{fig:yolah_moves2}
\end{figure}

% Un déplacement rapporte un point au joueur qui vient de se déplacer. Par exemple, dans la configuration de la figure \ref{fig:yolah_moves2}, le joueur noir a un point et le joueur blanc qui ne s'est pas encore déplacé a zéro point. Le but du jeu est assez simple à résumer~: il faut se déplacer plus longtemps que son adversaire~!

A move earns one point for the player who just moved. For example, in the configuration of Figure~\ref{fig:yolah_moves2}, the black player has one point and the white player who has not yet moved has zero points. The goal of the game is quite simple to summarize: you must move longer than your opponent!\\

% Maintenant c'est au tour des blancs de jouer. Ils doivent décider du pion blanc qu'ils vont bouger. Supposons que ce soit le pion en {\ttfamily e5}.
% Les déplacements possibles de ce pion blanc sont indiqués dans la figure \ref{fig:yolah_moves3}. Si blanc décide d'effectuer le coup {\ttfamily e5:f5}, on se retrouve dans la configuration de la figure \ref{fig:yolah_moves4} et le score est d'un point partout (chaque joueur a joué un coup).

Now it is white's turn to play. They must decide which white piece they will move. Suppose it is the piece at {\ttfamily e5}.
The possible moves for this white piece are shown in Figure~\ref{fig:yolah_moves3}. If white decides to make the move {\ttfamily e5:f5}, we end up in the configuration of Figure~\ref{fig:yolah_moves4} and the score is one point each (each player has played one move).

\begin{figure}[htpb]
\centering
\includegraphics[width=0.67\textwidth]{yolah_moves3.pdf}
\caption{Possible moves (small white crosses) for the white piece identified by a smiley (square {\ttfamily e5})}
% \caption{Mouvements possibles (petites croix blanches) pour le pion blanc identifié par un smiley (case {\ttfamily e5})}
\label{fig:yolah_moves3}
\end{figure}

\begin{figure}[htpb]
\centering
\includegraphics[width=0.6\textwidth]{yolah_moves4.png}
\caption{White just moved from {\ttfamily e5} to {\ttfamily f5}. The starting square {\ttfamily e5} becomes inaccessible and impassable for the rest of the game. The score is one point each (each player has moved once)}
% \caption{Blanc vient de se déplacer de {\ttfamily e5} en {\ttfamily f5}. La case de départ {\ttfamily e5} devient inaccessible et infranchissable pour le reste de la partie. Le score est d'un point partout (chaque joueur s'est déplacé une fois)}
\label{fig:yolah_moves4}
\end{figure}

% Pour résumer, les règles du jeu Yolah sont les suivantes~:

To summarize, the rules of Yolah are as follows:
\begin{itemize}
% \item Le jeu est un jeu à deux joueurs (les noirs et les blancs) au tour par tour.
\item The game is a two-player game (black and white) played in turns.
% \item Chaque joueur dispose de quatre pions.
\item Each player has four pieces.
% \item À son tour, le joueur choisit un des pions qui peut encore bouger, si aucun pion ne peut bouger il passe son tour (on indiquera par le coup {\ttfamily a1:a1} le fait de passer son tour).
\item On their turn, the player chooses one of the pieces that can still move; if no piece can move, they pass their turn (we will denote by the move {\ttfamily a1:a1} to skip one's turn).
% \item Il doit déplacer le pion choisi dans une des huit directions, d'autant de cases qu'il le souhaite, mais il ne doit pas atterrir ou être bloqué par un pion ou un trou.
\item They must move the chosen piece in one of the eight directions, as many squares as desired, but must not land on or be blocked by a piece or a hole.
% \item Après le déplacement du pion choisi, la case de départ du coup devient un trou et on ne peut plus alors la franchir ou se poser dessus.
\item After moving the chosen piece, the starting square of the move becomes a hole and can no longer be crossed or landed on.
% \item Après chaque déplacement le joueur gagne un point.
\item After each move, the player earns one point.
% \item La partie se termine lorsque les deux joueurs ne peuvent plus se déplacer.
\item The game ends when both players can no longer move.
% \item Le joueur qui a le plus de points gagne la partie.
\item The player with the most points wins the game.
% \item Si les deux joueurs ont le même nombre de points, la partie est déclarée nulle.
\item If both players have the same number of points, the game is declared a draw.
\end{itemize}

% \subsection{Caractéristiques intéressantes de Yolah pour développer des IA}
\subsection{Interesting Characteristics of Yolah for Developing AIs}



% \subsection{Exemple de partie}
\subsection{Game Example}

% Pour avoir une idée du déroulement d'une partie du jeu Yolah, nous allons faire jouer deux intelligences artificielles entre elles. La première IA sera basée sur la recherche arborescente Monte-Carlo et la deuxième sera basée sur le MinMax avec réseau neuronal. Nous étudierons ces deux IA dans la suite du livre. La seconde IA est plus forte et vous allez voir sa stratégie d'isolement de zones en action~!

To get an idea of how a Yolah game unfolds, we will have two artificial intelligences play against each other. The first AI will be based on Monte Carlo Tree Search and the second will be based on Minimax with a neural network. We will study both of these AIs later in the book. The second AI is stronger and you will see its zone isolation strategy in action!\\

% Le déroulement de la partie est décrit dans les figures \ref{fig:game_example_opening}, \ref{fig:game_example_middle} et \ref{fig:game_example_end}.

The progression of the game is described in Figures~\ref{fig:game_example_opening}, \ref{fig:game_example_middle} and \ref{fig:game_example_end}.\\

% L'IA des blancs estime qu'elle est gagnante à partir du coup 10 (voir figure \ref{fig:game_example_move10}). On peut voir au coup 30 (voir figure \ref{fig:game_example_move30}) qu'elle a réussi à isoler une zone où les noirs ne pourront plus accéder. Au coup 32 (voir figure \ref{fig:game_example_move32}) elle sort un de ses pions de la zone qu'elle a isolée car l'autre pion pourra collecter tous les points de celle-ci. Il est plus utile d'utiliser l'autre pion pour glaner des points autre part. Notons qu'à partir du coup 47 (voir à partir de la figure \ref{fig:game_example_move47}) les noirs n'ont plus aucun coup disponible et doivent donc passer leur tour.

The white AI estimates that it is winning starting from move 10 (see Figure~\ref{fig:game_example_move10}). We can see at move 30 (see Figure~\ref{fig:game_example_move30}) that it has successfully isolated a zone where black can no longer access. At move 32 (see Figure~\ref{fig:game_example_move32}) it moves one of its pieces out of the isolated zone because the other piece will be able to collect all the points from that zone. It is more useful to use the other piece to gather points elsewhere. Note that starting from move 47 (see Figure~\ref{fig:game_example_move47} onward), black has no more available moves and must therefore pass their turn.\\

% La partie est gagnée par le joueur blanc 32 à 24, ce qui est un très bon score car le jeu Yolah me semble être à l'avantage des noirs.

The game is won by the white player 32 to 24, which is a very good score because the Yolah game seems to me to favor black.

\begin{figure}[htpb] %H
\centering
\begin{subfigure}[b]{0.24\textwidth}
    \centering
    \includegraphics[width=\textwidth]{board.png}
    % \caption{Position initiale}
    \caption{Initial position}
    \label{fig:game_example_init}
\end{subfigure}
\hfill
\begin{subfigure}[b]{0.24\textwidth}
    \centering
    \includegraphics[width=\textwidth]{yolah_game_example_move1.png}
    % \caption{Coup 1 (Noir)}
    \caption{Move 1 (Black)}
    \label{fig:game_example_move1}
\end{subfigure}
\hfill
\begin{subfigure}[b]{0.24\textwidth}
    \centering
    \includegraphics[width=\textwidth]{yolah_game_example_move2.png}
    % \caption{Coup 2 (Blanc)}
    \caption{Move 2 (White)}
    \label{fig:game_example_move2}
\end{subfigure}
\hfill
\begin{subfigure}[b]{0.24\textwidth}
    \centering
    \includegraphics[width=\textwidth]{yolah_game_example_move3.png}
    % \caption{Coup 3 (Noir)}
    \caption{Move 3 (Black)}
    \label{fig:game_example_move3}
\end{subfigure}\\

\begin{subfigure}[b]{0.24\textwidth}
    \centering
    \includegraphics[width=\textwidth]{yolah_game_example_move4.png}
    % \caption{Coup 4 (Blanc)}
    \caption{Move 4 (White)}
    \label{fig:game_example_move4}
\end{subfigure}
\hfill
\begin{subfigure}[b]{0.24\textwidth}
    \centering
    \includegraphics[width=\textwidth]{yolah_game_example_move5.png}
    % \caption{Coup 5 (Noir)}
    \caption{Move 5 (Black)}
    \label{fig:game_example_move5}
\end{subfigure}
\hfill
\begin{subfigure}[b]{0.24\textwidth}
    \centering
    \includegraphics[width=\textwidth]{yolah_game_example_move6.png}
    % \caption{Coup 6 (Blanc)}
    \caption{Move 6 (White)}
    \label{fig:game_example_move6}
\end{subfigure}
\hfill
\begin{subfigure}[b]{0.24\textwidth}
    \centering
    \includegraphics[width=\textwidth]{yolah_game_example_move7.png}
    % \caption{Coup 7 (Noir)}
    \caption{Move 7 (Black)}
    \label{fig:game_example_move7}
\end{subfigure}\\

\begin{subfigure}[b]{0.24\textwidth}
    \centering
    \includegraphics[width=\textwidth]{yolah_game_example_move8.png}
    % \caption{Coup 8 (Blanc)}
    \caption{Move 8 (White)}
    \label{fig:game_example_move8}
\end{subfigure}
\hfill
\begin{subfigure}[b]{0.24\textwidth}
    \centering
    \includegraphics[width=\textwidth]{yolah_game_example_move9.png}
    % \caption{Coup 9 (Noir)}
    \caption{Move 9 (Black)}
    \label{fig:game_example_move9}
\end{subfigure}
\hfill
\begin{subfigure}[b]{0.24\textwidth}
    \centering
    \includegraphics[width=\textwidth]{yolah_game_example_move10.png}
    % \caption{Coup 10 (Blanc)}
    \caption{Move 10 (White)}
    \label{fig:game_example_move10}
\end{subfigure}
\hfill
\begin{subfigure}[b]{0.24\textwidth}
    \centering
    \includegraphics[width=\textwidth]{yolah_game_example_move11.png}
    % \caption{Coup 11 (Noir)}
    \caption{Move 11 (Black)}
    \label{fig:game_example_move11}
\end{subfigure}\\

\begin{subfigure}[b]{0.24\textwidth}
    \centering
    \includegraphics[width=\textwidth]{yolah_game_example_move12.png}
    % \caption{Coup 12 (Blanc)}
    \caption{Move 12 (White)}
    \label{fig:game_example_move12}
\end{subfigure}
\hfill
\begin{subfigure}[b]{0.24\textwidth}
    \centering
    \includegraphics[width=\textwidth]{yolah_game_example_move13.png}
    % \caption{Coup 13 (Noir)}
    \caption{Move 13 (Black)}
    \label{fig:game_example_move13}
\end{subfigure}
\hfill
\begin{subfigure}[b]{0.24\textwidth}
    \centering
    \includegraphics[width=\textwidth]{yolah_game_example_move14.png}
    % \caption{Coup 14 (Blanc)}
    \caption{Move 14 (White)}
    \label{fig:game_example_move14}
\end{subfigure}
\hfill
\begin{subfigure}[b]{0.24\textwidth}
    \centering
    \includegraphics[width=\textwidth]{yolah_game_example_move15.png}
    % \caption{Coup 15 (Noir)}
    \caption{Move 15 (Black)}
    \label{fig:game_example_move15}
\end{subfigure}\\

\begin{subfigure}[b]{0.24\textwidth}
    \centering
    \includegraphics[width=\textwidth]{yolah_game_example_move16.png}
    % \caption{Coup 16 (Blanc)}
    \caption{Move 16 (White)}
    \label{fig:game_example_move16}
\end{subfigure}
\hfill
\begin{subfigure}[b]{0.24\textwidth}
    \centering
    \includegraphics[width=\textwidth]{yolah_game_example_move17.png}
    % \caption{Coup 17 (Noir)}
    \caption{Move 17 (Black)}
    \label{fig:game_example_move17}
\end{subfigure}
\hfill
\begin{subfigure}[b]{0.24\textwidth}
    \centering
    \includegraphics[width=\textwidth]{yolah_game_example_move18.png}
    % \caption{Coup 18 (Blanc)}
    \caption{Move 18 (White)}
    \label{fig:game_example_move18}
\end{subfigure}
\hfill
\begin{subfigure}[b]{0.24\textwidth}
    \centering
    \includegraphics[width=\textwidth]{yolah_game_example_move19.png}
    % \caption{Coup 19 (Noir)}
    \caption{Move 19 (Black)}
    \label{fig:game_example_move19}
\end{subfigure}

% \caption{Example de partie entre deux IA - coups 1 à 19}
\caption{Game example between two AIs - moves 1 to 19}
\label{fig:game_example_opening}
\end{figure}

\begin{figure}[htpb]
\centering

\begin{subfigure}[b]{0.24\textwidth}
    \centering
    \includegraphics[width=\textwidth]{yolah_game_example_move20.png}
    % \caption{Coup 20 (Blanc)}
    \caption{Move 20 (White)}
    \label{fig:game_example_move20}
\end{subfigure}
\hfill
\begin{subfigure}[b]{0.24\textwidth}
    \centering
    \includegraphics[width=\textwidth]{yolah_game_example_move21.png}
    % \caption{Coup 21 (Noir)}
    \caption{Move 21 (Black)}
    \label{fig:game_example_move21}
\end{subfigure}
\hfill
\begin{subfigure}[b]{0.24\textwidth}
    \centering
    \includegraphics[width=\textwidth]{yolah_game_example_move22.png}
    % \caption{Coup 22 (Blanc)}
    \caption{Move 22 (White)}
    \label{fig:game_example_move22}
\end{subfigure}
\hfill
\begin{subfigure}[b]{0.24\textwidth}
    \centering
    \includegraphics[width=\textwidth]{yolah_game_example_move23.png}
    % \caption{Coup 23 (Noir)}
    \caption{Move 23 (Black)}
    \label{fig:game_example_move23}
\end{subfigure}\\

\begin{subfigure}[b]{0.24\textwidth}
    \centering
    \includegraphics[width=\textwidth]{yolah_game_example_move24.png}
    % \caption{Coup 24 (Blanc)}
    \caption{Move 24 (White)}
    \label{fig:game_example_move24}
\end{subfigure}
\hfill
\begin{subfigure}[b]{0.24\textwidth}
    \centering
    \includegraphics[width=\textwidth]{yolah_game_example_move25.png}
    % \caption{Coup 25 (Noir)}
    \caption{Move 25 (Black)}
    \label{fig:game_example_move25}
\end{subfigure}
\hfill
\begin{subfigure}[b]{0.24\textwidth}
    \centering
    \includegraphics[width=\textwidth]{yolah_game_example_move26.png}
    % \caption{Coup 26 (Blanc)}
    \caption{Move 26 (White)}
    \label{fig:game_example_move26}
\end{subfigure}
\hfill
\begin{subfigure}[b]{0.24\textwidth}
    \centering
    \includegraphics[width=\textwidth]{yolah_game_example_move27.png}
    % \caption{Coup 27 (Noir)}
    \caption{Move 27 (Black)}
    \label{fig:game_example_move27}
\end{subfigure}\\

\begin{subfigure}[b]{0.24\textwidth}
    \centering
    \includegraphics[width=\textwidth]{yolah_game_example_move28.png}
    % \caption{Coup 28 (Blanc)}
    \caption{Move 28 (White)}
    \label{fig:game_example_move28}
\end{subfigure}
\hfill
\begin{subfigure}[b]{0.24\textwidth}
    \centering
    \includegraphics[width=\textwidth]{yolah_game_example_move29.png}
    % \caption{Coup 29 (Noir)}
    \caption{Move 29 (Black)}
    \label{fig:game_example_move29}
\end{subfigure}
\hfill
\begin{subfigure}[b]{0.24\textwidth}
    \centering
    \includegraphics[width=\textwidth]{yolah_game_example_move30.png}
    % \caption{Coup 30 (Blanc)}
    \caption{Move 30 (White)}
    \label{fig:game_example_move30}
\end{subfigure}
\hfill
\begin{subfigure}[b]{0.24\textwidth}
    \centering
    \includegraphics[width=\textwidth]{yolah_game_example_move31.png}
    % \caption{Coup 31 (Noir)}
    \caption{Move 31 (Black)}
    \label{fig:game_example_move31}
\end{subfigure}\\

\begin{subfigure}[b]{0.24\textwidth}
    \centering
    \includegraphics[width=\textwidth]{yolah_game_example_move32.png}
    % \caption{Coup 32 (Blanc)}
    \caption{Move 32 (White)}
    \label{fig:game_example_move32}
\end{subfigure}
\hfill
\begin{subfigure}[b]{0.24\textwidth}
    \centering
    \includegraphics[width=\textwidth]{yolah_game_example_move33.png}
    % \caption{Coup 33 (Noir)}
    \caption{Move 33 (Black)}
    \label{fig:game_example_move33}
\end{subfigure}
\hfill
\begin{subfigure}[b]{0.24\textwidth}
    \centering
    \includegraphics[width=\textwidth]{yolah_game_example_move34.png}
    % \caption{Coup 34 (Blanc)}
    \caption{Move 34 (White)}
    \label{fig:game_example_move34}
\end{subfigure}
\hfill
\begin{subfigure}[b]{0.24\textwidth}
    \centering
    \includegraphics[width=\textwidth]{yolah_game_example_move35.png}
    % \caption{Coup 35 (Noir)}
    \caption{Move 35 (Black)}
    \label{fig:game_example_move35}
\end{subfigure}\\

\begin{subfigure}[b]{0.24\textwidth}
    \centering
    \includegraphics[width=\textwidth]{yolah_game_example_move36.png}
    % \caption{Coup 36 (Blanc)}
    \caption{Move 36 (White)}
    \label{fig:game_example_move36}
\end{subfigure}
\hfill
\begin{subfigure}[b]{0.24\textwidth}
    \centering
    \includegraphics[width=\textwidth]{yolah_game_example_move37.png}
    % \caption{Coup 37 (Noir)}
    \caption{Move 37 (Black)}
    \label{fig:game_example_move37}
\end{subfigure}
\hfill
\begin{subfigure}[b]{0.24\textwidth}
    \centering
    \includegraphics[width=\textwidth]{yolah_game_example_move38.png}
    % \caption{Coup 38 (Blanc)}
    \caption{Move 38 (White)}
    \label{fig:game_example_move38}
\end{subfigure}
\hfill
\begin{subfigure}[b]{0.24\textwidth}
    \centering
    \includegraphics[width=\textwidth]{yolah_game_example_move39.png}
    % \caption{Coup 39 (Noir)}
    \caption{Move 39 (Black)}
    \label{fig:game_example_move39}
\end{subfigure}\\

% \caption{Example de partie entre deux IA - coups 20 à 39}
\caption{Game example between two AIs - moves 20 to 39}
\label{fig:game_example_middle}
\end{figure}


\begin{figure}[htpb]
\centering

\begin{subfigure}[b]{0.24\textwidth}
    \centering
    \includegraphics[width=\textwidth]{yolah_game_example_move40.png}
    % \caption{Coup 40 (Blanc)}
    \caption{Move 40 (White)}
    \label{fig:game_example_move40}
\end{subfigure}
\hfill
\begin{subfigure}[b]{0.24\textwidth}
    \centering
    \includegraphics[width=\textwidth]{yolah_game_example_move41.png}
    % \caption{Coup 41 (Noir)}
    \caption{Move 41 (Black)}
    \label{fig:game_example_move41}
\end{subfigure}
\hfill
\begin{subfigure}[b]{0.24\textwidth}
    \centering
    \includegraphics[width=\textwidth]{yolah_game_example_move42.png}
    % \caption{Coup 42 (Blanc)}
    \caption{Move 42 (White)}
    \label{fig:game_example_move42}
\end{subfigure}
\hfill
\begin{subfigure}[b]{0.24\textwidth}
    \centering
    \includegraphics[width=\textwidth]{yolah_game_example_move43.png}
    % \caption{Coup 43 (Noir)}
    \caption{Move 43 (Black)}
    \label{fig:game_example_move43}
\end{subfigure}\\

\begin{subfigure}[b]{0.24\textwidth}
    \centering
    \includegraphics[width=\textwidth]{yolah_game_example_move44.png}
    % \caption{Coup 44 (Blanc)}
    \caption{Move 44 (White)}
    \label{fig:game_example_move44}
\end{subfigure}
\hfill
\begin{subfigure}[b]{0.24\textwidth}
    \centering
    \includegraphics[width=\textwidth]{yolah_game_example_move45.png}
    % \caption{Coup 45 (Noir)}
    \caption{Move 45 (Black)}
    \label{fig:game_example_move45}
\end{subfigure}
\hfill
\begin{subfigure}[b]{0.24\textwidth}
    \centering
    \includegraphics[width=\textwidth]{yolah_game_example_move46.png}
    % \caption{Coup 46 (Blanc)}
    \caption{Move 46 (White)}
    \label{fig:game_example_move46}
\end{subfigure}
\hfill
\begin{subfigure}[b]{0.24\textwidth}
    \centering
    \includegraphics[width=\textwidth]{yolah_game_example_move47.png}
    % \caption{Coup 47 (Noir)}
    \caption{Move 47 (Black)}
    \label{fig:game_example_move47}
\end{subfigure}\\

\begin{subfigure}[b]{0.24\textwidth}
    \centering
    \includegraphics[width=\textwidth]{yolah_game_example_move48.png}
    % \caption{Coup 48 (Blanc)}
    \caption{Move 48 (White)}
    \label{fig:game_example_move48}
\end{subfigure}
\hfill
\begin{subfigure}[b]{0.24\textwidth}
    \centering
    \includegraphics[width=\textwidth]{yolah_game_example_move49.png}
    % \caption{Coup 49 (Blanc)}
    \caption{Move 49 (White)}
    \label{fig:game_example_move49}
\end{subfigure}
\hfill
\begin{subfigure}[b]{0.24\textwidth}
    \centering
    \includegraphics[width=\textwidth]{yolah_game_example_move50.png}
    % \caption{Coup 50 (Blanc)}
    \caption{Move 50 (White)}
    \label{fig:game_example_move50}
\end{subfigure}
\hfill
\begin{subfigure}[b]{0.24\textwidth}
    \centering
    \includegraphics[width=\textwidth]{yolah_game_example_move51.png}
    % \caption{Coup 51 (Blanc)}
    \caption{Move 51 (White)}
    \label{fig:game_example_move51}
\end{subfigure}\\

\begin{subfigure}[b]{0.24\textwidth}
    \centering
    \includegraphics[width=\textwidth]{yolah_game_example_move52.png}
    % \caption{Coup 52 (Blanc)}
    \caption{Move 52 (White)}
    \label{fig:game_example_move52}
\end{subfigure}
\hfill
\begin{subfigure}[b]{0.24\textwidth}
    \centering
    \includegraphics[width=\textwidth]{yolah_game_example_move53.png}
    % \caption{Coup 53 (Blanc)}
    \caption{Move 53 (White)}
    \label{fig:game_example_move53}
\end{subfigure}
\hfill
\begin{subfigure}[b]{0.24\textwidth}
    \centering
    \includegraphics[width=\textwidth]{yolah_game_example_move54.png}
    % \caption{Coup 54 (Blanc)}
    \caption{Move 54 (White)}
    \label{fig:game_example_move54}
\end{subfigure}
\hfill
\begin{subfigure}[b]{0.24\textwidth}
    \centering
    \includegraphics[width=\textwidth]{yolah_game_example_move55.png}
    % \caption{Coup 55 (Blanc)}
    \caption{Move 55 (White)}
    \label{fig:game_example_move55}
\end{subfigure}\\

\begin{subfigure}[b]{0.24\textwidth}
    \centering
    \includegraphics[width=\textwidth]{yolah_game_example_move56.png}
    % \caption{Coup 56 (Blanc)}
    \caption{Move 56 (White)}
    \label{fig:game_example_move56}
\end{subfigure}
\hfill

% \caption{Example de partie entre deux IA - coups 39 à 56. Blanc gagne 32 à 24}
\caption{Game example between two AIs - moves 40 to 56. White wins 32 to 24}
\label{fig:game_example_end}
\end{figure}

% \subsection{La suite}
\subsection{What's Next}

% \subsection{Why Yolah for AI Research?}

% Yolah provides several advantages as a platform for AI development:

% \begin{itemize}
%     \item \textbf{Moderate complexity}: Complex enough to be interesting, simple enough to be tractable
%     \item \textbf{Clear evaluation}: Position quality can be measured objectively
%     \item \textbf{Strategic depth}: Multiple viable strategies and playing styles
%     \item \textbf{Computational feasibility}: Playable with various AI techniques
% \end{itemize}

% \section{Overview of Game AI}

% Game-playing has been a cornerstone of artificial intelligence research since the field's inception. From early checkers programs to modern systems that master Go and Chess, game AI has driven innovation in search, learning, and decision-making.

% \subsection{Historical Context}

% The evolution of game AI can be traced through several key milestones:

% \begin{table}[H]
% \centering
% \begin{tabular}{lp{8cm}}
% \toprule
% \textbf{Year} & \textbf{Achievement} \\
% \midrule
% 1950s & Early checkers programs \\
% 1970s & Chess programs reach amateur level \\
% 1997 & Deep Blue defeats world champion Garry Kasparov \\
% 2016 & AlphaGo defeats world champion Lee Sedol \\
% 2017 & AlphaZero masters Chess, Go, and Shogi through self-play \\
% \bottomrule
% \end{tabular}
% \caption{Key milestones in game-playing AI}
% \label{tab:milestones}
% \end{table}

% \subsection{Categories of Game AI Techniques}

% Game AI techniques can be broadly categorized as:

% \begin{description}
%     \item[Search-based methods] Use game tree exploration (Minimax, Alpha-Beta)
%     \item[Sampling methods] Use Monte Carlo simulations (MCTS)
%     \item[Learning-based methods] Use neural networks and reinforcement learning
%     \item[Hybrid approaches] Combine multiple techniques for optimal performance
% \end{description}

% \section{Book Structure}

% This book is organized into three main parts:

% \textbf{Part I: Foundations} covers the basic game engine, data structures, and move generation.

% \textbf{Part II: Classical Algorithms} explores traditional search-based AI techniques.

% \textbf{Part III: Modern Approaches} delves into neural networks and learning methods.

% Each chapter builds upon previous concepts, with complete code examples and performance analysis.

\chapter{Game Engine}
\label{ch:engine}

% \section{Core Data Structures}

% The foundation of any game engine lies in its data structures. For Yolah, we need efficient representations for:

% \begin{itemize}
%     \item Board state
%     \item Move history
%     \item Player positions
%     \item Game metadata
% \end{itemize}

% \subsection{Board Representation}

% Here's our basic board representation in Python:

% \begin{listing}[H]
% \begin{minted}[linenos, bgcolor=codebg]{python}
% class Board:
%     """Represents the Yolah game board state."""

%     def __init__(self, size=8):
%         self.size = size
%         self.grid = [[None for _ in range(size)]
%                      for _ in range(size)]
%         self.current_player = 1
%         self.move_count = 0

%     def make_move(self, move):
%         """Apply a move to the board."""
%         row, col = move.position
%         self.grid[row][col] = self.current_player
%         self.current_player = -self.current_player
%         self.move_count += 1

%     def is_valid_move(self, move):
%         """Check if a move is legal."""
%         row, col = move.position
%         if not (0 <= row < self.size and 0 <= col < self.size):
%             return False
%         return self.grid[row][col] is None
% \end{minted}
% \caption{Basic board representation in Yolah}
% \label{code:board}
% \end{listing}

% \begin{importantbox}
% Efficient board representation is critical for performance. Every AI algorithm will access board state thousands or millions of times during search.
% \end{importantbox}

% \section{Move Generation}

% Move generation is the process of finding all legal moves from a given position. This is a critical component that affects both correctness and performance.

% \subsection{Algorithm}

% The move generation algorithm follows this structure:

% \begin{algorithmbox}
% \begin{algorithmic}[1]
% \Procedure{GenerateMoves}{$board$}
%     \State $moves \gets []$
%     \For{each position $(row, col)$ on $board$}
%         \If{position is empty}
%             \State $move \gets \textsc{CreateMove}(row, col)$
%             \If{$move$ is legal}
%                 \State $moves.\textsc{append}(move)$
%             \EndIf
%         \EndIf
%     \EndFor
%     \State \Return $moves$
% \EndProcedure
% \end{algorithmic}
% \end{algorithmbox}

% \section{Performance Considerations}

% When building a game engine, performance is paramount. Consider these key metrics:

% \begin{table}[H]
% \centering
% \begin{tabular}{lcc}
% \toprule
% \textbf{Operation} & \textbf{Target Time} & \textbf{Frequency} \\
% \midrule
% Move generation & $< 1$ ms & Every node \\
% Make/unmake move & $< 0.1$ ms & Every node \\
% Position evaluation & $< 5$ ms & Leaf nodes \\
% Legal move check & $< 0.01$ ms & Very frequent \\
% \bottomrule
% \end{tabular}
% \caption{Performance targets for engine operations}
% \label{tab:performance}
% \end{table}

% \begin{resultbox}
% Our optimized engine achieves over 100,000 nodes per second on modern hardware, enabling deep search within reasonable time constraints.
% \end{resultbox}


\chapter{AI Players}
\label{ch:ai}

% \section{Introduction to Minimax}

% The Minimax algorithm is the foundation of game-playing AI. It assumes both players play optimally and searches the game tree to find the best move.

% \subsection{Basic Concept}

% The algorithm alternates between:
% \begin{itemize}
%     \item \textbf{Maximizing} player tries to maximize score
%     \item \textbf{Minimizing} player tries to minimize score
% \end{itemize}

% This creates a mathematical framework for two-player zero-sum games.

% \section{Implementation}

% Here's a complete Minimax implementation:

% \begin{listing}[H]
% \begin{minted}[linenos, bgcolor=codebg]{python}
% def minimax(board, depth, maximizing_player):
%     """
%     Standard Minimax algorithm implementation.

%     Args:
%         board: Current game state
%         depth: Remaining search depth
%         maximizing_player: True if maximizing, False if minimizing

%     Returns:
%         Best evaluation score for this position
%     """
%     # Base case: terminal position or depth limit
%     if depth == 0 or board.is_game_over():
%         return evaluate(board)

%     if maximizing_player:
%         max_eval = float('-inf')
%         for move in board.generate_moves():
%             board.make_move(move)
%             eval = minimax(board, depth - 1, False)
%             board.undo_move()
%             max_eval = max(max_eval, eval)
%         return max_eval
%     else:
%         min_eval = float('inf')
%         for move in board.generate_moves():
%             board.make_move(move)
%             eval = minimax(board, depth - 1, True)
%             board.undo_move()
%             min_eval = min(min_eval, eval)
%         return min_eval
% \end{minted}
% \caption{Minimax algorithm implementation}
% \label{code:minimax}
% \end{listing}

% \section{Complexity Analysis}

% The time complexity of Minimax is:

% \begin{equation}
%     T(d) = O(b^d)
%     \label{eq:minimax_complexity}
% \end{equation}

% where $b$ is the branching factor and $d$ is the search depth.

% For Yolah with an average branching factor of 30:
% \begin{align}
%     \text{Nodes at depth 1} &= 30 \\
%     \text{Nodes at depth 2} &= 30^2 = 900 \\
%     \text{Nodes at depth 4} &= 30^4 = 810,000 \\
%     \text{Nodes at depth 6} &= 30^6 = 729,000,000
% \end{align}

% \begin{importantbox}
% The exponential growth of nodes makes deep search impractical without optimization techniques like Alpha-Beta pruning.
% \end{importantbox}

% %%%%%%%%%%%%%%%%%%%%%%%%%%%%%%%%%%%%%%%%%%%%%%%%%%%%%%%%%%%%%%%%%%%%%%%%%
% % EXAMPLE FIGURE
% %%%%%%%%%%%%%%%%%%%%%%%%%%%%%%%%%%%%%%%%%%%%%%%%%%%%%%%%%%%%%%%%%%%%%%%%%

% \section{Visualizing Minimax}

% Figure~\ref{fig:minimax_tree} shows how Minimax explores the game tree.

% \begin{figure}[H]
% \centering
% \begin{tikzpicture}[
%     level distance=1.5cm,
%     level 1/.style={sibling distance=4cm},
%     level 2/.style={sibling distance=2cm},
%     every node/.style={circle, draw, minimum size=0.8cm}
% ]
% \node {Max}
%     child {node {Min}
%         child {node {5}}
%         child {node {3}}
%     }
%     child {node {Min}
%         child {node {7}}
%         child {node {2}}
%     }
%     child {node {Min}
%         child {node {4}}
%         child {node {6}}
%     };
% \end{tikzpicture}
% \caption{Minimax game tree example. The maximizing player chooses the move leading to the highest value.}
% \label{fig:minimax_tree}
% \end{figure}

% %%%%%%%%%%%%%%%%%%%%%%%%%%%%%%%%%%%%%%%%%%%%%%%%%%%%%%%%%%%%%%%%%%%%%%%%%
% % ADDITIONAL CHAPTERS
% %%%%%%%%%%%%%%%%%%%%%%%%%%%%%%%%%%%%%%%%%%%%%%%%%%%%%%%%%%%%%%%%%%%%%%%%%

% \chapter{Alpha-Beta Pruning}
% \label{ch:alphabeta}

% Alpha-Beta pruning dramatically improves Minimax by eliminating branches that cannot affect the final decision.

% \section{The Pruning Principle}

% Your content here...

\chapter{Monte Carlo Player}
\label{ch:monte_carlo}

% MCTS revolutionized game AI by using random simulations instead of exhaustive search.

% \section{UCB1 Formula}

% The Upper Confidence Bound formula balances exploration and exploitation:

% \begin{equation}
%     UCB1 = \frac{w_i}{n_i} + c\sqrt{\frac{\ln N}{n_i}}
%     \label{eq:ucb1}
% \end{equation}

% where:
% \begin{itemize}
%     \item $w_i$ = wins for node $i$
%     \item $n_i$ = visits to node $i$
%     \item $N$ = total visits to parent
%     \item $c$ = exploration constant
% \end{itemize}

% Continue with more chapters...

\chapter{MCTS Player}
\label{ch:mcts}

% Your content here...

\chapter{Minmax Player}
\label{ch:minmax}

% Your content here...

\chapter{Minmax with Neural Network Player}
\label{ch:nnue}

% Your content here...

\chapter{AI Tournament}
\label{ch:tournament}

% Your content here...

\chapter{Conclusion}
\label{ch:conclusion}

% This book has taken you on a journey from classical game-playing algorithms to modern neural network approaches. The Yolah engine demonstrates that combining multiple techniques yields the best results.

% \section{Key Takeaways}

% \begin{enumerate}
%     \item Classical algorithms remain relevant and effective
%     \item Neural networks provide powerful evaluation functions
%     \item Hybrid approaches leverage the strengths of multiple methods
%     \item Performance optimization is critical for practical systems
% \end{enumerate}

% \section{Future Directions}

% Promising areas for future research include:
% \begin{itemize}
%     \item More efficient neural architectures
%     \item Better exploration strategies
%     \item Transfer learning across game domains
%     \item Real-time learning and adaptation
% \end{itemize}

%%%%%%%%%%%%%%%%%%%%%%%%%%%%%%%%%%%%%%%%%%%%%%%%%%%%%%%%%%%%%%%%%%%%%%%%%
% BACK MATTER
%%%%%%%%%%%%%%%%%%%%%%%%%%%%%%%%%%%%%%%%%%%%%%%%%%%%%%%%%%%%%%%%%%%%%%%%%

\backmatter

% Appendix
\appendix

% \chapter{Installation Guide}
% \label{app:installation}

% \section{Requirements}

% To run the Yolah engine, you need:
% \begin{itemize}
%     \item Python 3.8 or higher
%     \item NumPy
%     \item TensorFlow or PyTorch
%     \item Additional dependencies listed in requirements.txt
% \end{itemize}

% \section{Setup Instructions}

% \begin{minted}[bgcolor=codebg]{bash}
% git clone https://github.com/yourusername/yolah.git
% cd yolah
% pip install -r requirements.txt
% python setup.py install
% \end{minted}

% Bibliography
\printbibliography[heading=bibintoc]

% Index (if needed)
% \printindex

\end{document}
