%%%%%%%%%%%%%%%%%%%%%%%%%%%%%%%%%%%%%%%%%%%%%%%%%%%%%%%%%%%%%%%%%%%%%%%%%
%
% Complete Book Template - Yolah Game AI Engine
%
% A comprehensive single-file template for writing a technical book
% This template includes all necessary packages, custom environments,
% and example content to get you started quickly.
%
% USAGE:
% 1. Copy this file: cp book_template.tex book.tex
% 2. Edit book.tex with your content
% 3. Compile: make pdf (or pdflatex --shell-escape book.tex)
%
%%%%%%%%%%%%%%%%%%%%%%%%%%%%%%%%%%%%%%%%%%%%%%%%%%%%%%%%%%%%%%%%%%%%%%%%%

% you were helping me translating the book from french to english. I write in french because it is easier for me and you correct my french, put it in comments and translate into english


\documentclass[12pt,a4paper,twoside,openright]{book}

%%%%%%%%%%%%%%%%%%%%%%%%%%%%%%%%%%%%%%%%%%%%%%%%%%%%%%%%%%%%%%%%%%%%%%%%%
% PACKAGES
%%%%%%%%%%%%%%%%%%%%%%%%%%%%%%%%%%%%%%%%%%%%%%%%%%%%%%%%%%%%%%%%%%%%%%%%%

% Encoding and Language
\usepackage[utf8]{inputenc}
\usepackage[T1]{fontenc}
\usepackage[english]{babel}

% Page Layout
\usepackage[
    top=1in,
    bottom=1in,
    left=1.25in,
    right=1in,
    headheight=15pt
]{geometry}

% Math
\usepackage{amsmath}
\usepackage{amssymb}
\usepackage{amsthm}
\usepackage{mathtools}

% Graphics and Colors
\usepackage{graphicx}
\graphicspath{{figures/}}
\usepackage{xcolor}
\usepackage{tikz}
\usetikzlibrary{arrows.meta, positioning, shapes.geometric, calc}

% Tables
\usepackage{booktabs}
\usepackage{array}
\usepackage{multirow}
\usepackage{longtable}
\usepackage{colortbl}

% Code Listings
\usepackage{minted}
\usemintedstyle{friendly}
\setminted{
    breaklines,
    fontsize=\small,
    frame=lines,
    framesep=2mm,
    baselinestretch=1.2,
    tabsize=4
}
\usepackage{mdframed}

% Define code background color
\definecolor{codebg}{RGB}{248,248,248}

% Algorithms
\usepackage{algorithm}
\usepackage{algpseudocode}

% Bibliography
\usepackage[
    backend=biber,
    style=numeric,
    sorting=none,
    maxbibnames=99
]{biblatex}
\addbibresource{references.bib}

% Glossary
\usepackage[acronym,toc]{glossaries}
\makeglossaries

% Hyperlinks
\usepackage{hyperref}
\hypersetup{
    colorlinks=true,
    linkcolor=blue!50!black,
    citecolor=green!50!black,
    urlcolor=blue!50!black,
    bookmarksdepth=3,
    pdfstartview=FitH
}

% Headers and Footers
\usepackage{fancyhdr}
\pagestyle{fancy}
\fancyhf{}
\fancyhead[LE]{\leftmark}
\fancyhead[RO]{\rightmark}
\fancyfoot[C]{\thepage}
\renewcommand{\headrulewidth}{0.4pt}

% Other Useful Packages
\usepackage{float}              % Better float control
\usepackage{caption}            % Better captions
\captionsetup[listing]{skip=2pt} % Reduce space between listing and caption
\usepackage{subcaption}         % Subfigures
\usepackage{enumitem}           % Better lists
\usepackage{footnote}           % Better footnotes
\usepackage{csquotes}           % Quotations

%%%%%%%%%%%%%%%%%%%%%%%%%%%%%%%%%%%%%%%%%%%%%%%%%%%%%%%%%%%%%%%%%%%%%%%%%
% CUSTOM COLORS
%%%%%%%%%%%%%%%%%%%%%%%%%%%%%%%%%%%%%%%%%%%%%%%%%%%%%%%%%%%%%%%%%%%%%%%%%

\definecolor{boxborder}{RGB}{52,101,164}
\definecolor{boxbg}{RGB}{232,241,250}
\definecolor{algoborder}{RGB}{46,125,50}
\definecolor{algobg}{RGB}{232,245,233}
\definecolor{resultborder}{RGB}{183,28,28}
\definecolor{resultbg}{RGB}{255,235,238}

%%%%%%%%%%%%%%%%%%%%%%%%%%%%%%%%%%%%%%%%%%%%%%%%%%%%%%%%%%%%%%%%%%%%%%%%%
% CUSTOM ENVIRONMENTS
%%%%%%%%%%%%%%%%%%%%%%%%%%%%%%%%%%%%%%%%%%%%%%%%%%%%%%%%%%%%%%%%%%%%%%%%%

% Important Box
\usepackage{tcolorbox}
\tcbuselibrary{most}

\newtcolorbox{importantbox}{
    colback=boxbg,
    colframe=boxborder,
    fonttitle=\bfseries,
    title=Important,
    boxrule=1.5pt,
    arc=3pt
}

% Algorithm Box
\newtcolorbox{algorithmbox}{
    colback=algobg,
    colframe=algoborder,
    fonttitle=\bfseries,
    title=Algorithm,
    boxrule=1.5pt,
    arc=3pt
}

% Result Box
\newtcolorbox{resultbox}{
    colback=resultbg,
    colframe=resultborder,
    fonttitle=\bfseries,
    title=Key Result,
    boxrule=1.5pt,
    arc=3pt
}

\usepackage{wasysym}
\usepackage{fontawesome}
\usepackage{hhline}

%%%%%%%%%%%%%%%%%%%%%%%%%%%%%%%%%%%%%%%%%%%%%%%%%%%%%%%%%%%%%%%%%%%%%%%%%
% DOCUMENT INFORMATION
%%%%%%%%%%%%%%%%%%%%%%%%%%%%%%%%%%%%%%%%%%%%%%%%%%%%%%%%%%%%%%%%%%%%%%%%%

\title{
    \Huge\textbf{Yolah Board Game} \\
    \vspace{0.5cm}
    \Large Building a Two-Player Perfect-Information Game with AI Players\\ 
}

\author{
    \Large Pascal Garcia \\
    % \vspace{0.3cm}
    % \normalsize INSA Rennes
}

\date{\today}

%%%%%%%%%%%%%%%%%%%%%%%%%%%%%%%%%%%%%%%%%%%%%%%%%%%%%%%%%%%%%%%%%%%%%%%%%
% GLOSSARY ENTRIES
%%%%%%%%%%%%%%%%%%%%%%%%%%%%%%%%%%%%%%%%%%%%%%%%%%%%%%%%%%%%%%%%%%%%%%%%%

% Acronyms
\newacronym{smt}{SMT}{Satisfiability Modulo Theories}
\newacronym{ai}{AI}{Artificial Intelligence}
\newacronym{cpu}{CPU}{Central Processing Unit}
\newacronym{gpu}{GPU}{Graphics Processing Unit}
\newacronym{z3}{Z3}{Z3 Theorem Prover}

% Terms
\newglossaryentry{bitboard}{
    name=bitboard,
    description={A data structure using bit arrays to represent game board positions, where each bit corresponds to a square on the board}
}

\newglossaryentry{magicbitboard}{
    name=magic bitboard,
    description={A technique using specially chosen multiplication constants (magic numbers) and bit shifting to create a minimal perfect hash for efficiently mapping occupied squares to move lookup tables}
}

%%%%%%%%%%%%%%%%%%%%%%%%%%%%%%%%%%%%%%%%%%%%%%%%%%%%%%%%%%%%%%%%%%%%%%%%%
% BEGIN DOCUMENT
%%%%%%%%%%%%%%%%%%%%%%%%%%%%%%%%%%%%%%%%%%%%%%%%%%%%%%%%%%%%%%%%%%%%%%%%%

\begin{document}

%%%%%%%%%%%%%%%%%%%%%%%%%%%%%%%%%%%%%%%%%%%%%%%%%%%%%%%%%%%%%%%%%%%%%%%%%
% FRONT MATTER
%%%%%%%%%%%%%%%%%%%%%%%%%%%%%%%%%%%%%%%%%%%%%%%%%%%%%%%%%%%%%%%%%%%%%%%%%

\frontmatter

% Title Page
\maketitle

\thispagestyle{empty}
\vspace*{\fill}
% \begin{center}
% \reflectbox{\copyright} \the\year{} Pascal Garcia
% \end{center}
\begin{center}
    \includegraphics[width=10cm]{logo.png}
\end{center}
\vspace*{\fill}
\clearpage


% Copyright Page
\thispagestyle{empty}
\vspace*{\fill}
% \begin{center}
% \reflectbox{\copyright} \the\year{} Pascal Garcia
% \end{center}
\begin{center}
{\ttfamily\Huge\textcolor{black}{C'est en forgeant qu'on devient forgeron}}
\end{center}
\vspace*{\fill}
\clearpage

% Dedication (Optional)
\thispagestyle{empty}
\vspace*{\fill}
\begin{center}
\textit{À Sar\textbf{ah}, \textbf{H}ug\textbf{o} et C\'e\textbf{ly}a \textcolor{red}{\faHeart}}\\
\textit{\textbf{ahholy}}\\
\textit{\textbf{aholy}}\\
\textit{\textbf{yolah}}\\
\end{center}
\vspace*{\fill}
\clearpage

% Table of Contents
\tableofcontents

% List of Figures
\listoffigures

% List of Tables
\listoftables

% Preface (Optional)
%\chapter{Preface}


%%%%%%%%%%%%%%%%%%%%%%%%%%%%%%%%%%%%%%%%%%%%%%%%%%%%%%%%%%%%%%%%%%%%%%%%%
% MAIN MATTER
%%%%%%%%%%%%%%%%%%%%%%%%%%%%%%%%%%%%%%%%%%%%%%%%%%%%%%%%%%%%%%%%%%%%%%%%%

\mainmatter

%%%%%%%%%%%%%%%%%%%%%%%%%%%%%%%%%%%%%%%%%%%%%%%%%%%%%%%%%%%%%%%%%%%%%%%%%
% CHAPTER 1: INTRODUCTION
%%%%%%%%%%%%%%%%%%%%%%%%%%%%%%%%%%%%%%%%%%%%%%%%%%%%%%%%%%%%%%%%%%%%%%%%%

\chapter{Introduction}
\label{ch:introduction}

\section{The Yolah Game}

% J'ai créé le jeu Yolah dans le but d'illustrer des techniques efficaces d'implémentation de jeux de plateaux
% et d'intelligences artificielles pour mes étudiants. Je me suis inspiré du jeu des pingouins dont vous pouvez voir la boîte dans la figure \ref{fig:pingouins} (je le conseille \smiley)

I created the Yolah game to illustrate effective techniques for implementing board games
and artificial intelligences for my students. I was inspired by the Pingouins game, whose box you can see in Figure~\ref{fig:pingouins} (I highly recommend it \smiley)

\begin{figure}[htpb]
\centering
\includegraphics[width=0.4\textwidth]{pingouins.png}
\caption{The Pingouins game box}
% \caption{La boîte du jeu des pingouins}
\label{fig:pingouins}
\end{figure}

\begin{importantbox}
% J'ai fait au mieux de mes connaissances actuelles (\emph{ars longa, vita brevis}) pour implémenter mon jeu et les IA associées. Mais comme tout bon scientifique, il faudra regarder mon travail avec un oeil critique.
I have done my best with my current knowledge (\emph{ars longa, vita brevis}) to implement my game and the associated AIs. But like any good scientist, you should look at my work with a critical eye.
% J'ai écrit le livre en français (plus facile pour moi) et j'ai demandé à un assistant (Claude) de le traduire pour moi.
I wrote the book in French (easier for me) and asked an AI assistant (Claude~\cite{anthropic2025claude}) to translate it for me.
\end{importantbox}

% Je vais maintenant décrire les règles du jeu, puis j'expliquerai pourquoi j'ai choisi ces règles, je donnerai un exemple de partie entre deux IA et je présenterai la suite du livre.
I will now describe the rules of the game, then I will explain why I chose these rules, I will give an example of a game between two AIs and then I will present the rest of the book.

\subsection{Game Rules}

\begin{figure}[htpb]
\centering
\includegraphics[width=0.7\textwidth]{yolah_initial_configuration.png}
\caption{The initial configuration of the Yolah game}
% \caption{La configuration initiale du jeu Yolah}
\label{fig:yolah_initial_configuration}
\end{figure}

% Le plateau du jeu Yolah est représenté dans la figure \ref{fig:yolah_initial_configuration}.
% On peut y voir quatre pions noirs et quatre pions blancs placés de manière symétrique. Les noirs commencent en choisissant une de leurs pièces parmi
% les quatre. Une pièce ne pourra jamais disparaître du plateau car Yolah est un jeu sans possibilité de captures. Une pièce se déplace dans les huit directions
% aussi loin qu'elle le souhaite tant qu'elle n'est pas bloquée par une autre pièce ou un trou (notion que nous allons bientôt aborder). Par exemple, si les noirs
% choisissent de déplacer leur pièce qui se situe en {\ttfamily d5}, les cases où elle peut atterrir sont indiquées par de petites croix noires dans la figure \ref{fig:yolah_moves1}.

The Yolah game board is shown in Figure~\ref{fig:yolah_initial_configuration}.
You can see four black pieces and four white pieces placed symmetrically. Black starts by choosing one of their four pieces.
A piece can never disappear from the board because Yolah is a game without captures. A piece moves in all eight directions
as far as it wishes as long as it is not blocked by another piece or a hole (a concept we will soon discuss). For example, if black
chooses to move their piece located at {\ttfamily d5}, the squares where it can land are indicated by small black crosses in Figure~\ref{fig:yolah_moves1}.\\

\begin{figure}[htpb]
\centering
\includegraphics[width=0.67\textwidth]{yolah_moves1.pdf}
% \caption{Mouvements possibles (petites croix noires) pour le pion noir situé sur la case {\ttfamily d5})}
\caption{Possible moves (small black crosses) for the black piece located on square {\ttfamily d5}}
\label{fig:yolah_moves1}
\end{figure}

% Maintenant, si le pion noir en {\ttfamily d5} se déplace en {\ttfamily b7}, ce que nous noterons par {\ttfamily d5:b7}, nous obtenons la configuration représentée dans la figure \ref{fig:yolah_moves2}. On peut remarquer que la case de départ du pion noir disparaît et devient un trou~! Cette case (ce trou) devient inaccessible et infranchissable pour le reste de la partie~! Ceci va créer des opportunités pour bloquer l'adversaire et essayer de créer des zones où l'adversaire ne pourra pas aller.

Now, if the black piece at {\ttfamily d5} moves to {\ttfamily b7}, which we will denote as {\ttfamily d5:b7}, we get the configuration shown in Figure~\ref{fig:yolah_moves2}. Notice that the starting square of the black piece disappears and becomes a hole! This square (this hole) becomes inaccessible and impassable for the rest of the game! This will create opportunities to block the opponent and try to create areas where the opponent cannot go.\\

\begin{figure}[htpb]
\centering
\includegraphics[width=0.6\textwidth]{yolah_moves2.png}
\caption{Black just moved from {\ttfamily d5} to {\ttfamily b7}. The starting square {\ttfamily d5} becomes inaccessible and impassable for the rest of the game}
% \caption{Noir vient de se déplacer de {\ttfamily d5} en {\ttfamily b7}. La case de départ {\ttfamily d5} devient inaccessible et infranchissable pour le reste de la partie}
\label{fig:yolah_moves2}
\end{figure}

% Un déplacement rapporte un point au joueur qui vient de se déplacer. Par exemple, dans la configuration de la figure \ref{fig:yolah_moves2}, le joueur noir a un point et le joueur blanc qui ne s'est pas encore déplacé a zéro point. Le but du jeu est assez simple à résumer~: il faut se déplacer plus longtemps que son adversaire~!

A move earns one point for the player who just moved. For example, in the configuration of Figure~\ref{fig:yolah_moves2}, the black player has one point and the white player who has not yet moved has zero points. The goal of the game is quite simple to summarize: you must move longer than your opponent!\\

% Maintenant c'est au tour des blancs de jouer. Ils doivent décider du pion blanc qu'ils vont bouger. Supposons que ce soit le pion en {\ttfamily e5}.
% Les déplacements possibles de ce pion blanc sont indiqués dans la figure \ref{fig:yolah_moves3}. Si blanc décide d'effectuer le coup {\ttfamily e5:f5}, on se retrouve dans la configuration de la figure \ref{fig:yolah_moves4} et le score est d'un point partout (chaque joueur a joué un coup).

Now it is white's turn to play. They must decide which white piece they will move. Suppose it is the piece at {\ttfamily e5}.
The possible moves for this white piece are shown in Figure~\ref{fig:yolah_moves3}. If white decides to make the move {\ttfamily e5:f5}, we end up in the configuration of Figure~\ref{fig:yolah_moves4} and the score is one point each (each player has played one move).

\begin{figure}[htpb]
\centering
\includegraphics[width=0.67\textwidth]{yolah_moves3.pdf}
% \caption{Mouvements possibles (petites croix blanches) pour le pion blanc situé sur la case {\ttfamily e5})}
\caption{Possible moves (small white crosses) for the white piece located on square {\ttfamily e5}}
\label{fig:yolah_moves3}
\end{figure}

\begin{figure}[htpb]
\centering
\includegraphics[width=0.6\textwidth]{yolah_moves4.png}
\caption{White just moved from {\ttfamily e5} to {\ttfamily f5}. The starting square {\ttfamily e5} becomes inaccessible and impassable for the rest of the game. The score is one point each (each player has moved once)}
% \caption{Blanc vient de se déplacer de {\ttfamily e5} en {\ttfamily f5}. La case de départ {\ttfamily e5} devient inaccessible et infranchissable pour le reste de la partie. Le score est d'un point partout (chaque joueur s'est déplacé une fois)}
\label{fig:yolah_moves4}
\end{figure}

% Pour résumer, les règles du jeu Yolah sont les suivantes~:

To summarize, the rules of Yolah are as follows:
\begin{itemize}
% \item Le jeu est un jeu à deux joueurs (les noirs et les blancs) au tour par tour.
\item The game is a two-player game (black and white) played in turns.
% \item Chaque joueur dispose de quatre pions.
\item Each player has four pieces.
% \item À son tour, le joueur choisit un des pions qui peut encore bouger, si aucun pion ne peut bouger il passe son tour (on indiquera par le coup {\ttfamily a1:a1} le fait de passer son tour).
\item On their turn, the player chooses one of the pieces that can still move; if no piece can move, they pass their turn (we will denote by the move {\ttfamily a1:a1} to skip one's turn).
% \item Il doit déplacer le pion choisi dans une des huit directions, d'autant de cases qu'il le souhaite, mais il ne doit pas atterrir ou être bloqué par un pion ou un trou.
\item They must move the chosen piece in one of the eight directions, as many squares as desired, but must not land on or be blocked by a piece or a hole.
% \item Après le déplacement du pion choisi, la case de départ du coup devient un trou et on ne peut plus alors la franchir ou se poser dessus.
\item After moving the chosen piece, the starting square of the move becomes a hole and can no longer be crossed or landed on.
% \item Après chaque déplacement le joueur gagne un point.
\item After each move, the player earns one point.
% \item La partie se termine lorsque les deux joueurs ne peuvent plus se déplacer.
\item The game ends when both players can no longer move.
% \item Le joueur qui a le plus de points gagne la partie.
\item The player with the most points wins the game.
% \item Si les deux joueurs ont le même nombre de points, la partie est déclarée nulle.
\item If both players have the same number of points, the game is declared a draw.
\end{itemize}

% \subsection{Caractéristiques intéressantes de Yolah pour développer des IA}
\subsection{Interesting Characteristics of Yolah for Developing AIs}

% J'ai choisi les règles précédentes pour Yolah, d'une part parce que j'aimais bien la dynamique du jeu des Pingouins, mais aussi parce qu'il n'y a pas de cycle dans le jeu, il n'y a donc pas besoin de règles spéciales pour éviter qu'une partie ne se termine pas. Le nombre de coups à disposition de chacun des joueurs est assez important au début (mais raisonnable) \footnote{56 coups possibles pour le joueur noir en début de partie.}, mais il va diminuer petit à petit avec l'apparition des trous \footnote{Le nombre de coups possibles ne diminue pas forcément après chaque coup, il existe des configurations où le joueur a plus de 56 coups à sa disposition.}. Cela permet aux IA de pouvoir regarder un nombre assez important de coups à l'avance.

I chose the above rules for Yolah, on the one hand because I liked the dynamics of the Penguin game, but also because there are no cycles in the game, so there is no need for special rules to prevent a game from never ending. The number of moves available to each player is quite large at the beginning (but reasonable)\footnote{56 possible moves for the black player at the start of the game.}, but it will decrease little by little with the appearance of holes\footnote{The number of possible moves does not necessarily decrease after each move; there are configurations where the player has more than 56 available moves.}. This allows the AIs to look ahead a fairly large number of moves.\\

% Je voulais aussi pouvoir réutiliser des concepts utilisés pour l'implémentation efficace du jeu d'échecs, et la taille du plateau et le déplacement des pions (même façon de se déplacer qu'une Dame aux échecs) me permettent de faire cela.

I also wanted to be able to reuse concepts used for the efficient implementation of chess, and the size of the board and the movement of the pieces (same way of moving as a Queen in chess) allow me to do that.

% \subsection{Exemple de partie}
\subsection{Game Example}

% Pour avoir une idée du déroulement d'une partie du jeu Yolah, nous allons faire jouer deux intelligences artificielles entre elles. La première IA sera basée sur la recherche arborescente Monte-Carlo et la deuxième sera basée sur le MinMax avec réseau neuronal. Nous étudierons ces deux IA dans la suite du livre. La seconde IA est plus forte et vous allez voir sa stratégie d'isolement de zones en action~!

To get an idea of how a Yolah game unfolds, we will have two artificial intelligences play against each other. The first AI will be based on Monte Carlo Tree Search and the second will be based on Minimax with a neural network. We will study both of these AIs later in the book. The second AI is stronger and you will see its zone isolation strategy in action!\\

% Le déroulement de la partie est décrit dans les figures \ref{fig:game_example_opening}, \ref{fig:game_example_middle} et \ref{fig:game_example_end}.

The progression of the game is described in Figures~\ref{fig:game_example_opening}, \ref{fig:game_example_middle} and \ref{fig:game_example_end}.\\

% L'IA des blancs estime qu'elle est gagnante à partir du coup 10 (voir figure \ref{fig:game_example_move10}). On peut voir au coup 30 (voir figure \ref{fig:game_example_move30}) qu'elle a réussi à isoler une zone où les noirs ne pourront plus accéder. Au coup 32 (voir figure \ref{fig:game_example_move32}) elle sort un de ses pions de la zone qu'elle a isolée car l'autre pion pourra collecter tous les points de celle-ci. Il est plus utile d'utiliser l'autre pion pour glaner des points autre part. Notons qu'à partir du coup 47 (voir à partir de la figure \ref{fig:game_example_move47}) les noirs n'ont plus aucun coup disponible et doivent donc passer leur tour.

The white AI estimates that it is winning starting from move 10 (see Figure~\ref{fig:game_example_move10}). We can see at move 30 (see Figure~\ref{fig:game_example_move30}) that it has successfully isolated a zone where black can no longer access. At move 32 (see Figure~\ref{fig:game_example_move32}) it moves one of its pieces out of the isolated zone because the other piece will be able to collect all the points from that zone. It is more useful to use the other piece to gather points elsewhere. Note that starting from move 47 (see Figure~\ref{fig:game_example_move47} onward), black has no more available moves and must therefore pass their turn.\\

% La partie est gagnée par le joueur blanc 32 à 24, ce qui est un très bon score car le jeu Yolah me semble être à l'avantage des noirs.

The game is won by the white player 32 to 24, which is a very good score because the Yolah game seems to favor black.

% a1:a2 h1:f3 h8:f6 a8:c6 d5:d6 d4:b2 d6:e6 e5:g5 f6:f5 f3:g4 f5:h7 g5:h5 e4:f4 h5:f7 e6:e8 g4:g1 e8:b8
% b2:d2 a2:a3 g1:f2 a3:a6 d2:d3 f4:h2 d3:b5 a6:a4 b5:b3 a4:c4 b3:c3 h2:g2 f2:f1 b8:c7 c3:a5 c4:c5
% c6:d7 g2:g3 f7:g7 c5:b4 a5:b6 g3:h4 b6:b7 h7:g8 d7:d8 c7:c8 f1:b1 g8:f8 d8:e7 h4:h3 b1:c1 c1:e1
% e1:e3 e3:e2 e2:d1 d1:c2 g7:g6 g6:h6 b7:a7

\begin{figure}[htpb] %H
\centering
\begin{subfigure}[b]{0.24\textwidth}
    \centering
    \includegraphics[width=\textwidth]{board.png}
    % \caption{Position initiale}
    \caption{Initial position}
    \label{fig:game_example_init}
\end{subfigure}
\hfill
\begin{subfigure}[b]{0.24\textwidth}
    \centering
    \includegraphics[width=\textwidth]{yolah_game_example_move1.png}
    % \caption{Coup 1 (Noir)}
    \caption{Move 1 (Black)}
    \label{fig:game_example_move1}
\end{subfigure}
\hfill
\begin{subfigure}[b]{0.24\textwidth}
    \centering
    \includegraphics[width=\textwidth]{yolah_game_example_move2.png}
    % \caption{Coup 2 (Blanc)}
    \caption{Move 2 (White)}
    \label{fig:game_example_move2}
\end{subfigure}
\hfill
\begin{subfigure}[b]{0.24\textwidth}
    \centering
    \includegraphics[width=\textwidth]{yolah_game_example_move3.png}
    % \caption{Coup 3 (Noir)}
    \caption{Move 3 (Black)}
    \label{fig:game_example_move3}
\end{subfigure}\\

\begin{subfigure}[b]{0.24\textwidth}
    \centering
    \includegraphics[width=\textwidth]{yolah_game_example_move4.png}
    % \caption{Coup 4 (Blanc)}
    \caption{Move 4 (White)}
    \label{fig:game_example_move4}
\end{subfigure}
\hfill
\begin{subfigure}[b]{0.24\textwidth}
    \centering
    \includegraphics[width=\textwidth]{yolah_game_example_move5.png}
    % \caption{Coup 5 (Noir)}
    \caption{Move 5 (Black)}
    \label{fig:game_example_move5}
\end{subfigure}
\hfill
\begin{subfigure}[b]{0.24\textwidth}
    \centering
    \includegraphics[width=\textwidth]{yolah_game_example_move6.png}
    % \caption{Coup 6 (Blanc)}
    \caption{Move 6 (White)}
    \label{fig:game_example_move6}
\end{subfigure}
\hfill
\begin{subfigure}[b]{0.24\textwidth}
    \centering
    \includegraphics[width=\textwidth]{yolah_game_example_move7.png}
    % \caption{Coup 7 (Noir)}
    \caption{Move 7 (Black)}
    \label{fig:game_example_move7}
\end{subfigure}\\

\begin{subfigure}[b]{0.24\textwidth}
    \centering
    \includegraphics[width=\textwidth]{yolah_game_example_move8.png}
    % \caption{Coup 8 (Blanc)}
    \caption{Move 8 (White)}
    \label{fig:game_example_move8}
\end{subfigure}
\hfill
\begin{subfigure}[b]{0.24\textwidth}
    \centering
    \includegraphics[width=\textwidth]{yolah_game_example_move9.png}
    % \caption{Coup 9 (Noir)}
    \caption{Move 9 (Black)}
    \label{fig:game_example_move9}
\end{subfigure}
\hfill
\begin{subfigure}[b]{0.24\textwidth}
    \centering
    \includegraphics[width=\textwidth]{yolah_game_example_move10.png}
    % \caption{Coup 10 (Blanc)}
    \caption{Move 10 (White)}
    \label{fig:game_example_move10}
\end{subfigure}
\hfill
\begin{subfigure}[b]{0.24\textwidth}
    \centering
    \includegraphics[width=\textwidth]{yolah_game_example_move11.png}
    % \caption{Coup 11 (Noir)}
    \caption{Move 11 (Black)}
    \label{fig:game_example_move11}
\end{subfigure}\\

\begin{subfigure}[b]{0.24\textwidth}
    \centering
    \includegraphics[width=\textwidth]{yolah_game_example_move12.png}
    % \caption{Coup 12 (Blanc)}
    \caption{Move 12 (White)}
    \label{fig:game_example_move12}
\end{subfigure}
\hfill
\begin{subfigure}[b]{0.24\textwidth}
    \centering
    \includegraphics[width=\textwidth]{yolah_game_example_move13.png}
    % \caption{Coup 13 (Noir)}
    \caption{Move 13 (Black)}
    \label{fig:game_example_move13}
\end{subfigure}
\hfill
\begin{subfigure}[b]{0.24\textwidth}
    \centering
    \includegraphics[width=\textwidth]{yolah_game_example_move14.png}
    % \caption{Coup 14 (Blanc)}
    \caption{Move 14 (White)}
    \label{fig:game_example_move14}
\end{subfigure}
\hfill
\begin{subfigure}[b]{0.24\textwidth}
    \centering
    \includegraphics[width=\textwidth]{yolah_game_example_move15.png}
    % \caption{Coup 15 (Noir)}
    \caption{Move 15 (Black)}
    \label{fig:game_example_move15}
\end{subfigure}\\

\begin{subfigure}[b]{0.24\textwidth}
    \centering
    \includegraphics[width=\textwidth]{yolah_game_example_move16.png}
    % \caption{Coup 16 (Blanc)}
    \caption{Move 16 (White)}
    \label{fig:game_example_move16}
\end{subfigure}
\hfill
\begin{subfigure}[b]{0.24\textwidth}
    \centering
    \includegraphics[width=\textwidth]{yolah_game_example_move17.png}
    % \caption{Coup 17 (Noir)}
    \caption{Move 17 (Black)}
    \label{fig:game_example_move17}
\end{subfigure}
\hfill
\begin{subfigure}[b]{0.24\textwidth}
    \centering
    \includegraphics[width=\textwidth]{yolah_game_example_move18.png}
    % \caption{Coup 18 (Blanc)}
    \caption{Move 18 (White)}
    \label{fig:game_example_move18}
\end{subfigure}
\hfill
\begin{subfigure}[b]{0.24\textwidth}
    \centering
    \includegraphics[width=\textwidth]{yolah_game_example_move19.png}
    % \caption{Coup 19 (Noir)}
    \caption{Move 19 (Black)}
    \label{fig:game_example_move19}
\end{subfigure}

% \caption{Example de partie entre deux IA - coups 1 à 19}
\caption{Game example between two AIs - moves 1 to 19}
\label{fig:game_example_opening}
\end{figure}

\begin{figure}[htpb]
\centering

\begin{subfigure}[b]{0.24\textwidth}
    \centering
    \includegraphics[width=\textwidth]{yolah_game_example_move20.png}
    % \caption{Coup 20 (Blanc)}
    \caption{Move 20 (White)}
    \label{fig:game_example_move20}
\end{subfigure}
\hfill
\begin{subfigure}[b]{0.24\textwidth}
    \centering
    \includegraphics[width=\textwidth]{yolah_game_example_move21.png}
    % \caption{Coup 21 (Noir)}
    \caption{Move 21 (Black)}
    \label{fig:game_example_move21}
\end{subfigure}
\hfill
\begin{subfigure}[b]{0.24\textwidth}
    \centering
    \includegraphics[width=\textwidth]{yolah_game_example_move22.png}
    % \caption{Coup 22 (Blanc)}
    \caption{Move 22 (White)}
    \label{fig:game_example_move22}
\end{subfigure}
\hfill
\begin{subfigure}[b]{0.24\textwidth}
    \centering
    \includegraphics[width=\textwidth]{yolah_game_example_move23.png}
    % \caption{Coup 23 (Noir)}
    \caption{Move 23 (Black)}
    \label{fig:game_example_move23}
\end{subfigure}\\

\begin{subfigure}[b]{0.24\textwidth}
    \centering
    \includegraphics[width=\textwidth]{yolah_game_example_move24.png}
    % \caption{Coup 24 (Blanc)}
    \caption{Move 24 (White)}
    \label{fig:game_example_move24}
\end{subfigure}
\hfill
\begin{subfigure}[b]{0.24\textwidth}
    \centering
    \includegraphics[width=\textwidth]{yolah_game_example_move25.png}
    % \caption{Coup 25 (Noir)}
    \caption{Move 25 (Black)}
    \label{fig:game_example_move25}
\end{subfigure}
\hfill
\begin{subfigure}[b]{0.24\textwidth}
    \centering
    \includegraphics[width=\textwidth]{yolah_game_example_move26.png}
    % \caption{Coup 26 (Blanc)}
    \caption{Move 26 (White)}
    \label{fig:game_example_move26}
\end{subfigure}
\hfill
\begin{subfigure}[b]{0.24\textwidth}
    \centering
    \includegraphics[width=\textwidth]{yolah_game_example_move27.png}
    % \caption{Coup 27 (Noir)}
    \caption{Move 27 (Black)}
    \label{fig:game_example_move27}
\end{subfigure}\\

\begin{subfigure}[b]{0.24\textwidth}
    \centering
    \includegraphics[width=\textwidth]{yolah_game_example_move28.png}
    % \caption{Coup 28 (Blanc)}
    \caption{Move 28 (White)}
    \label{fig:game_example_move28}
\end{subfigure}
\hfill
\begin{subfigure}[b]{0.24\textwidth}
    \centering
    \includegraphics[width=\textwidth]{yolah_game_example_move29.png}
    % \caption{Coup 29 (Noir)}
    \caption{Move 29 (Black)}
    \label{fig:game_example_move29}
\end{subfigure}
\hfill
\begin{subfigure}[b]{0.24\textwidth}
    \centering
    \includegraphics[width=\textwidth]{yolah_game_example_move30.png}
    % \caption{Coup 30 (Blanc)}
    \caption{Move 30 (White)}
    \label{fig:game_example_move30}
\end{subfigure}
\hfill
\begin{subfigure}[b]{0.24\textwidth}
    \centering
    \includegraphics[width=\textwidth]{yolah_game_example_move31.png}
    % \caption{Coup 31 (Noir)}
    \caption{Move 31 (Black)}
    \label{fig:game_example_move31}
\end{subfigure}\\

\begin{subfigure}[b]{0.24\textwidth}
    \centering
    \includegraphics[width=\textwidth]{yolah_game_example_move32.png}
    % \caption{Coup 32 (Blanc)}
    \caption{Move 32 (White)}
    \label{fig:game_example_move32}
\end{subfigure}
\hfill
\begin{subfigure}[b]{0.24\textwidth}
    \centering
    \includegraphics[width=\textwidth]{yolah_game_example_move33.png}
    % \caption{Coup 33 (Noir)}
    \caption{Move 33 (Black)}
    \label{fig:game_example_move33}
\end{subfigure}
\hfill
\begin{subfigure}[b]{0.24\textwidth}
    \centering
    \includegraphics[width=\textwidth]{yolah_game_example_move34.png}
    % \caption{Coup 34 (Blanc)}
    \caption{Move 34 (White)}
    \label{fig:game_example_move34}
\end{subfigure}
\hfill
\begin{subfigure}[b]{0.24\textwidth}
    \centering
    \includegraphics[width=\textwidth]{yolah_game_example_move35.png}
    % \caption{Coup 35 (Noir)}
    \caption{Move 35 (Black)}
    \label{fig:game_example_move35}
\end{subfigure}\\

\begin{subfigure}[b]{0.24\textwidth}
    \centering
    \includegraphics[width=\textwidth]{yolah_game_example_move36.png}
    % \caption{Coup 36 (Blanc)}
    \caption{Move 36 (White)}
    \label{fig:game_example_move36}
\end{subfigure}
\hfill
\begin{subfigure}[b]{0.24\textwidth}
    \centering
    \includegraphics[width=\textwidth]{yolah_game_example_move37.png}
    % \caption{Coup 37 (Noir)}
    \caption{Move 37 (Black)}
    \label{fig:game_example_move37}
\end{subfigure}
\hfill
\begin{subfigure}[b]{0.24\textwidth}
    \centering
    \includegraphics[width=\textwidth]{yolah_game_example_move38.png}
    % \caption{Coup 38 (Blanc)}
    \caption{Move 38 (White)}
    \label{fig:game_example_move38}
\end{subfigure}
\hfill
\begin{subfigure}[b]{0.24\textwidth}
    \centering
    \includegraphics[width=\textwidth]{yolah_game_example_move39.png}
    % \caption{Coup 39 (Noir)}
    \caption{Move 39 (Black)}
    \label{fig:game_example_move39}
\end{subfigure}\\

% \caption{Example de partie entre deux IA - coups 20 à 39}
\caption{Game example between two AIs - moves 20 to 39}
\label{fig:game_example_middle}
\end{figure}


\begin{figure}[htpb]
\centering

\begin{subfigure}[b]{0.24\textwidth}
    \centering
    \includegraphics[width=\textwidth]{yolah_game_example_move40.png}
    % \caption{Coup 40 (Blanc)}
    \caption{Move 40 (White)}
    \label{fig:game_example_move40}
\end{subfigure}
\hfill
\begin{subfigure}[b]{0.24\textwidth}
    \centering
    \includegraphics[width=\textwidth]{yolah_game_example_move41.png}
    % \caption{Coup 41 (Noir)}
    \caption{Move 41 (Black)}
    \label{fig:game_example_move41}
\end{subfigure}
\hfill
\begin{subfigure}[b]{0.24\textwidth}
    \centering
    \includegraphics[width=\textwidth]{yolah_game_example_move42.png}
    % \caption{Coup 42 (Blanc)}
    \caption{Move 42 (White)}
    \label{fig:game_example_move42}
\end{subfigure}
\hfill
\begin{subfigure}[b]{0.24\textwidth}
    \centering
    \includegraphics[width=\textwidth]{yolah_game_example_move43.png}
    % \caption{Coup 43 (Noir)}
    \caption{Move 43 (Black)}
    \label{fig:game_example_move43}
\end{subfigure}\\

\begin{subfigure}[b]{0.24\textwidth}
    \centering
    \includegraphics[width=\textwidth]{yolah_game_example_move44.png}
    % \caption{Coup 44 (Blanc)}
    \caption{Move 44 (White)}
    \label{fig:game_example_move44}
\end{subfigure}
\hfill
\begin{subfigure}[b]{0.24\textwidth}
    \centering
    \includegraphics[width=\textwidth]{yolah_game_example_move45.png}
    % \caption{Coup 45 (Noir)}
    \caption{Move 45 (Black)}
    \label{fig:game_example_move45}
\end{subfigure}
\hfill
\begin{subfigure}[b]{0.24\textwidth}
    \centering
    \includegraphics[width=\textwidth]{yolah_game_example_move46.png}
    % \caption{Coup 46 (Blanc)}
    \caption{Move 46 (White)}
    \label{fig:game_example_move46}
\end{subfigure}
\hfill
\begin{subfigure}[b]{0.24\textwidth}
    \centering
    \includegraphics[width=\textwidth]{yolah_game_example_move47.png}
    % \caption{Coup 47 (Noir)}
    \caption{Move 47 (Black)}
    \label{fig:game_example_move47}
\end{subfigure}\\

\begin{subfigure}[b]{0.24\textwidth}
    \centering
    \includegraphics[width=\textwidth]{yolah_game_example_move48.png}
    % \caption{Coup 48 (Blanc)}
    \caption{Move 48 (White)}
    \label{fig:game_example_move48}
\end{subfigure}
\hfill
\begin{subfigure}[b]{0.24\textwidth}
    \centering
    \includegraphics[width=\textwidth]{yolah_game_example_move49.png}
    % \caption{Coup 49 (Blanc)}
    \caption{Move 49 (White)}
    \label{fig:game_example_move49}
\end{subfigure}
\hfill
\begin{subfigure}[b]{0.24\textwidth}
    \centering
    \includegraphics[width=\textwidth]{yolah_game_example_move50.png}
    % \caption{Coup 50 (Blanc)}
    \caption{Move 50 (White)}
    \label{fig:game_example_move50}
\end{subfigure}
\hfill
\begin{subfigure}[b]{0.24\textwidth}
    \centering
    \includegraphics[width=\textwidth]{yolah_game_example_move51.png}
    % \caption{Coup 51 (Blanc)}
    \caption{Move 51 (White)}
    \label{fig:game_example_move51}
\end{subfigure}\\

\begin{subfigure}[b]{0.24\textwidth}
    \centering
    \includegraphics[width=\textwidth]{yolah_game_example_move52.png}
    % \caption{Coup 52 (Blanc)}
    \caption{Move 52 (White)}
    \label{fig:game_example_move52}
\end{subfigure}
\hfill
\begin{subfigure}[b]{0.24\textwidth}
    \centering
    \includegraphics[width=\textwidth]{yolah_game_example_move53.png}
    % \caption{Coup 53 (Blanc)}
    \caption{Move 53 (White)}
    \label{fig:game_example_move53}
\end{subfigure}
\hfill
\begin{subfigure}[b]{0.24\textwidth}
    \centering
    \includegraphics[width=\textwidth]{yolah_game_example_move54.png}
    % \caption{Coup 54 (Blanc)}
    \caption{Move 54 (White)}
    \label{fig:game_example_move54}
\end{subfigure}
\hfill
\begin{subfigure}[b]{0.24\textwidth}
    \centering
    \includegraphics[width=\textwidth]{yolah_game_example_move55.png}
    % \caption{Coup 55 (Blanc)}
    \caption{Move 55 (White)}
    \label{fig:game_example_move55}
\end{subfigure}\\

\begin{subfigure}[b]{0.24\textwidth}
    \centering
    \includegraphics[width=\textwidth]{yolah_game_example_move56.png}
    % \caption{Coup 56 (Blanc)}
    \caption{Move 56 (White)}
    \label{fig:game_example_move56}
\end{subfigure}
\hfill

% \caption{Example de partie entre deux IA - coups 39 à 56. Blanc gagne 32 à 24}
\caption{Game example between two AIs - moves 40 to 56. White wins 32 to 24}
\label{fig:game_example_end}
\end{figure}

% \subsection{La suite}
\section{What's Next}

% Dans le prochain chapitre, nous allons étudier l'implémentation du jeu Yolah en C++ \cite{cppreference}, cette implémentation se veut efficace car il sera important pour les IA de pouvoir jouer beaucoup de parties à la seconde~; leur niveau de jeu va en dépendre.

In the next chapter, we will study the implementation of the Yolah game in \cite{cppreference}. This implementation is designed to be efficient because it will be important for the AIs to be able to play many games per second; their level of play will depend on it.\\

% Le chapitre \ref{ch:ai} décrit l'interface commune de nos différentes IA, le chapitre \ref{ch:monte_carlo} présente une IA très simple basée sur une recherche Monte-Carlo. L'IA suivante, décrite au chapitre \ref{fig:mcts}, est une évolution de la précédente et permettra, contrairement à l'IA Monte-Carlo, de développer un arbre de jeu. Notons que ces deux IA ne nécessiteront pas d'heuristiques apportées par l'humain et n'ont besoin que des règles du jeu. Le chapitre \ref{ch:minmax} présente une IA basée sur une recherche arborescente minmax avec des heuristiques fournies par l'humain. L'heuristique utilisée par l'IA est une combinaison linéaire des heuristiques fournies par l'humain~; les pondérations de chacune des heuristiques dans cette combinaison linéaire sont apprises grâce à un algorithme génétique. Notre dernière, et plus forte IA, est présentée au chapitre \ref{ch:nnue}. Un réseau de neurones est utilisé à la place des heuristiques. Ce réseau de neurones est entraîné sur un ensemble de parties jouées par l'IA précédente.

Chapter~\ref{ch:ai} describes the common interface for our different AIs. Chapter~\ref{ch:monte_carlo} presents a very simple AI based on Monte Carlo search. The next AI, described in Chapter~\ref{fig:mcts}, is an evolution of the previous one and will allow, unlike the Monte Carlo AI, the development of a game tree. Note that these two AIs will not require heuristics provided by humans and only need the rules of the game. Chapter~\ref{ch:minmax} presents an AI based on minimax tree search with heuristics provided by humans. The heuristic used by the AI is a linear combination of the heuristics provided by humans; the weights of each heuristic in this linear combination are learned using a genetic algorithm. Our last and strongest AI is presented in Chapter~\ref{ch:nnue}. A neural network is used instead of heuristics. This neural network is trained on a set of games played by the previous AI.\\

% Le chapitre \ref{ch:tournament} évalue alors toutes ces IA, en les faisant se rencontrer dans un tournoi. Nous conclurons ensuite et proposerons différentes directions pour créer d'autres joueurs artificiels.

Chapter~\ref{ch:tournament} then evaluates all these AIs by having them compete in a tournament. We will then conclude and propose different directions for creating other artificial players.\\

% Bonne lecture~!

Happy reading!


% \subsection{Why Yolah for AI Research?}

% Yolah provides several advantages as a platform for AI development:

% \begin{itemize}
%     \item \textbf{Moderate complexity}: Complex enough to be interesting, simple enough to be tractable
%     \item \textbf{Clear evaluation}: Position quality can be measured objectively
%     \item \textbf{Strategic depth}: Multiple viable strategies and playing styles
%     \item \textbf{Computational feasibility}: Playable with various AI techniques
% \end{itemize}

% \section{Overview of Game AI}

% Game-playing has been a cornerstone of artificial intelligence research since the field's inception. From early checkers programs to modern systems that master Go and Chess, game AI has driven innovation in search, learning, and decision-making.

% \subsection{Historical Context}

% The evolution of game AI can be traced through several key milestones:

% \begin{table}[H]
% \centering
% \begin{tabular}{lp{8cm}}
% \toprule
% \textbf{Year} & \textbf{Achievement} \\
% \midrule
% 1950s & Early checkers programs \\
% 1970s & Chess programs reach amateur level \\
% 1997 & Deep Blue defeats world champion Garry Kasparov \\
% 2016 & AlphaGo defeats world champion Lee Sedol \\
% 2017 & AlphaZero masters Chess, Go, and Shogi through self-play \\
% \bottomrule
% \end{tabular}
% \caption{Key milestones in game-playing AI}
% \label{tab:milestones}
% \end{table}

% \subsection{Categories of Game AI Techniques}

% Game AI techniques can be broadly categorized as:

% \begin{description}
%     \item[Search-based methods] Use game tree exploration (Minimax, Alpha-Beta)
%     \item[Sampling methods] Use Monte Carlo simulations (MCTS)
%     \item[Learning-based methods] Use neural networks and reinforcement learning
%     \item[Hybrid approaches] Combine multiple techniques for optimal performance
% \end{description}

% \section{Book Structure}

% This book is organized into three main parts:

% \textbf{Part I: Foundations} covers the basic game engine, data structures, and move generation.

% \textbf{Part II: Classical Algorithms} explores traditional search-based AI techniques.

% \textbf{Part III: Modern Approaches} delves into neural networks and learning methods.

% Each chapter builds upon previous concepts, with complete code examples and performance analysis.

\chapter{Game Engine}
\label{ch:engine}

% Nous allons implémenter la gestion du jeu en C++ en essayant de créer une implémentation efficace pour pouvoir réaliser un maximum de parties à la seconde, ce qui sera important pour obtenir des IA de bons niveaux. Nous testerons notre implémentation en générant un maximum de parties aléatoires en un temps donné à la fin de ce chapitre.
% Nous allons nous inspirer de l'excellent moteur de jeu d'échecs Stockfish pour les structures de données de Yolah.
We will implement game management in C++, striving to create an efficient implementation capable of running as many games per second as possible—this will be crucial for developing high-level AIs. We will test our implementation by generating as many random games as possible within a given time at the end of this chapter.
We will draw inspiration from the excellent Stockfish chess engine \cite{stockfish2025} for Yolah's data structures.

\section{Data Structures}

% Nous allons représenter la position des noirs, des blancs et des cases détruites par des des entiers non signés sur 64 bits : uint64_t. Nous appelerons bitboard ces entiers. Nous prenons des uint64_t car le plateau de Yolah contient 64 cases, nous avons donc un bit par case. La table donne la position de chaque cases du plateau dans le bitboard. Cette information est représentée dans le code par l'énumération du listing.
We will represent the positions of black pieces, white pieces, and destroyed squares using $64$-bit unsigned integers: \mintinline{c++}{uint64_t}. We call these integers bitboards. We use \mintinline{c++}{uint64_t} because the Yolah board contains $64$ squares, giving us one bit per square. Table \ref{tab:bitboard_indices} shows the position of each board square in the bitboard. This information is represented in code by the enumeration in Listing \ref{lst:squares}\footnote{By default, the enumeration starts at value 0, so \mintinline{c++}{SQ_A1} equals 0, \mintinline{c++}{SQ_B1} equals 1, \ldots}.\\

\begin{table}[htpb]
\centering
% \caption{Positions de chaque cases du plateau dans le bitboard}
\caption{Position of each board square in the bitboard}
\label{tab:bitboard_indices}
\begin{tabular}{|c|c|c|c|c|c|c|c|c|}
\hline
8 & bit$_{56}$ & bit$_{57}$ & bit$_{58}$ & bit$_{59}$ & bit$_{60}$ & bit$_{61}$ & bit$_{62}$ & bit$_{63}$ \\ \hline
7 & bit$_{48}$ & bit$_{49}$ & bit$_{50}$ & bit$_{51}$ & bit$_{52}$ & bit$_{53}$ & bit$_{54}$ & bit$_{55}$ \\ \hline
6 & bit$_{40}$ & bit$_{41}$ & bit$_{42}$ & bit$_{43}$ & bit$_{44}$ & bit$_{45}$ & bit$_{46}$ & bit$_{47}$ \\ \hline
5 & bit$_{32}$ & bit$_{33}$ & bit$_{34}$ & bit$_{35}$ & bit$_{36}$ & bit$_{37}$ & bit$_{38}$ & bit$_{39}$ \\ \hline
4 & bit$_{24}$ & bit$_{25}$ & bit$_{26}$ & bit$_{27}$ & bit$_{28}$ & bit$_{29}$ & bit$_{30}$ & bit$_{31}$ \\ \hline
3 & bit$_{16}$ & bit$_{17}$ & bit$_{18}$ & bit$_{19}$ & bit$_{20}$ & bit$_{21}$ & bit$_{22}$ & bit$_{23}$ \\ \hline
2 & bit$_{8}$ & bit$_{9}$ & bit$_{10}$ & bit$_{11}$ & bit$_{12}$ & bit$_{13}$ & bit$_{14}$ & bit$_{15}$ \\ \hline
1 & bit$_{0}$ & bit$_{1}$ & bit$_{2}$ & bit$_{3}$ & bit$_{4}$ & bit$_{5}$ & bit$_{6}$ & bit$_{7}$ \\ \hline
\multicolumn{1}{c|}{} & a & b & c & d & e & f & g & h \\
\cline{2-9}
\end{tabular}
\end{table}

\begin{mdframed}[skipabove=\baselineskip,hidealllines=true]
\begin{minted}[linenos, linenos,breaklines,fontsize=\small]{cpp}
enum Square : int8_t {
    SQ_A1, SQ_B1, SQ_C1, SQ_D1, SQ_E1, SQ_F1, SQ_G1, SQ_H1,
    SQ_A2, SQ_B2, SQ_C2, SQ_D2, SQ_E2, SQ_F2, SQ_G2, SQ_H2,
    SQ_A3, SQ_B3, SQ_C3, SQ_D3, SQ_E3, SQ_F3, SQ_G3, SQ_H3,
    SQ_A4, SQ_B4, SQ_C4, SQ_D4, SQ_E4, SQ_F4, SQ_G4, SQ_H4,
    SQ_A5, SQ_B5, SQ_C5, SQ_D5, SQ_E5, SQ_F5, SQ_G5, SQ_H5,
    SQ_A6, SQ_B6, SQ_C6, SQ_D6, SQ_E6, SQ_F6, SQ_G6, SQ_H6,
    SQ_A7, SQ_B7, SQ_C7, SQ_D7, SQ_E7, SQ_F7, SQ_G7, SQ_H7,
    SQ_A8, SQ_B8, SQ_C8, SQ_D8, SQ_E8, SQ_F8, SQ_G8, SQ_H8,
    SQ_NONE,
    SQUARE_ZERO = 0,
    SQUARE_NB   = 64
};
\end{minted}
\vspace{-0.5em}
% \captionof{listing}{Les cases du plateau de jeu}
\captionof{listing}{Board squares}
\label{lst:squares}
\end{mdframed}


\begin{figure}[htpb]
\centering
\includegraphics[width=0.5\textwidth]{yolah_game_example_move21.png}
% \caption{Configuration du plateau de jeu correspondant au coup 21 de la partie donnée en example dans le chapitre précédent (figure \ref{fig:game_example_move21})}
\caption{Board configuration corresponding to move 21 of the game given as an example in the previous chapter (Figure \ref{fig:game_example_move21})}
\label{fig:yolah_game_engine1}
\end{figure}

% Prenons comme exemple le plateau représenté dans la figure.
Let us take as an example the board shown in Figure \ref{fig:yolah_game_engine1}.\\
\begin{itemize}
% \item Les positions des pièces noires sur le plateau sont représentées par le bitboard :
\item The positions of black pieces on the board (Table \ref{tab:black_positions_game_example_move21}) are represented by the bitboard:
\mintinline{c++}{0b0000001010000000000000010000000000100000000000000000000000000000}.\\

\begin{table}[htpb]
\centering
% \caption{Positions des pièces noires}
\caption{Black piece positions}
\label{tab:black_positions_game_example_move21}
\begin{tabular}{|c|c|c|c|c|c|c|c|c|}
\hline
8 & $\cdot$ & $\bullet$ & $\cdot$ & $\cdot$ & $\cdot$ & $\cdot$ & $\cdot$ & $\cdot$ \\ \hline
7 & $\cdot$ & $\cdot$ & $\cdot$ & $\cdot$ & $\cdot$ & $\cdot$ & $\cdot$ & $\bullet$ \\ \hline
6 & $\bullet$ & $\cdot$ & $\cdot$ & $\cdot$ & $\cdot$ & $\cdot$ & $\cdot$ & $\cdot$ \\ \hline
5 & $\cdot$ & $\cdot$ & $\cdot$ & $\cdot$ & $\cdot$ & $\cdot$ & $\cdot$ & $\cdot$ \\ \hline
4 & $\cdot$ & $\cdot$ & $\cdot$ & $\cdot$ & $\cdot$ & $\bullet$ & $\cdot$ & $\cdot$ \\ \hline
3 & $\cdot$ & $\cdot$ & $\cdot$ & $\cdot$ & $\cdot$ & $\cdot$ & $\cdot$ & $\cdot$ \\ \hline
2 & $\cdot$ & $\cdot$ & $\cdot$ & $\cdot$ & $\cdot$ & $\cdot$ & $\cdot$ & $\cdot$ \\ \hline
1 & $\cdot$ & $\cdot$ & $\cdot$ & $\cdot$ & $\cdot$ & $\cdot$ & $\cdot$ & $\cdot$ \\ \hline
\multicolumn{1}{c|}{} & a & b & c & d & e & f & g & h \\
\cline{2-9}
\end{tabular}
\end{table}


% \item Les positions des pièces blanches sur le plateau sont représentées par le bitboard :
\item The positions of white pieces on the board (Table \ref{tab:white_positions_game_example_move21}) are represented by the bitboard:
\mintinline{c++}{0b0000000000100000000001000000000000000000000000000010100000000000}.\\

\begin{table}[htpb]
\centering
% \caption{Positions des pièces blanches}
\caption{White piece positions}
\label{tab:white_positions_game_example_move21}
\begin{tabular}{|c|c|c|c|c|c|c|c|c|}
\hline
8 & $\cdot$ & $\cdot$ & $\cdot$ & $\cdot$ & $\cdot$ & $\cdot$ & $\cdot$ & $\cdot$ \\ \hline
7 & $\cdot$ & $\cdot$ & $\cdot$ & $\cdot$ & $\cdot$ & $\circ$ & $\cdot$ & $\cdot$ \\ \hline
6 & $\cdot$ & $\cdot$ & $\circ$ & $\cdot$ & $\cdot$ & $\cdot$ & $\cdot$ & $\cdot$ \\ \hline
5 & $\cdot$ & $\cdot$ & $\cdot$ & $\cdot$ & $\cdot$ & $\cdot$ & $\cdot$ & $\cdot$ \\ \hline
4 & $\cdot$ & $\cdot$ & $\cdot$ & $\cdot$ & $\cdot$ & $\cdot$ & $\cdot$ & $\cdot$ \\ \hline
3 & $\cdot$ & $\cdot$ & $\cdot$ & $\cdot$ & $\cdot$ & $\cdot$ & $\cdot$ & $\cdot$ \\ \hline
2 & $\cdot$ & $\cdot$ & $\cdot$ & $\circ$ & $\cdot$ & $\circ$ & $\cdot$ & $\cdot$ \\ \hline
1 & $\cdot$ & $\cdot$ & $\cdot$ & $\cdot$ & $\cdot$ & $\cdot$ & $\cdot$ & $\cdot$ \\ \hline
\multicolumn{1}{c|}{} & a & b & c & d & e & f & g & h \\
\cline{2-9}
\end{tabular}
\end{table}

% \item Les positions des trous sont représentées par le bitboard :
\item The positions of holes (Table \ref{tab:empty_positions_game_example_move21}) are represented by the bitboard:\\
\mintinline{c++}{0b1001000100000000001110001111100001011000001000010000001111000001}.\\

\begin{table}[htpb]
\centering
% \caption{Positions des cases détruites (les trous)}
\caption{Destroyed square positions (holes)}
\label{tab:empty_positions_game_example_move21}
\begin{tabular}{|c|c|c|c|c|c|c|c|c|}
\hline
8 & \raisebox{-0.3ex}{$\blacksquare$} & $\cdot$ & $\cdot$ & $\cdot$ & \raisebox{-0.3ex}{$\blacksquare$} & $\cdot$ & $\cdot$ & \raisebox{-0.3ex}{$\blacksquare$} \\ \hline
7 & $\cdot$ & $\cdot$ & $\cdot$ & $\cdot$ & $\cdot$ & $\cdot$ & $\cdot$ & $\cdot$ \\ \hline
6 & $\cdot$ & $\cdot$ & $\cdot$ & \raisebox{-0.3ex}{$\blacksquare$} & \raisebox{-0.3ex}{$\blacksquare$} & \raisebox{-0.3ex}{$\blacksquare$} & $\cdot$ & $\cdot$ \\ \hline
5 & $\cdot$ & $\cdot$ & $\cdot$ & \raisebox{-0.3ex}{$\blacksquare$} & \raisebox{-0.3ex}{$\blacksquare$} & \raisebox{-0.3ex}{$\blacksquare$} & \raisebox{-0.3ex}{$\blacksquare$} & \raisebox{-0.3ex}{$\blacksquare$} \\ \hline
4 & $\cdot$ & $\cdot$ & $\cdot$ & \raisebox{-0.3ex}{$\blacksquare$} & \raisebox{-0.3ex}{$\blacksquare$} & $\cdot$ & \raisebox{-0.3ex}{$\blacksquare$} & $\cdot$ \\ \hline
3 & \raisebox{-0.3ex}{$\blacksquare$} & $\cdot$ & $\cdot$ & $\cdot$ & $\cdot$ & \raisebox{-0.3ex}{$\blacksquare$} & $\cdot$ & $\cdot$ \\ \hline
2 & \raisebox{-0.3ex}{$\blacksquare$} & \raisebox{-0.3ex}{$\blacksquare$} & $\cdot$ & $\cdot$ & $\cdot$ & $\cdot$ & $\cdot$ & $\cdot$ \\ \hline
1 & \raisebox{-0.3ex}{$\blacksquare$} & $\cdot$ & $\cdot$ & $\cdot$ & $\cdot$ & $\cdot$ & \raisebox{-0.3ex}{$\blacksquare$} & \raisebox{-0.3ex}{$\blacksquare$} \\ \hline
\multicolumn{1}{c|}{} & a & b & c & d & e & f & g & h \\
\cline{2-9}
\end{tabular}
\end{table}

\end{itemize}


% Les bitboards vont nous permettre une manipulation efficace du plateau, en utilisant des opérations bits à bits, et nécessiterons peu de place mémoire pour représenter celui-ci.
Bitboards allow us to efficiently manipulate the board using bitwise operations while requiring minimal memory to represent it.\\

% Nous représentons le plateau de jeu par la classe Yolah dont un extrait est donné dans le listing.
We represent the game board with the \mintinline{c++}{Yolah} class, an excerpt of which is given in Listing \ref{lst:yolah_class_attributes}.

\begin{mdframed}[skipabove=\baselineskip,hidealllines=true]
\begin{minted}[linenos, breaklines,fontsize=\small]{cpp}
constexpr uint64_t BLACK_INITIAL_POSITION =
0b10000000'00000000'00000000'00001000'00010000'00000000'00000000'00000001;
constexpr uint64_t WHITE_INITIAL_POSITION =
0b00000001'00000000'00000000'00010000'00001000'00000000'00000000'10000000;
class Yolah {
    uint64_t black = BLACK_INITIAL_POSITION;
    uint64_t white = WHITE_INITIAL_POSITION;
    uint64_t holes = 0;
    uint8_t black_score = 0;
    uint8_t white_score = 0;
    uint8_t ply = 0;
public:
    // ...
};
\end{minted}
\vspace{-0.5em}
% \captionof{listing}{Les attributs de la classe Yolah représentant le plateau de jeu}
\captionof{listing}{Attributes of the Yolah class representing the game board}
\label{lst:yolah_class_attributes}
\end{mdframed}

% Les attributs de la classe Yolah sont :
The attributes of the \mintinline{c++}{Yolah} class are:\\
\begin{itemize}
% \item Ligne 6, black : le bitboard pour les pièces noires.
\item Line 6, \mintinline{c++}{black}: the bitboard for black pieces.
% \item Ligne 7, white : le bitboard pour les pièces blanches.
\item Line 7, \mintinline{c++}{white}: the bitboard for white pieces.
% \item Ligne 8, holes : le bitboard pour les trous (les cases détruites).
\item Line 8, \mintinline{c++}{holes}: the bitboard for holes (destroyed squares).
% \item Ligne 9, black_score : le score, ou le nombre de déplacements, du joueur noir.
\item Line 9, \mintinline{c++}{black_score}: the score, or number of moves, of the black player.
% \item Ligne 10, white_score : le score, ou le nombre de déplacements, du joueur blanc.
\item Line 10, \mintinline{c++}{white_score}: the score, or number of moves, of the white player.
% \item Ligne 11, ply : le nombre de coups joués par les deux joueurs depuis le début de la partie.
\item Line 11, \mintinline{c++}{ply}: the number of moves played by both players since the start of the game.\\
\end{itemize}

% Un objet de type Yolah prendra en mémoire sizeof(Yolah) == 32 octets. Notons qu'avec le padding, il n'aurait pas été judicieux d'écrire, par exemple,
A \mintinline{c++}{Yolah} object will occupy \mintinline{c++}{sizeof(Yolah) == 32} bytes in memory. Note that due to padding\footnote{\url{https://en.wikipedia.org/wiki/Data_structure_alignment}.}, it would not have been wise to write, for example,

\begin{minted}[linenos, linenos,breaklines,fontsize=\small]{cpp}
class Yolah {
    uint64_t black = BLACK_INITIAL_POSITION;
    uint8_t black_score = 0;    
    uint64_t white = WHITE_INITIAL_POSITION;
    uint8_t white_score = 0;
    uint64_t holes = 0;
    uint8_t ply = 0;
public:
    // ...
};
\end{minted}

% car avec cette implémentation, nous aurions eu sizeof(Yolah) == 48 octets.
because with this implementation, we would have \mintinline{c++}{sizeof(Yolah) == 48} bytes.\\

% La représentation en bitboard permet d'obtenir efficacement des informations sur le jeu. Par exemple pour voir les cases occupées par les pièces noires et les pièces blanches, il suffit d'effectuer : black | white. Pour les positions des tables, on obtiendrait en une seule opération très efficace les positions des pièces noires et blanches. En notation binaire, on obtient :
The bitboard representation allows for efficiently obtaining game information. For example, to see the squares occupied by black and white pieces, simply perform: \mintinline{c++}{black | white}. For the positions in Tables \ref{tab:black_positions_game_example_move21} and \ref{tab:white_positions_game_example_move21}, we would obtain in a single highly efficient operation\footnote{You can find information on instruction latency and throughput for various processors at: \url{https://agner.org/optimize/instruction_tables.pdf}} the black and white piece positions shown in Table \ref{tab:black_white_positions_game_example_move21}. In binary notation, we get:\\
\begin{minted}[breaklines,fontsize=\small,frame=none]{cpp}
black | white
== 0000001010000000000000010000000000100000000000000000000000000000 |
   0000000000100000000001000000000000000000000000000010100000000000
== 0000001010100000000001010000000000100000000000000010100000000000
\end{minted}

\begin{table}[H]
\centering
% \caption{Positions des pièces noires et blanches que l'on obtient en faisant : black | white}
\caption{Black and white piece positions obtained by computing: \mintinline{c++}{black | white}}
\label{tab:black_white_positions_game_example_move21}
\begin{tabular}{|c|c|c|c|c|c|c|c|c|}
\hline
8 & $\cdot$ & $\bullet$ & $\cdot$ & $\cdot$ & $\cdot$ & $\cdot$ & $\cdot$ & $\cdot$ \\ \hline
7 & $\cdot$ & $\cdot$ & $\cdot$ & $\cdot$ & $\cdot$ & $\circ$ & $\cdot$ & $\bullet$ \\ \hline
6 & $\bullet$ & $\cdot$ & $\circ$ & $\cdot$ & $\cdot$ & $\cdot$ & $\cdot$ & $\cdot$ \\ \hline
5 & $\cdot$ & $\cdot$ & $\cdot$ & $\cdot$ & $\cdot$ & $\cdot$ & $\cdot$ & $\cdot$ \\ \hline
4 & $\cdot$ & $\cdot$ & $\cdot$ & $\cdot$ & $\cdot$ & $\bullet$ & $\cdot$ & $\cdot$ \\ \hline
3 & $\cdot$ & $\cdot$ & $\cdot$ & $\cdot$ & $\cdot$ & $\cdot$ & $\cdot$ & $\cdot$ \\ \hline
2 & $\cdot$ & $\cdot$ & $\cdot$ & $\circ$ & $\cdot$ & $\circ$ & $\cdot$ & $\cdot$ \\ \hline
1 & $\cdot$ & $\cdot$ & $\cdot$ & $\cdot$ & $\cdot$ & $\cdot$ & $\cdot$ & $\cdot$ \\ \hline
\multicolumn{1}{c|}{} & a & b & c & d & e & f & g & h \\
\cline{2-9}
\end{tabular}
\end{table}

% Nous allons continuer d'utiliser des opérations bits-à-bits dans la suite de ce chapitre, pour tester notamment la fin d'une partie et la génération des coups possibles.
We will continue using bitwise operations throughout this chapter, particularly for testing game over conditions and generating possible moves.

\section{Game Over Test}

% Pour tester la fin de partie nous allons avoir besoin de quelques énumérations et constantes et de la fonction shift. La constante NORTH de l'énumération Direction est égale à 8. Pourquoi avoir choisi cette valeur ? Reprenons ci-dessous la table vue au chapitre précédent représentant la position de chaque bit d'un bitboard dans le plateau. On peut voir que si l'on ajoute 8 au numéro de bit contenu dans une des cases, on obtient le numéro de bit de la case au nord de cette dernière (sauf si l'on sort du plateau de jeu). Il en va de même pour les autres valeurs des constantes de l'énumération Direction.
To test for game over, we will need some enumerations and constants (Listings \ref{lst:directions} and \ref{lst:files_and_ranks}) and the \mintinline{c++}{shift} function (Listing \ref{lst:shift}). The \mintinline{c++}{NORTH} constant in the \mintinline{c++}{Direction} enumeration (Listing \ref{lst:directions}) equals $8$. Why this value? Let us revisit below, see Table \ref{tab:bitboard_indices2}, the table from the previous chapter showing the position of each bit in a bitboard on the board. We can see that adding $8$ to the bit number in any square gives us the bit number of the square to the north (unless we exit the board). The same applies to the other constant values in the \mintinline{c++}{Direction} enumeration (Listing \ref{lst:directions}). 
\begin{table}[htpb]
\centering
% \caption{Positions de chaque cases du plateau dans le bitboard. Remarquons que pour une case qui ne se trouve pas dans la rangée 8, bit$_{i+8}$ est le bit correspondant à la case au nord de la case du bit$_i$}
\caption{Position of each board square in the bitboard. Note that for a square not in rank 8, bit$_{i+8}$ corresponds to the square north of bit$_i$}
\label{tab:bitboard_indices2}
\begin{tabular}{|c|c|c|c|c|c|c|c|c|}
\hline
8 & bit$_{56}$ & bit$_{57}$ & bit$_{58}$ & bit$_{59}$ & bit$_{60}$ & bit$_{61}$ & bit$_{62}$ & bit$_{63}$ \\ \hline
7 & bit$_{48}$ & bit$_{49}$ & bit$_{50}$ & bit$_{51}$ & bit$_{52}$ & bit$_{53}$ & bit$_{54}$ & bit$_{55}$ \\ \hline
6 & bit$_{40}$ & bit$_{41}$ & bit$_{42}$ & bit$_{43}$ & bit$_{44}$ & bit$_{45}$ & bit$_{46}$ & bit$_{47}$ \\ \hline
5 & bit$_{32}$ & bit$_{33}$ & bit$_{34}$ & bit$_{35}$ & bit$_{36}$ & bit$_{37}$ & bit$_{38}$ & bit$_{39}$ \\ \hline
4 & bit$_{24}$ & bit$_{25}$ & bit$_{26}$ & bit$_{27}$ & bit$_{28}$ & bit$_{29}$ & bit$_{30}$ & bit$_{31}$ \\ \hline
3 & bit$_{16}$ & bit$_{17}$ & bit$_{18}$ & bit$_{19}$ & bit$_{20}$ & bit$_{21}$ & bit$_{22}$ & bit$_{23}$ \\ \hline
2 & bit$_{8}$ & bit$_{9}$ & bit$_{10}$ & bit$_{11}$ & bit$_{12}$ & bit$_{13}$ & bit$_{14}$ & bit$_{15}$ \\ \hline
1 & bit$_{0}$ & bit$_{1}$ & bit$_{2}$ & bit$_{3}$ & bit$_{4}$ & bit$_{5}$ & bit$_{6}$ & bit$_{7}$ \\ \hline
\multicolumn{1}{c|}{} & a & b & c & d & e & f & g & h \\
\cline{2-9}
\end{tabular}
\end{table}

\begin{mdframed}[skipabove=\baselineskip,hidealllines=true]
\begin{minted}[linenos, linenos,breaklines,fontsize=\small]{cpp}
enum Direction : int8_t {
    NORTH = 8,
    EAST  = 1,
    SOUTH = -NORTH,
    WEST  = -EAST,
    NORTH_EAST = NORTH + EAST,
    SOUTH_EAST = SOUTH + EAST,
    SOUTH_WEST = SOUTH + WEST,
    NORTH_WEST = NORTH + WEST
};
\end{minted}
\captionof{listing}{Directions}
\label{lst:directions}
\end{mdframed}

\begin{mdframed}[skipabove=\baselineskip,hidealllines=true]
\begin{minted}[linenos,breaklines,fontsize=\small]{cpp}
template<Direction D>
constexpr uint64_t shift(uint64_t b) {
    if constexpr (D == NORTH)
        return b << NORTH;
    else if constexpr (D == SOUTH)      
        return b >> -SOUTH;
    else if constexpr (D == EAST)       
        return (b & ~FileHBB) << EAST;
    else if constexpr (D == WEST)       
        return (b & ~FileABB) >> -WEST;
    else if constexpr (D == NORTH_EAST) 
        return (b & ~FileHBB) << NORTH_EAST;
    else if constexpr (D == NORTH_WEST) 
        return (b & ~FileABB) << NORTH_WEST;
    else if constexpr (D == SOUTH_EAST) 
        return (b & ~FileHBB) >> -SOUTH_EAST;
    else if constexpr (D == SOUTH_WEST) 
        return (b & ~FileABB) >> -SOUTH_WEST;
    else return 0;
}
\end{minted}
% \captionof{listing}{Décalage selon une direction}
\captionof{listing}{Shift by direction}
\label{lst:shift}
\end{mdframed}


% La fonction shift(uint64_t b) donnée dans le listing nous permet de décaler tous les bits à 1 du bitboard en paramètre dans la direction D. Grâce aux if constexpr, lorsque l'on effectuera shift<NORTH>(b) par exemple, le compilateur transformera le code en : return b << NORTH;.
The \mintinline{c++}{shift(uint64_t b)} function given in Listing \ref{lst:shift} shifts all 1-bits in the bitboard parameter in direction \mintinline{c++}{D}\footnote{Thanks to \mintinline{c++}{if constexpr}, when we call \mintinline{c++}{shift<NORTH>(b)} for example, the compiler will transform the code to: \mintinline{c++}{return b << NORTH;}.}. Given
\begin{minted}[breaklines,fontsize=\small,frame=none]{cpp}
black == 1000000000000000000000000000100000100000000000000000000000000001
\end{minted}
% le bitboard représenté dans la table. Pour décaler les pièces noires d'une case vers le nord, on effectue : shift<NORTH>(black). On obtient alors le bitboard représenté dans la table. Notons que la pièce en h8 n'est plus sur le plateau.
the bitboard shown in Table \ref{tab:shift_north_before}. To shift the black pieces one square north, we perform: \mintinline{c++}{shift<NORTH>(black)}. We then get the bitboard
  \begin{minted}[breaklines,fontsize=\small,frame=none]{cpp}
1000000000000000000000000000100000100000000000000000000000000001
\end{minted}
shown in Table \ref{tab:shift_north_after}. Note that the piece on {\ttfamily h8} is no longer on the board.\\

\begin{table}[H]
\centering
\begin{minipage}{0.45\textwidth}
\centering
% \caption{Plateau de jeu avant d'effectuer shift<NORTH>}
\caption{Game board before applying \mintinline{c++}{shift<NORTH>}}
\label{tab:shift_north_before}
\begin{tabular}{|c|c|c|c|c|c|c|c|c|}
\hline
8 & $\cdot$ & $\cdot$ & $\cdot$ & $\cdot$ & $\cdot$ & $\cdot$ & $\cdot$ & $\bullet$ \\ \hline
7 & $\cdot$ & $\cdot$ & $\cdot$ & $\cdot$ & $\cdot$ & $\cdot$ & $\cdot$ & $\cdot$ \\ \hline
6 & $\cdot$ & $\cdot$ & $\cdot$ & $\cdot$ & $\cdot$ & $\cdot$ & $\cdot$ & $\cdot$ \\ \hline
5 & $\cdot$ & $\cdot$ & $\cdot$ & $\bullet$ & $\cdot$ & $\cdot$ & $\cdot$ & $\cdot$ \\ \hline
4 & $\cdot$ & $\cdot$ & $\cdot$ & $\cdot$ & $\cdot$ & $\bullet$ & $\cdot$ & $\cdot$ \\ \hline
3 & $\cdot$ & $\cdot$ & $\cdot$ & $\cdot$ & $\cdot$ & $\cdot$ & $\cdot$ & $\cdot$ \\ \hline
2 & $\cdot$ & $\cdot$ & $\cdot$ & $\cdot$ & $\cdot$ & $\cdot$ & $\cdot$ & $\cdot$ \\ \hline
1 & $\bullet$ & $\cdot$ & $\cdot$ & $\cdot$ & $\cdot$ & $\cdot$ & $\cdot$ & $\cdot$ \\ \hline
\multicolumn{1}{c|}{} & a & b & c & d & e & f & g & h \\
\cline{2-9}
\end{tabular}
\end{minipage}
\hfill
\begin{minipage}{0.45\textwidth}
\centering
% \caption{Plateau de jeu après avoir effectué shift<NORTH>}
\caption{Game board after applying \mintinline{c++}{shift<NORTH>}}
\label{tab:shift_north_after}
\begin{tabular}{|c|c|c|c|c|c|c|c|c|}
\hline
8 & $\cdot$ & $\cdot$ & $\cdot$ & $\cdot$ & $\cdot$ & $\cdot$ & $\cdot$ & $\cdot$ \\ \hline
7 & $\cdot$ & $\cdot$ & $\cdot$ & $\cdot$ & $\cdot$ & $\cdot$ & $\cdot$ & $\cdot$ \\ \hline
6 & $\cdot$ & $\cdot$ & $\cdot$ & $\bullet$ & $\cdot$ & $\cdot$ & $\cdot$ & $\cdot$ \\ \hline
5 & $\cdot$ & $\cdot$ & $\cdot$ & $\cdot$ & $\cdot$ & $\bullet$ & $\cdot$ & $\cdot$ \\ \hline
4 & $\cdot$ & $\cdot$ & $\cdot$ & $\cdot$ & $\cdot$ & $\cdot$ & $\cdot$ & $\cdot$ \\ \hline
3 & $\cdot$ & $\cdot$ & $\cdot$ & $\cdot$ & $\cdot$ & $\cdot$ & $\cdot$ & $\cdot$ \\ \hline
2 & $\bullet$ & $\cdot$ & $\cdot$ & $\cdot$ & $\cdot$ & $\cdot$ & $\cdot$ & $\cdot$ \\ \hline
1 & $\cdot$ & $\cdot$ & $\cdot$ & $\cdot$ & $\cdot$ & $\cdot$ & $\cdot$ & $\cdot$ \\ \hline
\multicolumn{1}{c|}{} & a & b & c & d & e & f & g & h \\
\cline{2-9}
\end{tabular}
\end{minipage}
\end{table}

% On peut ainsi facilement obtenir toutes les cases directement au contact des pièces d'un plateau donné board, en effectuant l'opération suivante :
We can thus easily obtain all squares directly adjacent to the pieces on a given \mintinline{c++}{board} by performing the following operation:

\begin{minted}[linenos, linenos,breaklines,fontsize=\small]{cpp}
shift<NORTH>(board) | shift<SOUTH>(board) | shift<EAST>(board) | 
shift<WEST>(board) | shift<NORTH_EAST>(board) | shift<NORTH_WEST>(board) |
shift<SOUTH_EAST>(board) | shift<SOUTH_WEST>(board)
\end{minted}

% Pour le plateau de la table on obtient les positions représentées par des étoiles dans la table.
For the board in Table \ref{tab:before_all_shift}, we get the positions represented by stars in Table \ref{tab:after_all_shift}.\\

\begin{table}[H]
\centering
\begin{minipage}{0.45\textwidth}
\centering
% \caption{Plateau de jeu}
\caption{Game board}
\label{tab:before_all_shift}
\begin{tabular}{|c|c|c|c|c|c|c|c|c|}
\hline
8 & $\circ$ & $\cdot$ & $\cdot$ & $\cdot$ & $\cdot$ & $\cdot$ & $\cdot$ & $\bullet$ \\ \hline
7 & $\cdot$ & $\bullet$ & $\cdot$ & $\cdot$ & $\cdot$ & $\cdot$ & $\cdot$ & $\cdot$ \\ \hline
6 & $\cdot$ & $\cdot$ & $\cdot$ & $\cdot$ & $\cdot$ & $\cdot$ & $\cdot$ & $\cdot$ \\ \hline
5 & $\cdot$ & $\cdot$ & $\cdot$ & \raisebox{-0.3ex}{$\blacksquare$} & $\circ$ & $\cdot$ & $\cdot$ & $\cdot$ \\ \hline
4 & $\cdot$ & $\cdot$ & $\cdot$ & \raisebox{-0.3ex}{$\blacksquare$} & \raisebox{-0.3ex}{$\blacksquare$} & $\bullet$ & $\cdot$ & $\cdot$ \\ \hline
3 & $\cdot$ & $\cdot$ & $\cdot$ & $\cdot$ & $\cdot$ & $\cdot$ & $\cdot$ & $\cdot$ \\ \hline
2 & $\cdot$ & $\cdot$ & $\cdot$ & $\circ$ & $\cdot$ & $\cdot$ & $\cdot$ & $\cdot$ \\ \hline
1 & $\bullet$ & $\cdot$ & $\cdot$ & $\cdot$ & $\cdot$ & $\cdot$ & $\cdot$ & $\circ$ \\ \hline
\multicolumn{1}{c|}{} & a & b & c & d & e & f & g & h \\
\cline{2-9}
\end{tabular}
\end{minipage}
\hfill
\begin{minipage}{0.45\textwidth}
\centering
% \caption{Positions autour des pièces}
\caption{Positions around the pieces}
\label{tab:after_all_shift}
\begin{tabular}{|c|c|c|c|c|c|c|c|c|}
\hline
8 & $\ast$ & $\ast$ & $\ast$ & $\cdot$ & $\cdot$ & $\cdot$ & $\ast$ & $\cdot$ \\ \hline
7 & $\ast$ & $\ast$ & $\ast$ & $\cdot$ & $\cdot$ & $\cdot$ & $\ast$ & $\ast$ \\ \hline
6 & $\ast$ & $\ast$ & $\ast$ & $\ast$ & $\cdot$ & $\ast$ & $\cdot$ & $\cdot$ \\ \hline
5 & $\cdot$ & $\cdot$ & $\cdot$ & $\ast$ & $\ast$ & $\ast$ & $\ast$ & $\cdot$ \\ \hline
4 & $\cdot$ & $\cdot$ & $\cdot$ & $\ast$ & $\ast$ & $\ast$ & $\ast$ & $\cdot$ \\ \hline
3 & $\cdot$ & $\cdot$ & $\ast$ & $\ast$ & $\ast$ & $\ast$ & $\ast$ & $\cdot$ \\ \hline
2 & $\ast$ & $\ast$ & $\ast$ & $\cdot$ & $\ast$ & $\cdot$ & $\ast$ & $\ast$ \\ \hline
1 & $\cdot$ & $\ast$ & $\ast$ & $\ast$ & $\ast$ & $\cdot$ & $\ast$ & $\cdot$ \\ \hline
\multicolumn{1}{c|}{} & a & b & c & d & e & f & g & h \\
\cline{2-9}
\end{tabular}
\end{minipage}
\end{table}


% Dans le code de la fonction shift on peut voir l'utilisation des constantes FileHBB et FileABB (BB pour bitboard). Ces constantes vont nous permettre de masquer certains bits qui se retrouveraient après décalage dans une position incorrecte. Par exemple, si nous décalons dans la direction EAST le bitboard représenté dans la table, le pion blanc en h1 se retrouverait en a2. Pour éviter cela, avant le décalage, on élimine les éléments de la colonne h pour qu'ils ne se retrouvent pas dans la colonne a. Les constantes FileABB et FileHBB ainsi que les énumérations pour les rangées et colonnes sont définies dans le listing.
In the \mintinline{c++}{shift} function code (Listing \ref{lst:shift}), we can see the use of constants \mintinline{c++}{FileHBB} and \mintinline{c++}{FileABB} ({\ttfamily BB} for bitboard). These constants allow us to mask certain bits that would end up in incorrect positions after shifting. For example, if we shift the bitboard shown in Table \ref{tab:before_all_shift} in the \mintinline{c++}{EAST} direction, the white pawn on {\ttfamily h1} would end up on {\ttfamily a2}. To avoid this, before shifting, we eliminate elements from column {\ttfamily h} so they don't end up in column {\ttfamily a} (line 8 of Listing \ref{lst:shift}). The constants \mintinline{c++}{FileABB} and \mintinline{c++}{FileHBB} along with the rank and file enumerations are defined in Listing \ref{lst:files_and_ranks}.    

\begin{mdframed}[skipabove=\baselineskip,hidealllines=true]
\begin{minted}[linenos, linenos,breaklines,fontsize=\small]{cpp}
enum File : uint8_t {
    FILE_A, FILE_B, FILE_C, FILE_D, FILE_E, FILE_F, FILE_G, FILE_H,
    FILE_NB
};
enum Rank : uint8_t {
    RANK_1, RANK_2, RANK_3, RANK_4, RANK_5, RANK_6, RANK_7, RANK_8,
    RANK_NB
};
constexpr uint64_t FileABB = 0x0101010101010101;
//   0x0101010101010101
//== 0b0000000100000001000000010000000100000001000000010000000100000001  
//            a8      a7      a6      a5      a4      a3      a2      a1 

constexpr uint64_t FileHBB = FileABB << 7;
//   0x8080808080808080
//== 0b1000000010000000100000001000000010000000100000001000000010000000
//     h8      h7      h6      h5      h4      h3      h2      h1 
\end{minted}
% \captionof{listing}{Colonnes et rangées}
\captionof{listing}{Files and ranks}
\label{lst:files_and_ranks}
\end{mdframed}

% La fonction qui teste la fin de partie s'appelle game_over et elle est décrite dans le listing. Son implémentation est assez simple,
The function that tests for game over is called \mintinline{c++}{game_over} and is described in Listing \ref{lst:game_over}. Its implementation is straightforward:

\begin{itemize}

% \item À la ligne 2, on récupère dans possible le bitboard où il y a des 1 dans les positions libres.
\item On line 2, we retrieve in \mintinline{c++}{possible} the bitboard with $1$s in free positions.

% \item À la ligne 3, on crée le bitboard players qui contient des 1 dans les positions occupées par l'un des joueurs.
\item On line 3, we create the \mintinline{c++}{players} bitboard containing $1$s in positions occupied by either player.

% \item Les lignes 4 à 8 permettent de créer le bitboard around_players qui contient des 1 aux positions autour de chacune des pièces des joueurs.
\item Lines 4 to 8 create the \mintinline{c++}{around_players} bitboard containing $1$s at positions around each player's pieces.

% \item Enfin, à la ligne 8, on teste s'il n'existe aucune position libre à côté d'un des joueurs. En effet, le et bit-à-bit ne gardera un 1 dans un bit du résultat, que si et seulement s'il y a un 1 à cette position dans le bitboard possible et dans le bitboard around_players. Donc que la case correspondante est libre et accessible par le joueur. Si le résultat de around_players \& possible vaut 0, cela veut dire qu'aucune position n'est à la fois à côté d'un des joueurs et libre.
\item Finally, on line 9, we test whether no free position exists adjacent to either player. The bitwise AND will keep a $1$ in a result bit if and only if there is a $1$ at that position in both the \mintinline{c++}{possible} and \mintinline{c++}{around_players} bitboards—meaning the corresponding square is both free and accessible by a player. If \mintinline{c++}{around_players & possible} equals $0$, it means no position is both adjacent to a player and free.

\end{itemize}

\begin{mdframed}[skipabove=\baselineskip,hidealllines=true]
\begin{minted}[linenos, breaklines,fontsize=\small]{cpp}
bool Yolah::game_over() const {
    uint64_t possible = ~holes & ~black & ~white;
    uint64_t players  = black | white;
    uint64_t around_players = shift<NORTH>(players) | 
        shift<SOUTH>(players) | shift<EAST>(players) | 
        shift<WEST>(players) | shift<NORTH_EAST>(players) | 
        shift<NORTH_WEST>(players) | shift<SOUTH_EAST>(players) | 
        shift<SOUTH_WEST>(players);
    return (around_players & possible) == 0;    
}
\end{minted}
% \captionof{listing}{Test de fin de partie}
\captionof{listing}{Game over test}
\label{lst:game_over}
\end{mdframed}

% Maintenant que nous savons tester la fin d'une partie, nous allons étudier comment générer efficacement les coups possibles dans une position donnée.
Now that we know how to test for game over, we will study how to efficiently generate possible moves in a given position.

\section{Generating Possible Piece Moves}

% La génération des coups possibles va utiliser une technique appelée les magic bitboard. Pour comprendre cette technique je vais tout d'abord présenter sur un exemple simple le hachage parfait magique. Je me suis inspiré de l'article pour cet exemple.
Move generation will use a technique called \emph{magic bitboards}. To understand this technique, I will first present magic perfect hashing through a simple example. I drew inspiration from \cite{kannan2008magicbitboards} for this example.

\subsection{Magic Perfect Hashing Function}

% Supposons que nous devions trouver la position du seul bit à 1 dans un entier long non-signé (uint64_t). Par exemple, si nous appelons position la fonction permettant de calculer cette position, nous obtenons les résultats suivants :
Suppose we need to find the position of the only 1-bit in an unsigned long integer (\mintinline{c++}{uint64_t}). For example, if we call \mintinline{c++}{position} the function that calculates this position, we get the following results:

\begin{minted}[breaklines,fontsize=\small,frame=none]{cpp}
    position(1)          == 0
    position(0b1000)     == 3
    position(0b10000000) == 7
    position(0x8000000000000000) == 63
\end{minted}

% Notons que les processeurs possèdent des instructions permettant de calculer efficacement cette fonction, et le compilateur gcc permet de calculer efficacement cette position avec la fonction __builtin_ctzll(unsigned long long x), par exemple,
Note that processors have instructions to efficiently compute this function\footnote{\mintinline{asm}{tzcnt} on Intel.}, and the {\ttfamily gcc} compiler allows efficient calculation of this position with the function\\ \mintinline{c++}{__builtin_ctzll(unsigned long long x)}\footnote{Returns the number of trailing 0-bits in x, starting at the least significant bit position. If x is 0, the result is undefined.}, for example,

\begin{minted}[breaklines,fontsize=\small,frame=none]{cpp}
    position(0b1000) == __builtin_ctzll(0b1000)
\end{minted}

% Mais nous allons coder cette fonction position en utilisant une fonction magique de hachage parfait. Cette fonction nous servira pour la génération des coups possibles par la suite. En utilisant une table de hachage classique, nous pouvons tout d'abord remplir celle-ci avec les 64 valeurs et simplement consulter cette table par la suite, comme dans le code suivant :
However, we will code this \mintinline{c++}{position} function using a magic perfect hashing function. This function will be useful for move generation later. Using a standard hash table, we could first fill it with 64 values and then simply look up this table, as in the following code:

\begin{minted}[linenos, breaklines,fontsize=\small]{cpp}
unordered_map<uint64_t, uint8_t> table {
    {1, 0}, {0b10, 1}, {0b100, 2}, {0b1000, 3}, //...
}
int position(uint64_t x) {
    return table[x];
}
\end{minted}

% La consultation de cette table de hachage est bien plus coûteuse que l'indexation d'un simple tableau. Mais on ne peut pas directement utiliser un tableau, car les valeurs pour indexer celui-ci sont dans une plage de valeurs trop grande. L'idée c'est de trouver une fonction de hachage efficace pour transformer chacune des 64 valeurs, que nous allons appeler clés, dans la plage [0, 63].
Looking up values in this hash table is far more costly than indexing a simple array\footnote{We will measure the performance gain from magic perfect hashing compared to using a standard hash table at the end of this section.}. However, we cannot directly use an array because the indexing values span too large a range—for example, the value \mintinline{c++}{0x8000000000000000} ($9,223,372,036,854,775,808$!). The idea is to find an efficient hash function that transforms each of the $64$ values\footnote{\mintinline{c++}{1, 0b10, 0b100, 0b1000, 0b10000, ...}.}—which we call bitboards—into the range $[0, 2^6 - 1 = 63]$.
% De plus on veut qu'il n'y ait aucune collision, on parle alors d'un hachage parfait. S'il y a des collisions des positions seraient erronées. Par exemple, si les clés 0b100 et 0b1000000 étaient transformées en l'indice 42, on devrait mettre dans table[42] la valeur 2 ou 6, mais on ne peut pas stocker plus d'une valeur au même emplacement.
Moreover, we want no collisions—this is called perfect hashing. If there were collisions, positions would be incorrect. For example, if keys \mintinline{c++}{0b100} and \mintinline{c++}{0b1000000} were both transformed to index $42$, we would need to store value $2$ and $6$ in \mintinline{c++}{table[42]}, but we cannot store more than one value at the same location.\\
% La fonction de hachage parfait magique va avoir la forme suivante :
The magic perfect hashing function will have the following form (listing \ref{lst:magic_perfect_hashing}):

\begin{mdframed}[skipabove=\baselineskip,hidealllines=true]
\begin{minted}[linenos,breaklines,fontsize=\small]{cpp}
constexpr uint64_t MAGIC = //...
constexpr int K = 6;
int magic_perfect_hashing(uint64_t bitboard) {
    return bitboard * MAGIC >> (64 - K);
}
\end{minted}
\vspace{-0.5em}
\captionof{listing}{Magic bitboard perfect hashing}
\label{lst:magic_perfect_hashing}
\end{mdframed}

% Dans cette fonction,
In this function:

\begin{itemize}

% \item À la ligne 2, la constante K donne le nombre de bits pour notre index. Ici avec K == 6, cela nous donne un maximum de 2^K = 2^6 = 64 valeurs possibles dans la table.
\item On line 2, the constant \mintinline{c++}{K} gives the number of bits for our index. Here with \mintinline{c++}{K == 6}, this gives us a maximum of $2^K = 2^6 = 64$ possible values in the table.

% \item À la ligne 4, on peut voir la formule pour calculer l'index dans la table selon la clé uint64_t key donnée en paramètre. On multiplie la clé par la constante MAGIC et ensuite on décale vers la droite pour ne garder que les K bits de poids forts du résultat. Cette opération est très peu coûteuse, mais comment trouver cette constante magique ?
\item On line 4, we see the formula for calculating the table index from the \mintinline{c++}{uint64_t key} parameter. We multiply the key by the \mintinline{c++}{MAGIC} constant and then right-shift to keep only the \mintinline{c++}{K} most significant bits of the result. This operation is very inexpensive, but how do we find this magic constant?\\

\end{itemize}

% Une méthode simple pour essayer de trouver la constante MAGIC est de la générer au hasard jusqu'à trouver une valeur qui ne crée aucune collision ! Ce genre d'approche est utilisée dans Stockfish pour générer les magic bitboard. Le listing montre cette approche.
A simple method to find the \mintinline{c++}{MAGIC} constant is to generate it randomly until we find a value that produces no collisions! This approach is used in Stockfish \cite{stockfish2025} to generate magic bitboards. Listing \ref{lst:random_magic} shows this approach.

\begin{mdframed}[skipabove=\baselineskip,hidealllines=true]
\begin{minted}[linenos, breaklines,fontsize=\small]{cpp}
#include <bits/stdc++.h>

using namespace std;

int main() {
    random_device rd;
    mt19937_64 mt(rd());
    uniform_int_distribution<uint64_t> d;
    unordered_map<uint64_t, int> positions;
    for (int i = 0; i < 64; i++) {
        positions[1ULL << i] = i;
    }
    constexpr int K = 7;
    auto index = [](uint64_t magic, int k, uint64_t bitboard) {
        return bitboard * magic >> (64 - k);
    };
    while (true) {
        uint64_t MAGIC = d(mt);
        bool found = true;
        set<uint64_t> seen;
        for (const auto [bitboard, pos] : positions) {
            int64_t i = index(MAGIC, K, bitboard);
            if (seen.contains(i)) {
                found = false;
                break;
            }
            seen.insert(i);
        }
        if (found) {
            cout << format("found magic for K = {}: {:#x}\n", K, MAGIC);
            int size = 1 << K;
            vector<int> table(size, -1);
            for (const auto [bitboard, pos] : positions) {
                table[index(MAGIC, K, bitboard)] = pos;
            }
            cout << format("uint8_t positions[{}] = {{", size);
            for (int i = 0; i < size; i++) {
                cout << table[i] << ',';
            }
            cout << "};\n";
            break;
        }
    }
}
\end{minted}
% \captionof{listing}{Recherche aléatoire de la constante MAGIC}
\captionof{listing}{Random search for the \mintinline{c++}{MAGIC} constant}
\label{lst:random_magic}
\end{mdframed}

\begin{itemize}

% \item Aux lignes 6 à 8, nous mettons en place un générateur aléatoire pour générer aléatoirement des uint64_t.
\item On lines 6 to 8, we set up a random generator to randomly generate \mintinline{c++}{uint64_t} values.

% \item À la ligne 9, la table de correspondance positions va associer à un entier long ne contenant qu'un seul bit à 1, la position de ce dernier. Cette table est initialisée aux lignes 10 à 12.
\item On line 9, the lookup table \mintinline{c++}{positions} associates each long integer containing only one 1-bit with its position. This table is initialized on lines 10 to 12.

% \item À la ligne 13, la constante K est le nombre de bits de notre clé obtenue par hachage parfait magique. Il faut que K permette de représenter toutes les valeurs que l'on doit couvrir. Par exemple, ici nous avons 64 valeurs possibles, car nous avons 64 positions possibles pour le bit à 1 dans un entier long sur 64 bits. Avec K == 7, on peut couvrir 2^7, soit 128 valeurs différentes. Notons que c'est plus que les 64 valeurs dont nous avons besoin et que K == 6 suffirait (K == 5 serait trop petit). Mais en essayant de trouver la constante MAGIC au hasard pour K == 6, ce programme ne trouvait pas en un temps raisonnable (nous verrons après une autre façon de faire pour réussir pour K == 6).
\item On line 13, the constant \mintinline{c++}{K} is the number of bits in our key obtained by magic perfect hashing. \mintinline{c++}{K} must be large enough to represent all values we need to cover. For example, here we have $64$ possible values since we have $64$ possible positions for the 1-bit in a $64$-bit long integer. With \mintinline{c++}{K == 7}, we can cover $2^7 = 128$ different values. Note that this is more than the $64$ values we need, and \mintinline{c++}{K == 6} would suffice (\mintinline{c++}{K == 5} would be too small). However, when trying to find the \mintinline{c++}{MAGIC} constant randomly for \mintinline{c++}{K == 6}, this program could not find one in reasonable time \frownie{} (we will see another approach to succeed with \mintinline{c++}{K == 6} \smiley).

% \item Aux lignes 14 à 16, la fonction index calcule l'index de clé bitboard (ce bitboard contient un seul bit à 1) en utilisant la formule pour notre hachage parfait magique.
\item On lines 14 to 16, the \mintinline{c++}{index} function calculates the index for key \mintinline{c++}{bitboard} (this bitboard contains only one 1-bit) using the magic perfect hashing formula.

% \item La boucle while, lignes 17 à 43, va boucler jusqu'à trouver la constante MAGIC qui permet de réaliser un hachage parfait magique (du coup cette boucle peut être infinie).
\item The \mintinline{c++}{while} loop, lines 17 to 43, loops until it finds a \mintinline{c++}{MAGIC} constant that achieves magic perfect hashing (this loop could potentially run forever).

% \item À la ligne 18, on crée la valeur MAGIC au hasard.
\item On line 18, we create the \mintinline{c++}{MAGIC} value randomly.

% \item Le set<uint64_t> de la ligne 20 va nous permettre de mémoriser les index (obtenues grâce à la fonction index) que nous avons déjà utilisés pour vérifier que nous n'avons pas de collisions.
\item The \mintinline{c++}{set<uint64_t>} on line 20 allows us to remember the indices (obtained via the \mintinline{c++}{index} function) we have already used, to verify we have no collisions.

% \item Aux lignes 21 à 28, la boucle for va tester tous les bitboards, trouver l'index de chacun d'entre eux grâce au hachage (fonction index) et vérifier que nous n'avons pas de collision (ligne 23) avec les index obtenus pour les bitboard précédents.
\item On lines 21 to 28, the \mintinline{c++}{for} loop tests all bitboards, finds the index for each using the hash (\mintinline{c++}{index} function), and verifies there are no collisions (line 23) with indices obtained for previous bitboards.

% \item Enfin, aux lignes 29 à 42, si nous avons trouvé une constante MAGIC qui permet de créer le hachage parfait magique, nous affichons celle-ci puis nous affichons un tableau en C++ qui contient la position du bit à 1 pour chacun des bitboard. La sortie de ce programme, pour une exécution donnée, est donnée ci-dessous.
\item Finally, on lines 29 to 42, if we found a \mintinline{c++}{MAGIC} constant that creates magic perfect hashing, we display it and then display a \emph{C++} array containing the position of the 1-bit for each bitboard. The output of this program for a given execution is shown below. 

\begin{minted}[breaklines,fontsize=\small]{cpp}
found magic for K = 7: 0x65e4d4ee86638416
uint8_t positions[128] = {
    63, -1, 54, -1, 49, 55, 33, -1, 50, -1, -1, 56, 34, -1, 43, -1,
    51, -1, -1, 11, -1, -1, 57,  3, 39, 35, 14, -1, 44, 22, -1, -1,
    52, 31, -1, -1, -1, -1, 12, 20, -1, 18, -1, -1, 58, -1, -1,  4,
    60, 40,  0, 36, -1, 15, -1, -1, 45, -1, 27, 23,  6, -1, -1, -1,
    62, 53, 48, 32, -1, -1, -1, 42, -1, 10, -1,  2, 38, 13, 21, -1,
    30, -1, -1, 19, 17, -1, -1, -1, 59, -1, -1, -1, -1, 26,  5, -1,
    61, 47, -1, 41,  9,  1, 37, -1, 29, -1, 16, -1, -1, -1, 25, -1,
    46, -1,  8, -1, 28, -1, -1, 24, -1,  7, -1, -1, -1, -1, -1, -1
};
\end{minted}

% Pour obtenir la position du bit à 1 pour le bitboard 0b1000 par exemple, il faut donc consulter la valeur suivante :
To get the position of the 1-bit for bitboard \mintinline{c++}{0b1000} for example, we need to look up the following value:

\begin{minted}[breaklines,fontsize=\small]{cpp}
   positions[0b1000 * 0x65e4d4ee86638416 >> (64 - 7)]
== positions[0x2f26a774331c20b0 >> 57]
== positions[0x17]
== positions[23]
== 3
\end{minted}

% On remarque qu'il y a des places vacantes dans le tableau positions, celles contenant la valeur -1, donc on a perdu de la place. Nous n'aurions gaspillé aucune place si nous avions trouvé une constante MAGIC pour K == 6. Mais est-ce possible ?
Notice that there are vacant slots in the \mintinline{c++}{positions} array—those containing $-1$—so we wasted space. We would not have wasted any space if we had found a \mintinline{c++}{MAGIC} constant for \mintinline{c++}{K == 6}. But is this possible?

\end{itemize}

% Pour répondre à cette question nous allons procéder d'une autre manière. Nous ne voulons pas, bien entendu, balayer les 2^64 valeurs possibles puis à chaque fois tester si la valeur permet le hachage parfait magique. Nous allons utiliser un solveur SMT qui va nous permettre de fixer des contraintes avant de balayer tout l'espace de recherche. Le fait de fixer tout d'abord des contraintes va permettre d'élaguer l'espace de recherche. Notons que nous n'avons pas la garantie que la recherche soit rapide, mais pour ce problème, l'obtention d'une constante MAGIC pour K == 6 sera très rapide. Si vous êtes intéressé par le fonctionnement d'un solveur SMT vous pouvez consulter les références.
To answer this question, we will proceed differently. Obviously, we do not want to scan through all $2^{64}$ ($9, 223, 372, 036, 854, 775, 808$) possible values and test each one for magic perfect hashing. We will use an \gls{smt} solver that allows us to set constraints before searching the entire space. Setting constraints first will prune the search space. Note that we have no guarantee the search will be fast, but for this problem, finding a \mintinline{c++}{MAGIC} constant for \mintinline{c++}{K == 6} will be very quick. If you are interested in how an \gls{smt} solver works, you can consult \cite{kroening2016decision} and \cite{biere2021handbook}.\\

% Le code du listing décrit l'approche utilisant le solveur Z3.
The code in Listing \ref{lst:z3_magic} describes the approach using the \gls{z3} solver \cite{moura2008z3}.

\begin{mdframed}[skipabove=\baselineskip,hidealllines=true]
\begin{minted}[linenos, breaklines,fontsize=\small]{python}
from z3 import *

positions = {}
for i in range(64):
    positions[1 << i] = i

solver = Solver()
MAGIC  = BitVec('magic', 64)
K = 6
bitboards = list(positions.keys())

def index(magic, k, bitboard):
    return magic * bitboard >> (64 - k)

for i in range(64):
    index1 = index(MAGIC, K, bitboards[i])
    for j in range(i + 1, 64):
        index2 = index(MAGIC, K, bitboards[j])
        solver.add(index1 != index2)

if solver.check() == sat:
    model = solver.model()
    m = model[MAGIC].as_long()
    print(f'found magic for K = {K}: {m:#x}')
    size = 1 << K
    table = [-1] * size
    for bitboard, pos in positions.items():
        table[index(m, K, bitboard) & size - 1] = pos
    print(f'constexpr uint8_t positions[{size}] = {{')
    for i in range(size):
        print(f'{table[i]},', end='')
    print('\n};')
\end{minted}
% \captionof{listing}{Recherche avec un solveur SMT de la constante MAGIC}
\captionof{listing}{Searching for the \mintinline{c++}{MAGIC} constant with an \gls{smt} solver}
\label{lst:z3_magic}
\end{mdframed}

\begin{itemize}

% \item Les lignes 3 à 5 permettent de faire correspondre à un bitboard donné ne contenant qu'un bit à 1, la position correspondante de ce bit. On utilise pour ce faire le dictionnaire positions (ligne 3).
\item Lines 3 to 5 associate each bitboard containing only one 1-bit with the corresponding position of that bit. We use the \mintinline{python}{positions} dictionary (line 3) for this.

% \item À la ligne 7, on instantie le solveur Z3.
\item On line 7, we instantiate the \gls{z3} solver.

% \item À la ligne 8, on déclare que la constante MAGIC est du type z3.BitVec qui est un type proposé par le solveur Z3 qui permet de représenter des vecteurs de bits, ici nous utilisons un vecteur de 64 bits.
\item On line 8, we declare that the \mintinline{python}{MAGIC} constant is of type \mintinline{python}{z3.BitVec}, a type provided by the \gls{z3} solver for representing bit vectors—here we use a $64$-bit vector.

% \item La fonction index définie aux lignes 12 et 13 permet de calculer le hachage en fonction de la constante MAGIC et de K.
\item The \mintinline{python}{index} function defined on lines 12 and 13 calculates the hash based on the \mintinline{python}{MAGIC} constant and \mintinline{python}{K}.

% \item Aux lignes 15 à 19 on donne au solver les contraintes que les valeurs hachées par la fonction index de deux bitboards différents ne doivent pas se retrouver au même emplacement. Cette contrainte permet d'assurer que le hachage est parfait.
\item On lines 15 to 19, we give the solver the constraints that hashed values from the \mintinline{python}{index} function for two different bitboards must not map to the same location. This constraint ensures perfect hashing.

% \item À la ligne 21 on execute le solveur et il nous indique par la valeur sat s'il a trouvé une valeur pour MAGIC qui satisfait les contraintes. Notons que si le solveur répond unsat à la place de sat, cela veut dire qu'il n'existe aucun entier sur 64 bits qui permette le hachage parfait. Même si l'espace de recherche est très grand, sur ma machine, l'obtention d'une solution se fait en moins d'une seconde !
\item On line 21, we run the solver, which returns \mintinline{python}{sat} if it found a \mintinline{python}{MAGIC} value satisfying the constraints. Note that if the solver returns \mintinline{python}{unsat} instead of \mintinline{python}{sat}, it means no $64$-bit integer allows perfect hashing. Even though the search space is very large, on my machine, finding a solution takes less than a second!

% \item Les lignes 22 à 32 permettent d'afficher sur la sortie standard la valeur MAGIC ainsi que le tableau C++ des valeurs obtenues. Ce tableau est donné ci-dessous.
\item Lines 22 to 32 print to standard output the \mintinline{python}{MAGIC} value and the \emph{C++} array of obtained values. This array is shown below. 

\begin{minted}[breaklines,fontsize=\small]{cpp}
found magic for K = 6: 0x2643c51ab9dfa5b
constexpr uint8_t positions[64] = {
    0,  1,  2, 14,  3, 22, 28, 15, 11,  4, 23, 55,  7, 29, 41, 16,
   12, 26, 53,  5, 24, 33, 56, 35, 61,  8, 30, 58, 37, 42, 17, 46,
   63, 13, 21, 27, 10, 54,  6, 40, 25, 52, 32, 34, 60, 57, 36, 45,
   62, 20,  9, 39, 51, 31, 59, 44, 19, 38, 50, 43, 18, 49, 48, 47
};
\end{minted}


% Notons qu'en Python il était important d'effectuer l'opération index(m, K, bitboard) & size - 1 à la ligne 28 car les nombres entiers ont une précision arbitraire et par exemple nous avons
Note that in \emph{Python} it was important to perform the operation \mintinline{python}{index(m, K, bitboard) & size - 1} on line 28 because Python integers have arbitrary precision. For example, we have:

\begin{minted}[breaklines,fontsize=\small]{python}
   0x2643c51ab9dfa5b * 0x8000000000000000 >> 58
== 0x4c878a3573bf4b60

   (0x2643c51ab9dfa5b * 0x8000000000000000 >> 58) & size - 1
== 0x4c878a3573bf4b60 & 0b111111
== 32
\end{minted}

% En C++ nous aurions
In \emph{C++} we would have:

\begin{minted}[breaklines,fontsize=\small]{cpp}
   0x2643c51ab9dfa5b * 0x8000000000000000 >> 58
== 32
\end{minted}

% Remarquons que nous n'avons pas le même problème à la ligne 18 car MAGIC est de type BitVec(64).
Note that we don't have the same problem on line 18 because \mintinline{python}{MAGIC} is of type \mintinline{python}{BitVec(64)}.\\

\end{itemize}

% Pour tester l'efficacité de notre technique de hachage parfait magique, nous allons mettre en place le micro-benchmark donné dans le listing. Nous utilisons la librairie google benchmark pour effectuer ces mesures. Le résultat de l'exécution de ce benchmark est donné dans la figure. La fonction BM_magic_positions utilisant le hachage parfait magique est bien plus rapide que la fonction BM_unordered_map_positions utilisant une table de hachage classique. Bien entendu, la fonction BM_builtin_ctz_positions utilisant une instruction dédiée est la plus efficace (mais ne sera d'aucune utilité pour la génération des coups possibles).
To test the efficiency of our magic perfect hashing technique, we set up the micro-benchmark shown in Listing \ref{lst:compare_hashmap_magic}. We use the \emph{Google Benchmark} library for these measurements \cite{googlebenchmark2025}. The benchmark execution results are shown in Listing \ref{lst:result_compare_hashmap_magic}. The \mintinline{c++}{BM_magic_positions} function using magic perfect hashing is much faster than the \mintinline{c++}{BM_unordered_map_positions} function using a standard hash table. Of course, the \mintinline{c++}{BM_builtin_ctz_positions} function using a dedicated instruction is the most efficient (but will be of no use for move generation).\\

\begin{mdframed}[skipabove=\baselineskip,hidealllines=true]
\begin{minted}[linenos, breaklines,fontsize=\small]{cpp}
#include <benchmark/benchmark.h>
#include <unordered_map>
#include <vector>

std::vector<uint64_t> generate_isolated_bits() {
    std::vector<uint64_t> samples;
    for (int i = 0; i < 64; i++) {
        samples.push_back(1ULL << i);
    }
    return samples;
}
static void BM_magic_positions(benchmark::State& state) {
    constexpr uint64_t MAGIC = 0x2643c51ab9dfa5b;
    constexpr int K = 6;
    constexpr uint8_t positions[64] = {
         0,  1,  2, 14,  3, 22, 28, 15, 11,  4, 23, 55,  7, 29, 41, 16,
        12, 26, 53,  5, 24, 33, 56, 35, 61,  8, 30, 58, 37, 42, 17, 46,
        63, 13, 21, 27, 10, 54,  6, 40, 25, 52, 32, 34, 60, 57, 36, 45,
        62, 20,  9, 39, 51, 31, 59, 44, 19, 38, 50, 43, 18, 49, 48, 47
    };
    auto samples = generate_isolated_bits();
    size_t idx = 0;
    for (auto _ : state) {
     uint64_t bitboard = samples[idx++ & 63];
     benchmark::DoNotOptimize(positions[bitboard * MAGIC >> (64 - K)]);
    }
    state.SetItemsProcessed(state.iterations());
}
BENCHMARK(BM_magic_positions);
static void BM_unordered_map_positions(benchmark::State& state) {
    std::unordered_map<uint64_t, uint8_t> map;
    for (uint8_t i = 0; i < 64; i++) {
        map[1ULL << i] = i;
    }
    auto samples = generate_isolated_bits();
    size_t idx = 0;
    for (auto _ : state) {
        uint64_t bitboard = samples[idx++ & 63];
        benchmark::DoNotOptimize(map[bitboard]);
    }
    state.SetItemsProcessed(state.iterations());
}
BENCHMARK(BM_unordered_map_positions);
static void BM_builtin_ctz_positions(benchmark::State& state) {
    auto samples = generate_isolated_bits();
    size_t idx = 0;
    for (auto _ : state) {
        uint64_t bitboard = samples[idx++ & 63];
        benchmark::DoNotOptimize(__builtin_ctzll(bitboard));
    }
    state.SetItemsProcessed(state.iterations());
}
BENCHMARK(BM_builtin_ctz_positions);
BENCHMARK_MAIN();
\end{minted}
% \captionof{listing}{Micro-benchmark pour comparer les temps d'exécution du hachage parfait magique, d'une table de hachage classique et de l'instruction machine dédiée.}
\captionof{listing}{Micro-benchmark comparing execution times of magic perfect hashing, a standard hash table, and a dedicated machine instruction.}
\label{lst:compare_hashmap_magic}
\end{mdframed} 

\begin{mdframed}[skipabove=\baselineskip,hidealllines=true]
\begin{minted}[linenos, breaklines,fontsize=\footnotesize]{text}
Run on (12 X 4400 MHz CPU s)
CPU Caches: L1 Data 32 KiB (x6), L1 Instruction 32 KiB (x6),
            L2 Unified 256 KiB (x6), L3 Unified 12288 KiB (x1)
Load Average: 0.62, 1.33, 2.92

Benchmark                      Time        CPU    Iterations  Throughput
--------------------------------------------------------------------------
BM_magic_positions          0.542 ns   0.542 ns   1000000000  1.85 G/s
BM_unordered_map_positions   30.2 ns    30.2 ns     23361655  33.1 M/s
BM_builtin_ctz_positions    0.480 ns   0.480 ns   1000000000  2.08 G/s
\end{minted}
% \captionof{listing}{Résultats de l'exécution du micro-benchmark du listing}
\captionof{listing}{Micro-benchmark execution results from Listing \ref{lst:compare_hashmap_magic}}
\label{lst:result_compare_hashmap_magic}
\end{mdframed}

% Maintenant que nous avons vu sur un exemple simple comment marche la technique du hachage parfait magique, nous allons pouvoir étudier la génération des mouvements possibles des pièces en utilisant cette technique.
Now that we have seen how the magic perfect hashing technique works through a simple example, we can study generating possible piece moves using this technique.

\subsection{Magic Bitboards for Piece Moves}

Pour chaque configuration du plateau de jeu, nous voulons pouvoir, pour une case donnée, donner la liste des coups possibles pour cette configuration de jeu. Par exemple, pour le plateau de la figure \ref{fig:move23_position} dont les coups possibles pour la pièce en {\ttfamily d3} sont représentées par des étoiles dans la table \ref{tab:move23_d3_moves}, on voudrait très efficacement pouvoir obtenir le bitboard de la table \ref{tab:bb_move23_d3_moves} indiquant les différents coups possibles.\\

\begin{figure}[htpb]
\centering
\includegraphics[width=0.45\textwidth]{yolah_game_example_move23_no_arrow.png}
% \caption{Position du plateau après le coup 23}
\caption{Board position example (after move 23 in figure \ref{fig:game_example_middle})}
\label{fig:move23_position}
\end{figure}


\begin{table}[htpb]
\centering
\caption{Possible moves for white piece at d3}
\label{tab:move23_d3_moves}
\begin{tabular}{|c|c|c|c|c|c|c|c|c|}
\hline
8 & \raisebox{-0.3ex}{$\blacksquare$} & $\bullet$ & $\cdot$ & $\cdot$ & \raisebox{-0.3ex}{$\blacksquare$} & $\cdot$ & $\cdot$ & \raisebox{-0.3ex}{$\blacksquare$} \\ \hline
7 & $\cdot$ & $\cdot$ & $\cdot$ & $\cdot$ & $\cdot$ & $\circ$ & $\cdot$ & $\bullet$ \\ \hline
6 & $\bullet$ & $\cdot$ & $\circ$ & \raisebox{-0.3ex}{$\blacksquare$} & \raisebox{-0.3ex}{$\blacksquare$} & \raisebox{-0.3ex}{$\blacksquare$} & $\cdot$ & $\cdot$ \\ \hline
5 & $\cdot$ & $\boldsymbol{\ast}$ & $\cdot$ & \raisebox{-0.3ex}{$\blacksquare$} & \raisebox{-0.3ex}{$\blacksquare$} & \raisebox{-0.3ex}{$\blacksquare$} & \raisebox{-0.3ex}{$\blacksquare$} & \raisebox{-0.3ex}{$\blacksquare$} \\ \hline
4 & $\cdot$ & $\cdot$ & $\boldsymbol{\ast}$ & \raisebox{-0.3ex}{$\blacksquare$} & \raisebox{-0.3ex}{$\blacksquare$} & \raisebox{-0.3ex}{$\blacksquare$} & \raisebox{-0.3ex}{$\blacksquare$} & $\cdot$ \\ \hline
3 & \raisebox{-0.3ex}{$\blacksquare$} & $\boldsymbol{\ast}$ & $\boldsymbol{\ast}$ & $\circ$ & $\boldsymbol{\ast}$ & \raisebox{-0.3ex}{$\blacksquare$} & $\cdot$ & $\cdot$ \\ \hline
2 & \raisebox{-0.3ex}{$\blacksquare$} & \raisebox{-0.3ex}{$\blacksquare$} & $\boldsymbol{\ast}$ & \raisebox{-0.3ex}{$\blacksquare$} & $\boldsymbol{\ast}$ & $\circ$ & $\cdot$ & $\bullet$ \\ \hline
1 & \raisebox{-0.3ex}{$\blacksquare$} & $\boldsymbol{\ast}$ & $\cdot$ & $\cdot$ & $\cdot$ & $\boldsymbol{\ast}$ & \raisebox{-0.3ex}{$\blacksquare$} & \raisebox{-0.3ex}{$\blacksquare$} \\ \hline
\multicolumn{1}{c|}{} & a & b & c & d & e & f & g & h \\
\cline{2-9}
\end{tabular}
\end{table}

\begin{table}[htpb]
\centering
\caption{Bitboard of possible moves for the piece at d3 (see table \ref{tab:move23_d3_moves})}
\label{tab:bb_move23_d3_moves}
\begin{tabular}{|c|*{8}{>{\centering\arraybackslash}p{0.5cm}|}}
\hline
8 & \begin{tabular}{@{}c@{}}{\tiny 56}\\[1pt]0\end{tabular} & \begin{tabular}{@{}c@{}}{\tiny 57}\\[1pt]0\end{tabular} & \begin{tabular}{@{}c@{}}{\tiny 58}\\[1pt]0\end{tabular} & \begin{tabular}{@{}c@{}}{\tiny 59}\\[1pt]0\end{tabular} & \begin{tabular}{@{}c@{}}{\tiny 60}\\[1pt]0\end{tabular} & \begin{tabular}{@{}c@{}}{\tiny 61}\\[1pt]0\end{tabular} & \begin{tabular}{@{}c@{}}{\tiny 62}\\[1pt]0\end{tabular} & \begin{tabular}{@{}c@{}}{\tiny 63}\\[1pt]0\end{tabular} \\[5pt] \hline
7 & \begin{tabular}{@{}c@{}}{\tiny 48}\\[1pt]0\end{tabular} & \begin{tabular}{@{}c@{}}{\tiny 49}\\[1pt]0\end{tabular} & \begin{tabular}{@{}c@{}}{\tiny 50}\\[1pt]0\end{tabular} & \begin{tabular}{@{}c@{}}{\tiny 51}\\[1pt]0\end{tabular} & \begin{tabular}{@{}c@{}}{\tiny 52}\\[1pt]0\end{tabular} & \begin{tabular}{@{}c@{}}{\tiny 53}\\[1pt]0\end{tabular} & \begin{tabular}{@{}c@{}}{\tiny 54}\\[1pt]0\end{tabular} & \begin{tabular}{@{}c@{}}{\tiny 55}\\[1pt]0\end{tabular} \\[5pt] \hline
6 & \begin{tabular}{@{}c@{}}{\tiny 40}\\[1pt]0\end{tabular} & \begin{tabular}{@{}c@{}}{\tiny 41}\\[1pt]0\end{tabular} & \begin{tabular}{@{}c@{}}{\tiny 42}\\[1pt]0\end{tabular} & \begin{tabular}{@{}c@{}}{\tiny 43}\\[1pt]0\end{tabular} & \begin{tabular}{@{}c@{}}{\tiny 44}\\[1pt]0\end{tabular} & \begin{tabular}{@{}c@{}}{\tiny 45}\\[1pt]0\end{tabular} & \begin{tabular}{@{}c@{}}{\tiny 46}\\[1pt]0\end{tabular} & \begin{tabular}{@{}c@{}}{\tiny 47}\\[1pt]0\end{tabular} \\[5pt] \hline
5 & \begin{tabular}{@{}c@{}}{\tiny 32}\\[1pt]0\end{tabular} & \begin{tabular}{@{}c@{}}{\tiny 33}\\[1pt]\textbf{1}\end{tabular} & \begin{tabular}{@{}c@{}}{\tiny 34}\\[1pt]0\end{tabular} & \begin{tabular}{@{}c@{}}{\tiny 35}\\[1pt]0\end{tabular} & \begin{tabular}{@{}c@{}}{\tiny 36}\\[1pt]0\end{tabular} & \begin{tabular}{@{}c@{}}{\tiny 37}\\[1pt]0\end{tabular} & \begin{tabular}{@{}c@{}}{\tiny 38}\\[1pt]0\end{tabular} & \begin{tabular}{@{}c@{}}{\tiny 39}\\[1pt]0\end{tabular} \\[5pt] \hline
4 & \begin{tabular}{@{}c@{}}{\tiny 24}\\[1pt]0\end{tabular} & \begin{tabular}{@{}c@{}}{\tiny 25}\\[1pt]0\end{tabular} & \begin{tabular}{@{}c@{}}{\tiny 26}\\[1pt]\textbf{1}\end{tabular} & \begin{tabular}{@{}c@{}}{\tiny 27}\\[1pt]0\end{tabular} & \begin{tabular}{@{}c@{}}{\tiny 28}\\[1pt]0\end{tabular} & \begin{tabular}{@{}c@{}}{\tiny 29}\\[1pt]0\end{tabular} & \begin{tabular}{@{}c@{}}{\tiny 30}\\[1pt]0\end{tabular} & \begin{tabular}{@{}c@{}}{\tiny 31}\\[1pt]0\end{tabular} \\[5pt] \hline
3 & \begin{tabular}{@{}c@{}}{\tiny 16}\\[1pt]0\end{tabular} & \begin{tabular}{@{}c@{}}{\tiny 17}\\[1pt]\textbf{1}\end{tabular} & \begin{tabular}{@{}c@{}}{\tiny 18}\\[1pt]\textbf{1}\end{tabular} & \cellcolor{lightgray}\begin{tabular}{@{}c@{}}{\tiny 19}\\[1pt]0\end{tabular} & \begin{tabular}{@{}c@{}}{\tiny 20}\\[1pt]\textbf{1}\end{tabular} & \begin{tabular}{@{}c@{}}{\tiny 21}\\[1pt]0\end{tabular} & \begin{tabular}{@{}c@{}}{\tiny 22}\\[1pt]0\end{tabular} & \begin{tabular}{@{}c@{}}{\tiny 23}\\[1pt]0\end{tabular} \\[5pt] \hline
2 & \begin{tabular}{@{}c@{}}{\tiny 8}\\[1pt]0\end{tabular} & \begin{tabular}{@{}c@{}}{\tiny 9}\\[1pt]0\end{tabular} & \begin{tabular}{@{}c@{}}{\tiny 10}\\[1pt]\textbf{1}\end{tabular} & \begin{tabular}{@{}c@{}}{\tiny 11}\\[1pt]0\end{tabular} & \begin{tabular}{@{}c@{}}{\tiny 12}\\[1pt]\textbf{1}\end{tabular} & \begin{tabular}{@{}c@{}}{\tiny 13}\\[1pt]0\end{tabular} & \begin{tabular}{@{}c@{}}{\tiny 14}\\[1pt]0\end{tabular} & \begin{tabular}{@{}c@{}}{\tiny 15}\\[1pt]0\end{tabular} \\[5pt] \hline
1 & \begin{tabular}{@{}c@{}}{\tiny 0}\\[1pt]0\end{tabular} & \begin{tabular}{@{}c@{}}{\tiny 1}\\[1pt]\textbf{1}\end{tabular} & \begin{tabular}{@{}c@{}}{\tiny 2}\\[1pt]0\end{tabular} & \begin{tabular}{@{}c@{}}{\tiny 3}\\[1pt]0\end{tabular} & \begin{tabular}{@{}c@{}}{\tiny 4}\\[1pt]0\end{tabular} & \begin{tabular}{@{}c@{}}{\tiny 5}\\[1pt]\textbf{1}\end{tabular} & \begin{tabular}{@{}c@{}}{\tiny 6}\\[1pt]0\end{tabular} & \begin{tabular}{@{}c@{}}{\tiny 7}\\[1pt]0\end{tabular} \\[5pt] \hline
\multicolumn{1}{c|}{} & a & b & c & d & e & f & g & h \\
\cline{2-9}
\end{tabular}
\end{table}

\begin{table}[htpb]
\centering
\caption{Masque pour les cases atteignables par la pièce en d3 sans considérer les éventuels obstacles}
\label{tab:mask_for_d3}
\begin{tabular}{|c|*{8}{>{\centering\arraybackslash}p{0.5cm}|}}
\hline
8 & \begin{tabular}{@{}c@{}}{\tiny 56}\\[1pt]0\end{tabular} & \begin{tabular}{@{}c@{}}{\tiny 57}\\[1pt]0\end{tabular} & \begin{tabular}{@{}c@{}}{\tiny 58}\\[1pt]0\end{tabular} & \begin{tabular}{@{}c@{}}{\tiny 59}\\[1pt]\textbf{1}\end{tabular} & \begin{tabular}{@{}c@{}}{\tiny 60}\\[1pt]0\end{tabular} & \begin{tabular}{@{}c@{}}{\tiny 61}\\[1pt]0\end{tabular} & \begin{tabular}{@{}c@{}}{\tiny 62}\\[1pt]0\end{tabular} & \begin{tabular}{@{}c@{}}{\tiny 63}\\[1pt]0\end{tabular} \\[5pt] \hline
7 & \begin{tabular}{@{}c@{}}{\tiny 48}\\[1pt]0\end{tabular} & \begin{tabular}{@{}c@{}}{\tiny 49}\\[1pt]0\end{tabular} & \begin{tabular}{@{}c@{}}{\tiny 50}\\[1pt]0\end{tabular} & \begin{tabular}{@{}c@{}}{\tiny 51}\\[1pt]\textbf{1}\end{tabular} & \begin{tabular}{@{}c@{}}{\tiny 52}\\[1pt]0\end{tabular} & \begin{tabular}{@{}c@{}}{\tiny 53}\\[1pt]0\end{tabular} & \begin{tabular}{@{}c@{}}{\tiny 54}\\[1pt]0\end{tabular} & \begin{tabular}{@{}c@{}}{\tiny 55}\\[1pt]\textbf{1}\end{tabular} \\[5pt] \hline
6 & \begin{tabular}{@{}c@{}}{\tiny 40}\\[1pt]\textbf{1}\end{tabular} & \begin{tabular}{@{}c@{}}{\tiny 41}\\[1pt]0\end{tabular} & \begin{tabular}{@{}c@{}}{\tiny 42}\\[1pt]0\end{tabular} & \begin{tabular}{@{}c@{}}{\tiny 43}\\[1pt]\textbf{1}\end{tabular} & \begin{tabular}{@{}c@{}}{\tiny 44}\\[1pt]0\end{tabular} & \begin{tabular}{@{}c@{}}{\tiny 45}\\[1pt]0\end{tabular} & \begin{tabular}{@{}c@{}}{\tiny 46}\\[1pt]\textbf{1}\end{tabular} & \begin{tabular}{@{}c@{}}{\tiny 47}\\[1pt]0\end{tabular} \\[5pt] \hline
5 & \begin{tabular}{@{}c@{}}{\tiny 32}\\[1pt]0\end{tabular} & \begin{tabular}{@{}c@{}}{\tiny 33}\\[1pt]\textbf{1}\end{tabular} & \begin{tabular}{@{}c@{}}{\tiny 34}\\[1pt]0\end{tabular} & \begin{tabular}{@{}c@{}}{\tiny 35}\\[1pt]\textbf{1}\end{tabular} & \begin{tabular}{@{}c@{}}{\tiny 36}\\[1pt]0\end{tabular} & \begin{tabular}{@{}c@{}}{\tiny 37}\\[1pt]\textbf{1}\end{tabular} & \begin{tabular}{@{}c@{}}{\tiny 38}\\[1pt]0\end{tabular} & \begin{tabular}{@{}c@{}}{\tiny 39}\\[1pt]0\end{tabular} \\[5pt] \hline
4 & \begin{tabular}{@{}c@{}}{\tiny 24}\\[1pt]0\end{tabular} & \begin{tabular}{@{}c@{}}{\tiny 25}\\[1pt]0\end{tabular} & \begin{tabular}{@{}c@{}}{\tiny 26}\\[1pt]\textbf{1}\end{tabular} & \begin{tabular}{@{}c@{}}{\tiny 27}\\[1pt]\textbf{1}\end{tabular} & \begin{tabular}{@{}c@{}}{\tiny 28}\\[1pt]\textbf{1}\end{tabular} & \begin{tabular}{@{}c@{}}{\tiny 29}\\[1pt]0\end{tabular} & \begin{tabular}{@{}c@{}}{\tiny 30}\\[1pt]0\end{tabular} & \begin{tabular}{@{}c@{}}{\tiny 31}\\[1pt]0\end{tabular} \\[5pt] \hline
3 & \begin{tabular}{@{}c@{}}{\tiny 16}\\[1pt]\textbf{1}\end{tabular} & \begin{tabular}{@{}c@{}}{\tiny 17}\\[1pt]\textbf{1}\end{tabular} & \begin{tabular}{@{}c@{}}{\tiny 18}\\[1pt]\textbf{1}\end{tabular} & \cellcolor{lightgray}\begin{tabular}{@{}c@{}}{\tiny 19}\\[1pt]\textbf{0}\end{tabular} & \begin{tabular}{@{}c@{}}{\tiny 20}\\[1pt]\textbf{1}\end{tabular} & \begin{tabular}{@{}c@{}}{\tiny 21}\\[1pt]\textbf{1}\end{tabular} & \begin{tabular}{@{}c@{}}{\tiny 22}\\[1pt]\textbf{1}\end{tabular} & \begin{tabular}{@{}c@{}}{\tiny 23}\\[1pt]\textbf{1}\end{tabular} \\[5pt] \hline
2 & \begin{tabular}{@{}c@{}}{\tiny 8}\\[1pt]0\end{tabular} & \begin{tabular}{@{}c@{}}{\tiny 9}\\[1pt]0\end{tabular} & \begin{tabular}{@{}c@{}}{\tiny 10}\\[1pt]\textbf{1}\end{tabular} & \begin{tabular}{@{}c@{}}{\tiny 11}\\[1pt]\textbf{1}\end{tabular} & \begin{tabular}{@{}c@{}}{\tiny 12}\\[1pt]\textbf{1}\end{tabular} & \begin{tabular}{@{}c@{}}{\tiny 13}\\[1pt]0\end{tabular} & \begin{tabular}{@{}c@{}}{\tiny 14}\\[1pt]0\end{tabular} & \begin{tabular}{@{}c@{}}{\tiny 15}\\[1pt]0\end{tabular} \\[5pt] \hline
1 & \begin{tabular}{@{}c@{}}{\tiny 0}\\[1pt]0\end{tabular} & \begin{tabular}{@{}c@{}}{\tiny 1}\\[1pt]\textbf{1}\end{tabular} & \begin{tabular}{@{}c@{}}{\tiny 2}\\[1pt]0\end{tabular} & \begin{tabular}{@{}c@{}}{\tiny 3}\\[1pt]\textbf{1}\end{tabular} & \begin{tabular}{@{}c@{}}{\tiny 4}\\[1pt]0\end{tabular} & \begin{tabular}{@{}c@{}}{\tiny 5}\\[1pt]\textbf{1}\end{tabular} & \begin{tabular}{@{}c@{}}{\tiny 6}\\[1pt]0\end{tabular} & \begin{tabular}{@{}c@{}}{\tiny 7}\\[1pt]0\end{tabular} \\[5pt] \hline
\multicolumn{1}{c|}{} & a & b & c & d & e & f & g & h \\
\cline{2-9}
\end{tabular}
\end{table}

\begin{table}[htpb]
\centering
\caption{Board bitboard}
\label{tab:board_move23}
\begin{tabular}{|c|*{8}{>{\centering\arraybackslash}p{0.5cm}|}}
\hline
8 & \begin{tabular}{@{}c@{}}{\tiny 56}\\[1pt]\textbf{1}\end{tabular} & \begin{tabular}{@{}c@{}}{\tiny 57}\\[1pt]\textbf{1}\end{tabular} & \begin{tabular}{@{}c@{}}{\tiny 58}\\[1pt]0\end{tabular} & \begin{tabular}{@{}c@{}}{\tiny 59}\\[1pt]0\end{tabular} & \begin{tabular}{@{}c@{}}{\tiny 60}\\[1pt]\textbf{1}\end{tabular} & \begin{tabular}{@{}c@{}}{\tiny 61}\\[1pt]0\end{tabular} & \begin{tabular}{@{}c@{}}{\tiny 62}\\[1pt]0\end{tabular} & \begin{tabular}{@{}c@{}}{\tiny 63}\\[1pt]\textbf{1}\end{tabular} \\[5pt] \hline
7 & \begin{tabular}{@{}c@{}}{\tiny 48}\\[1pt]0\end{tabular} & \begin{tabular}{@{}c@{}}{\tiny 49}\\[1pt]0\end{tabular} & \begin{tabular}{@{}c@{}}{\tiny 50}\\[1pt]0\end{tabular} & \begin{tabular}{@{}c@{}}{\tiny 51}\\[1pt]0\end{tabular} & \begin{tabular}{@{}c@{}}{\tiny 52}\\[1pt]0\end{tabular} & \begin{tabular}{@{}c@{}}{\tiny 53}\\[1pt]\textbf{1}\end{tabular} & \begin{tabular}{@{}c@{}}{\tiny 54}\\[1pt]0\end{tabular} & \begin{tabular}{@{}c@{}}{\tiny 55}\\[1pt]\textbf{1}\end{tabular} \\[5pt] \hline
6 & \begin{tabular}{@{}c@{}}{\tiny 40}\\[1pt]\textbf{1}\end{tabular} & \begin{tabular}{@{}c@{}}{\tiny 41}\\[1pt]0\end{tabular} & \begin{tabular}{@{}c@{}}{\tiny 42}\\[1pt]\textbf{1}\end{tabular} & \begin{tabular}{@{}c@{}}{\tiny 43}\\[1pt]\textbf{1}\end{tabular} & \begin{tabular}{@{}c@{}}{\tiny 44}\\[1pt]\textbf{1}\end{tabular} & \begin{tabular}{@{}c@{}}{\tiny 45}\\[1pt]\textbf{1}\end{tabular} & \begin{tabular}{@{}c@{}}{\tiny 46}\\[1pt]0\end{tabular} & \begin{tabular}{@{}c@{}}{\tiny 47}\\[1pt]0\end{tabular} \\[5pt] \hline
5 & \begin{tabular}{@{}c@{}}{\tiny 32}\\[1pt]0\end{tabular} & \begin{tabular}{@{}c@{}}{\tiny 33}\\[1pt]0\end{tabular} & \begin{tabular}{@{}c@{}}{\tiny 34}\\[1pt]0\end{tabular} & \begin{tabular}{@{}c@{}}{\tiny 35}\\[1pt]\textbf{1}\end{tabular} & \begin{tabular}{@{}c@{}}{\tiny 36}\\[1pt]\textbf{1}\end{tabular} & \begin{tabular}{@{}c@{}}{\tiny 37}\\[1pt]\textbf{1}\end{tabular} & \begin{tabular}{@{}c@{}}{\tiny 38}\\[1pt]\textbf{1}\end{tabular} & \begin{tabular}{@{}c@{}}{\tiny 39}\\[1pt]\textbf{1}\end{tabular} \\[5pt] \hline
4 & \begin{tabular}{@{}c@{}}{\tiny 24}\\[1pt]0\end{tabular} & \begin{tabular}{@{}c@{}}{\tiny 25}\\[1pt]0\end{tabular} & \begin{tabular}{@{}c@{}}{\tiny 26}\\[1pt]0\end{tabular} & \begin{tabular}{@{}c@{}}{\tiny 27}\\[1pt]\textbf{1}\end{tabular} & \begin{tabular}{@{}c@{}}{\tiny 28}\\[1pt]\textbf{1}\end{tabular} & \begin{tabular}{@{}c@{}}{\tiny 29}\\[1pt]\textbf{1}\end{tabular} & \begin{tabular}{@{}c@{}}{\tiny 30}\\[1pt]\textbf{1}\end{tabular} & \begin{tabular}{@{}c@{}}{\tiny 31}\\[1pt]0\end{tabular} \\[5pt] \hline
3 & \begin{tabular}{@{}c@{}}{\tiny 16}\\[1pt]\textbf{1}\end{tabular} & \begin{tabular}{@{}c@{}}{\tiny 17}\\[1pt]0\end{tabular} & \begin{tabular}{@{}c@{}}{\tiny 18}\\[1pt]0\end{tabular} & \begin{tabular}{@{}c@{}}{\tiny 19}\\[1pt]\textbf{1}\end{tabular} & \begin{tabular}{@{}c@{}}{\tiny 20}\\[1pt]0\end{tabular} & \begin{tabular}{@{}c@{}}{\tiny 21}\\[1pt]\textbf{1}\end{tabular} & \begin{tabular}{@{}c@{}}{\tiny 22}\\[1pt]0\end{tabular} & \begin{tabular}{@{}c@{}}{\tiny 23}\\[1pt]0\end{tabular} \\[5pt] \hline
2 & \begin{tabular}{@{}c@{}}{\tiny 8}\\[1pt]\textbf{1}\end{tabular} & \begin{tabular}{@{}c@{}}{\tiny 9}\\[1pt]\textbf{1}\end{tabular} & \begin{tabular}{@{}c@{}}{\tiny 10}\\[1pt]0\end{tabular} & \begin{tabular}{@{}c@{}}{\tiny 11}\\[1pt]\textbf{1}\end{tabular} & \begin{tabular}{@{}c@{}}{\tiny 12}\\[1pt]0\end{tabular} & \begin{tabular}{@{}c@{}}{\tiny 13}\\[1pt]\textbf{1}\end{tabular} & \begin{tabular}{@{}c@{}}{\tiny 14}\\[1pt]0\end{tabular} & \begin{tabular}{@{}c@{}}{\tiny 15}\\[1pt]\textbf{1}\end{tabular} \\[5pt] \hline
1 & \begin{tabular}{@{}c@{}}{\tiny 0}\\[1pt]\textbf{1}\end{tabular} & \begin{tabular}{@{}c@{}}{\tiny 1}\\[1pt]0\end{tabular} & \begin{tabular}{@{}c@{}}{\tiny 2}\\[1pt]0\end{tabular} & \begin{tabular}{@{}c@{}}{\tiny 3}\\[1pt]0\end{tabular} & \begin{tabular}{@{}c@{}}{\tiny 4}\\[1pt]0\end{tabular} & \begin{tabular}{@{}c@{}}{\tiny 5}\\[1pt]0\end{tabular} & \begin{tabular}{@{}c@{}}{\tiny 6}\\[1pt]\textbf{1}\end{tabular} & \begin{tabular}{@{}c@{}}{\tiny 7}\\[1pt]\textbf{1}\end{tabular} \\[5pt] \hline
\multicolumn{1}{c|}{} & a & b & c & d & e & f & g & h \\
\cline{2-9}
\end{tabular}
\end{table}

\begin{table}[htpb]
\centering
\caption{Occupancies bitboard (\mintinline{c++}{mask & board}) for d3}
\label{tab:occupancies_d3_move23}
\begin{tabular}{|c|*{8}{>{\centering\arraybackslash}p{0.5cm}|}}
\hline
8 & \begin{tabular}{@{}c@{}}{\tiny 56}\\[1pt]0\end{tabular} & \begin{tabular}{@{}c@{}}{\tiny 57}\\[1pt]0\end{tabular} & \begin{tabular}{@{}c@{}}{\tiny 58}\\[1pt]0\end{tabular} & \begin{tabular}{@{}c@{}}{\tiny 59}\\[1pt]0\end{tabular} & \begin{tabular}{@{}c@{}}{\tiny 60}\\[1pt]0\end{tabular} & \begin{tabular}{@{}c@{}}{\tiny 61}\\[1pt]0\end{tabular} & \begin{tabular}{@{}c@{}}{\tiny 62}\\[1pt]0\end{tabular} & \begin{tabular}{@{}c@{}}{\tiny 63}\\[1pt]0\end{tabular} \\[5pt] \hline
7 & \begin{tabular}{@{}c@{}}{\tiny 48}\\[1pt]0\end{tabular} & \begin{tabular}{@{}c@{}}{\tiny 49}\\[1pt]0\end{tabular} & \begin{tabular}{@{}c@{}}{\tiny 50}\\[1pt]0\end{tabular} & \begin{tabular}{@{}c@{}}{\tiny 51}\\[1pt]0\end{tabular} & \begin{tabular}{@{}c@{}}{\tiny 52}\\[1pt]0\end{tabular} & \begin{tabular}{@{}c@{}}{\tiny 53}\\[1pt]0\end{tabular} & \begin{tabular}{@{}c@{}}{\tiny 54}\\[1pt]0\end{tabular} & \begin{tabular}{@{}c@{}}{\tiny 55}\\[1pt]\textbf{1}\end{tabular} \\[5pt] \hline
6 & \begin{tabular}{@{}c@{}}{\tiny 40}\\[1pt]\textbf{1}\end{tabular} & \begin{tabular}{@{}c@{}}{\tiny 41}\\[1pt]0\end{tabular} & \begin{tabular}{@{}c@{}}{\tiny 42}\\[1pt]0\end{tabular} & \begin{tabular}{@{}c@{}}{\tiny 43}\\[1pt]\textbf{1}\end{tabular} & \begin{tabular}{@{}c@{}}{\tiny 44}\\[1pt]0\end{tabular} & \begin{tabular}{@{}c@{}}{\tiny 45}\\[1pt]0\end{tabular} & \begin{tabular}{@{}c@{}}{\tiny 46}\\[1pt]0\end{tabular} & \begin{tabular}{@{}c@{}}{\tiny 47}\\[1pt]0\end{tabular} \\[5pt] \hline
5 & \begin{tabular}{@{}c@{}}{\tiny 32}\\[1pt]0\end{tabular} & \begin{tabular}{@{}c@{}}{\tiny 33}\\[1pt]0\end{tabular} & \begin{tabular}{@{}c@{}}{\tiny 34}\\[1pt]0\end{tabular} & \begin{tabular}{@{}c@{}}{\tiny 35}\\[1pt]\textbf{1}\end{tabular} & \begin{tabular}{@{}c@{}}{\tiny 36}\\[1pt]0\end{tabular} & \begin{tabular}{@{}c@{}}{\tiny 37}\\[1pt]\textbf{1}\end{tabular} & \begin{tabular}{@{}c@{}}{\tiny 38}\\[1pt]0\end{tabular} & \begin{tabular}{@{}c@{}}{\tiny 39}\\[1pt]0\end{tabular} \\[5pt] \hline
4 & \begin{tabular}{@{}c@{}}{\tiny 24}\\[1pt]0\end{tabular} & \begin{tabular}{@{}c@{}}{\tiny 25}\\[1pt]0\end{tabular} & \begin{tabular}{@{}c@{}}{\tiny 26}\\[1pt]0\end{tabular} & \begin{tabular}{@{}c@{}}{\tiny 27}\\[1pt]\textbf{1}\end{tabular} & \begin{tabular}{@{}c@{}}{\tiny 28}\\[1pt]\textbf{1}\end{tabular} & \begin{tabular}{@{}c@{}}{\tiny 29}\\[1pt]0\end{tabular} & \begin{tabular}{@{}c@{}}{\tiny 30}\\[1pt]0\end{tabular} & \begin{tabular}{@{}c@{}}{\tiny 31}\\[1pt]0\end{tabular} \\[5pt] \hline
3 & \begin{tabular}{@{}c@{}}{\tiny 16}\\[1pt]\textbf{1}\end{tabular} & \begin{tabular}{@{}c@{}}{\tiny 17}\\[1pt]0\end{tabular} & \begin{tabular}{@{}c@{}}{\tiny 18}\\[1pt]0\end{tabular} & \cellcolor{lightgray}\begin{tabular}{@{}c@{}}{\tiny 19}\\[1pt]\textbf{0}\end{tabular} & \begin{tabular}{@{}c@{}}{\tiny 20}\\[1pt]0\end{tabular} & \begin{tabular}{@{}c@{}}{\tiny 21}\\[1pt]\textbf{1}\end{tabular} & \begin{tabular}{@{}c@{}}{\tiny 22}\\[1pt]0\end{tabular} & \begin{tabular}{@{}c@{}}{\tiny 23}\\[1pt]0\end{tabular} \\[5pt] \hline
2 & \begin{tabular}{@{}c@{}}{\tiny 8}\\[1pt]0\end{tabular} & \begin{tabular}{@{}c@{}}{\tiny 9}\\[1pt]0\end{tabular} & \begin{tabular}{@{}c@{}}{\tiny 10}\\[1pt]0\end{tabular} & \begin{tabular}{@{}c@{}}{\tiny 11}\\[1pt]\textbf{1}\end{tabular} & \begin{tabular}{@{}c@{}}{\tiny 12}\\[1pt]0\end{tabular} & \begin{tabular}{@{}c@{}}{\tiny 13}\\[1pt]0\end{tabular} & \begin{tabular}{@{}c@{}}{\tiny 14}\\[1pt]0\end{tabular} & \begin{tabular}{@{}c@{}}{\tiny 15}\\[1pt]0\end{tabular} \\[5pt] \hline
1 & \begin{tabular}{@{}c@{}}{\tiny 0}\\[1pt]0\end{tabular} & \begin{tabular}{@{}c@{}}{\tiny 1}\\[1pt]0\end{tabular} & \begin{tabular}{@{}c@{}}{\tiny 2}\\[1pt]0\end{tabular} & \begin{tabular}{@{}c@{}}{\tiny 3}\\[1pt]0\end{tabular} & \begin{tabular}{@{}c@{}}{\tiny 4}\\[1pt]0\end{tabular} & \begin{tabular}{@{}c@{}}{\tiny 5}\\[1pt]0\end{tabular} & \begin{tabular}{@{}c@{}}{\tiny 6}\\[1pt]0\end{tabular} & \begin{tabular}{@{}c@{}}{\tiny 7}\\[1pt]0\end{tabular} \\[5pt] \hline
\multicolumn{1}{c|}{} & a & b & c & d & e & f & g & h \\
\cline{2-9}
\end{tabular}
\end{table}

Prenons comme exemple la case {\ttfamily d3}. Nous avons tout d'abord besoin d'un masque qui va nous permettre d'isoler les cases du plateau atteignables par la pièce en {\ttfamily d3} sans prendre en considération les obstacles. Ce masque est représenté dans la table \ref{tab:mask_for_d3}. Soit \mintinline{c++}{mask} le masque de la table \ref{tab:mask_for_d3} et \mintinline{c++}{board} le bitboard de la table \ref{tab:board_move23} représentant le plateau de jeu de la figure \ref{fig:move23_position}. On obtient le bitboard \mintinline{c++}{occupancies}, donnant les obstacles sur les trajectoires possibles de la pièce en {\ttfamily d3} en faisant~: \mintinline{c++}{occupancies = mask & board}. Ce bitboard est donné dans la table \ref{tab:occupancies_d3_move23}. Il y a un $1$ pour chaque case contenant une pièce ou un trou. Ce qu'il nous faut, c'est de pouvoir indexer une table, appelons la \mintinline{c++}{uint64_t MOVES_D3[]}, grâce au bitboard \mintinline{c++}{occupancies}, pour obtenir le bitboard des coups possibles à partir de la case {\ttfamily d3}. Ce bitboard est représenté dans la table \ref{tab:bb_move23_d3_moves}. Il ne nous restera plus qu'à parcourir chacun des bits à $1$ dans ce bitboard pour connaitre les cases où l'on peut se rendre et créer le coup correspondant. Pour obtenir l'index dans \mintinline{c++}{MOVES_D3}, on va utiliser la technique des magic bitboard que nous avons décrite dans la section précédente. Pour la case {\ttfamily d3}, il va nous falloir trouver une valeur \mintinline{c++}{K} et une valeur \mintinline{c++}{MAGIC} (voir listing \ref{lst:magic_perfect_hashing}) pour pouvoir obtenir le bitboard des coups possibles en faisant~: \mintinline{c++}{MOVES_D3[occupancies * MAGIC >> (64 - K)]}.\\

Pour trouver une constante \mintinline{c++}{MAGIC} pour la case {\ttfamily d3}, nous allons avoir besoin d'énumérer tous les placements d'obstacles possibles sur les cases où peut se rendre la pièce en {\ttfamily d3}. Cela veut dire que l'on doit énumérer tous les sous-ensembles de $1$ du bitboard de la table \ref{tab:mask_for_d3}. Huit sous-ensembles de cette énumération sont présentés dans la figure \ref{fig:subsets_for_d3}. Il y a $25$ cases sur la trajectoire de la pièce en {\ttfamily d3} (voir la table \ref{tab:mask_for_d3}), cela veut dire qu'il y a $2^{25} = 33,554,432$ sous-ensembles à énumérer~!
Nous voudrions réduire ce nombre de configurations des obstacles possibles sur la trajectoire de la pièce. En effet, comme nous le verrons lorsque nous détaillerons le code pour déterminer la valeur de la constante \mintinline{c++}{MAGIC}, la place mémoire pour stocker les coups possibles pour une case donnée va dépendre de ce nombre de configurations.\\

\begin{figure}[htpb]
\centering
\begin{subfigure}[b]{0.45\textwidth}
\centering
\begin{tabular}{|c|*{8}{>{\centering\arraybackslash}p{0.3cm}|}}
\hline
{\scriptsize 8} & \tiny 0 & \tiny 0 & \tiny 0 & \tiny 0 & \tiny 0 & \tiny 0 & \tiny 0 & \tiny 0 \\[2pt] \hline
{\scriptsize 7} & \tiny 0 & \tiny 0 & \tiny 0 & \tiny 0 & \tiny 0 & \tiny 0 & \tiny 0 & \tiny 0 \\[2pt] \hline
{\scriptsize 6} & \tiny 0 & \tiny 0 & \tiny 0 & \tiny 0 & \tiny 0 & \tiny 0 & \tiny 0 & \tiny 0 \\[2pt] \hline
{\scriptsize 5} & \tiny 0 & \tiny 0 & \tiny 0 & \tiny 0 & \tiny 0 & \tiny 0 & \tiny 0 & \tiny 0 \\[2pt] \hline
{\scriptsize 4} & \tiny 0 & \tiny 0 & \tiny 0 & \tiny 0 & \tiny 0 & \tiny 0 & \tiny 0 & \tiny 0 \\[2pt] \hline
{\scriptsize 3} & \tiny 0 & \tiny 0 & \tiny 0 & \cellcolor{lightgray}\tiny 0 & \tiny 0 & \tiny 0 & \tiny 0 & \tiny 0 \\[2pt] \hline
{\scriptsize 2} & \tiny 0 & \tiny 0 & \tiny 0 & \tiny 0 & \tiny 0 & \tiny 0 & \tiny 0 & \tiny 0 \\[2pt] \hline
{\scriptsize 1} & \tiny 0 & \tiny 0 & \tiny 0 & \tiny 0 & \tiny 0 & \tiny 0 & \tiny 0 & \tiny 0 \\[2pt] \hline
\multicolumn{1}{c|}{} & \scriptsize a & \scriptsize b & \scriptsize c & \scriptsize d & \scriptsize e & \scriptsize f & \scriptsize g & \scriptsize h \\
\cline{2-9}
\end{tabular}
\caption{Empty (0 bits)}
\end{subfigure}
\hfill
\begin{subfigure}[b]{0.45\textwidth}
\centering
\begin{tabular}{|c|*{8}{>{\centering\arraybackslash}p{0.3cm}|}}
\hline
{\scriptsize 8} & \tiny 0 & \tiny 0 & \tiny 0 & \tiny 0 & \tiny 0 & \tiny 0 & \tiny 0 & \tiny 0 \\[2pt] \hline
{\scriptsize 7} & \tiny 0 & \tiny 0 & \tiny 0 & \tiny 0 & \tiny 0 & \tiny 0 & \tiny 0 & \tiny 0 \\[2pt] \hline
{\scriptsize 6} & \tiny 0 & \tiny 0 & \tiny 0 & \tiny 0 & \tiny 0 & \tiny 0 & \tiny 0 & \tiny 0 \\[2pt] \hline
{\scriptsize 5} & \tiny 0 & \tiny 0 & \tiny 0 & \tiny 0 & \tiny 0 & \tiny 0 & \tiny 0 & \tiny 0 \\[2pt] \hline
{\scriptsize 4} & \tiny 0 & \tiny 0 & \tiny 0 & \tiny\textbf{1} & \tiny 0 & \tiny 0 & \tiny 0 & \tiny 0 \\[2pt] \hline
{\scriptsize 3} & \tiny 0 & \tiny 0 & \tiny 0 & \cellcolor{lightgray}\tiny 0 & \tiny 0 & \tiny 0 & \tiny 0 & \tiny 0 \\[2pt] \hline
{\scriptsize 2} & \tiny 0 & \tiny 0 & \tiny 0 & \tiny 0 & \tiny 0 & \tiny 0 & \tiny 0 & \tiny 0 \\[2pt] \hline
{\scriptsize 1} & \tiny 0 & \tiny 0 & \tiny 0 & \tiny 0 & \tiny 0 & \tiny 0 & \tiny 0 & \tiny 0 \\[2pt] \hline
\multicolumn{1}{c|}{} & \scriptsize a & \scriptsize b & \scriptsize c & \scriptsize d & \scriptsize e & \scriptsize f & \scriptsize g & \scriptsize h \\
\cline{2-9}
\end{tabular}
\caption{1 bit set}
\end{subfigure}

\vspace{0.8cm}

\begin{subfigure}[b]{0.45\textwidth}
\centering
\begin{tabular}{|c|*{8}{>{\centering\arraybackslash}p{0.3cm}|}}
\hline
{\scriptsize 8} & \tiny 0 & \tiny 0 & \tiny 0 & \tiny 0 & \tiny 0 & \tiny 0 & \tiny 0 & \tiny 0 \\[2pt] \hline
{\scriptsize 7} & \tiny 0 & \tiny 0 & \tiny 0 & \tiny 0 & \tiny 0 & \tiny 0 & \tiny 0 & \tiny 0 \\[2pt] \hline
{\scriptsize 6} & \tiny 0 & \tiny 0 & \tiny 0 & \tiny 0 & \tiny 0 & \tiny 0 & \tiny 0 & \tiny 0 \\[2pt] \hline
{\scriptsize 5} & \tiny 0 & \tiny\textbf{1} & \tiny 0 & \tiny 0 & \tiny 0 & \tiny 0 & \tiny 0 & \tiny 0 \\[2pt] \hline
{\scriptsize 4} & \tiny 0 & \tiny 0 & \tiny 0 & \tiny 0 & \tiny 0 & \tiny 0 & \tiny 0 & \tiny 0 \\[2pt] \hline
{\scriptsize 3} & \tiny\textbf{1} & \tiny 0 & \tiny 0 & \cellcolor{lightgray}\tiny 0 & \tiny 0 & \tiny 0 & \tiny 0 & \tiny 0 \\[2pt] \hline
{\scriptsize 2} & \tiny 0 & \tiny 0 & \tiny 0 & \tiny 0 & \tiny 0 & \tiny 0 & \tiny 0 & \tiny 0 \\[2pt] \hline
{\scriptsize 1} & \tiny 0 & \tiny 0 & \tiny 0 & \tiny 0 & \tiny 0 & \tiny 0 & \tiny 0 & \tiny 0 \\[2pt] \hline
\multicolumn{1}{c|}{} & \scriptsize a & \scriptsize b & \scriptsize c & \scriptsize d & \scriptsize e & \scriptsize f & \scriptsize g & \scriptsize h \\
\cline{2-9}
\end{tabular}
\caption{2 bits set}
\end{subfigure}
\hfill
\begin{subfigure}[b]{0.45\textwidth}
\centering
\begin{tabular}{|c|*{8}{>{\centering\arraybackslash}p{0.3cm}|}}
\hline
{\scriptsize 8} & \tiny 0 & \tiny 0 & \tiny 0 & \tiny\textbf{1} & \tiny 0 & \tiny 0 & \tiny 0 & \tiny 0 \\[2pt] \hline
{\scriptsize 7} & \tiny 0 & \tiny 0 & \tiny 0 & \tiny 0 & \tiny 0 & \tiny 0 & \tiny 0 & \tiny 0 \\[2pt] \hline
{\scriptsize 6} & \tiny 0 & \tiny 0 & \tiny 0 & \tiny 0 & \tiny 0 & \tiny 0 & \tiny 0 & \tiny 0 \\[2pt] \hline
{\scriptsize 5} & \tiny 0 & \tiny 0 & \tiny 0 & \tiny 0 & \tiny 0 & \tiny 0 & \tiny 0 & \tiny 0 \\[2pt] \hline
{\scriptsize 4} & \tiny 0 & \tiny 0 & \tiny\textbf{1} & \tiny 0 & \tiny 0 & \tiny 0 & \tiny 0 & \tiny 0 \\[2pt] \hline
{\scriptsize 3} & \tiny 0 & \tiny 0 & \tiny 0 & \cellcolor{lightgray}\tiny 0 & \tiny 0 & \tiny\textbf{1} & \tiny 0 & \tiny 0 \\[2pt] \hline
{\scriptsize 2} & \tiny 0 & \tiny 0 & \tiny 0 & \tiny 0 & \tiny 0 & \tiny 0 & \tiny 0 & \tiny 0 \\[2pt] \hline
{\scriptsize 1} & \tiny 0 & \tiny 0 & \tiny 0 & \tiny 0 & \tiny 0 & \tiny 0 & \tiny 0 & \tiny 0 \\[2pt] \hline
\multicolumn{1}{c|}{} & \scriptsize a & \scriptsize b & \scriptsize c & \scriptsize d & \scriptsize e & \scriptsize f & \scriptsize g & \scriptsize h \\
\cline{2-9}
\end{tabular}
\caption{3 bits set}
\end{subfigure}

\vspace{0.8cm}

\begin{subfigure}[b]{0.45\textwidth}
\centering
\begin{tabular}{|c|*{8}{>{\centering\arraybackslash}p{0.3cm}|}}
\hline
{\scriptsize 8} & \tiny 0 & \tiny 0 & \tiny 0 & \tiny 0 & \tiny 0 & \tiny 0 & \tiny 0 & \tiny 0 \\[2pt] \hline
{\scriptsize 7} & \tiny 0 & \tiny 0 & \tiny 0 & \tiny\textbf{1} & \tiny 0 & \tiny 0 & \tiny 0 & \tiny\textbf{1} \\[2pt] \hline
{\scriptsize 6} & \tiny\textbf{1} & \tiny 0 & \tiny 0 & \tiny 0 & \tiny 0 & \tiny 0 & \tiny\textbf{1} & \tiny 0 \\[2pt] \hline
{\scriptsize 5} & \tiny 0 & \tiny\textbf{1} & \tiny 0 & \tiny 0 & \tiny 0 & \tiny 0 & \tiny 0 & \tiny 0 \\[2pt] \hline
{\scriptsize 4} & \tiny 0 & \tiny 0 & \tiny 0 & \tiny\textbf{1} & \tiny\textbf{1} & \tiny 0 & \tiny 0 & \tiny 0 \\[2pt] \hline
{\scriptsize 3} & \tiny 0 & \tiny\textbf{1} & \tiny 0 & \cellcolor{lightgray}\tiny 0 & \tiny 0 & \tiny 0 & \tiny 0 & \tiny\textbf{1} \\[2pt] \hline
{\scriptsize 2} & \tiny 0 & \tiny 0 & \tiny 0 & \tiny 0 & \tiny 0 & \tiny 0 & \tiny 0 & \tiny 0 \\[2pt] \hline
{\scriptsize 1} & \tiny 0 & \tiny\textbf{1} & \tiny 0 & \tiny 0 & \tiny 0 & \tiny 0 & \tiny 0 & \tiny 0 \\[2pt] \hline
\multicolumn{1}{c|}{} & \scriptsize a & \scriptsize b & \scriptsize c & \scriptsize d & \scriptsize e & \scriptsize f & \scriptsize g & \scriptsize h \\
\cline{2-9}
\end{tabular}
\caption{9 bits set}
\end{subfigure}
\hfill
\begin{subfigure}[b]{0.45\textwidth}
\centering
\begin{tabular}{|c|*{8}{>{\centering\arraybackslash}p{0.3cm}|}}
\hline
{\scriptsize 8} & \tiny 0 & \tiny 0 & \tiny 0 & \tiny\textbf{1} & \tiny 0 & \tiny 0 & \tiny 0 & \tiny 0 \\[2pt] \hline
{\scriptsize 7} & \tiny 0 & \tiny 0 & \tiny 0 & \tiny\textbf{1} & \tiny 0 & \tiny 0 & \tiny 0 & \tiny\textbf{1} \\[2pt] \hline
{\scriptsize 6} & \tiny\textbf{1} & \tiny 0 & \tiny 0 & \tiny\textbf{1} & \tiny 0 & \tiny 0 & \tiny\textbf{1} & \tiny 0 \\[2pt] \hline
{\scriptsize 5} & \tiny 0 & \tiny\textbf{1} & \tiny 0 & \tiny\textbf{1} & \tiny 0 & \tiny\textbf{1} & \tiny 0 & \tiny 0 \\[2pt] \hline
{\scriptsize 4} & \tiny 0 & \tiny 0 & \tiny\textbf{1} & \tiny\textbf{1} & \tiny\textbf{1} & \tiny 0 & \tiny 0 & \tiny 0 \\[2pt] \hline
{\scriptsize 3} & \tiny\textbf{1} & \tiny\textbf{1} & \tiny\textbf{1} & \cellcolor{lightgray}\tiny 0 & \tiny\textbf{1} & \tiny\textbf{1} & \tiny\textbf{1} & \tiny\textbf{1} \\[2pt] \hline
{\scriptsize 2} & \tiny 0 & \tiny 0 & \tiny\textbf{1} & \tiny\textbf{1} & \tiny\textbf{1} & \tiny 0 & \tiny 0 & \tiny 0 \\[2pt] \hline
{\scriptsize 1} & \tiny 0 & \tiny\textbf{1} & \tiny 0 & \tiny\textbf{1} & \tiny 0 & \tiny\textbf{1} & \tiny 0 & \tiny 0 \\[2pt] \hline
\multicolumn{1}{c|}{} & \scriptsize a & \scriptsize b & \scriptsize c & \scriptsize d & \scriptsize e & \scriptsize f & \scriptsize g & \scriptsize h \\
\cline{2-9}
\end{tabular}
\caption{18 bits set (full)}
\end{subfigure}

\caption{Six subsets of the mask for d3 showing different occupancy patterns (0 to 25 bits set) from the $2^{25}$ possible ones}
\label{fig:subsets_for_d3}
\end{figure}
 
\label{par:edge_exclusion}
Pour réduire ce nombre de configurations, nous n'allons pas considérer les cases sur les bords du plateau pour enlever les bits à $1$ sur les lignes 1 et 8 et les colonnes a et h du masque de la table \ref{tab:mask_for_d3}. On obtient le bitboard de la table \ref{tab:mask_for_d3_without_edge}. Il nous manquera l'information sur les bords, mais on supposera que s'il est possible de jouer en {\ttfamily d7} par exemple, il sera aussi possible de jouer en {\ttfamily d8}. Bien sûr il se peut qu'il y ait un obstacle en {\ttfamily d8}, mais on pourra facilement éliminer ce coup impossible comme nous le verrons dans la suite de ce chapitre (partie \ref{subsection:making_and_unmaking_moves}). Avec cette réduction, il nous reste $2^{17} = 131072$ sous-ensembles à énumérer. C'est bien mieux, mais nous allons pouvoir encore réduire ce nombre.\\

\begin{table}[htpb]
\centering
\caption{Masque pour les cases atteignables pour la pièce en {\ttfamily d3} en excluant les bords du plateau}
\label{tab:mask_for_d3_without_edge}
\begin{tabular}{|c|*{8}{>{\centering\arraybackslash}p{0.5cm}|}}
\hline
8 & \begin{tabular}{@{}c@{}}{\tiny 56}\\[1pt]0\end{tabular} & \begin{tabular}{@{}c@{}}{\tiny 57}\\[1pt]0\end{tabular} & \begin{tabular}{@{}c@{}}{\tiny 58}\\[1pt]0\end{tabular} & \begin{tabular}{@{}c@{}}{\tiny 59}\\[1pt]0\end{tabular} & \begin{tabular}{@{}c@{}}{\tiny 60}\\[1pt]0\end{tabular} & \begin{tabular}{@{}c@{}}{\tiny 61}\\[1pt]0\end{tabular} & \begin{tabular}{@{}c@{}}{\tiny 62}\\[1pt]0\end{tabular} & \begin{tabular}{@{}c@{}}{\tiny 63}\\[1pt]0\end{tabular} \\[5pt] \hline
7 & \begin{tabular}{@{}c@{}}{\tiny 48}\\[1pt]0\end{tabular} & \begin{tabular}{@{}c@{}}{\tiny 49}\\[1pt]0\end{tabular} & \begin{tabular}{@{}c@{}}{\tiny 50}\\[1pt]0\end{tabular} & \begin{tabular}{@{}c@{}}{\tiny 51}\\[1pt]\textbf{1}\end{tabular} & \begin{tabular}{@{}c@{}}{\tiny 52}\\[1pt]0\end{tabular} & \begin{tabular}{@{}c@{}}{\tiny 53}\\[1pt]0\end{tabular} & \begin{tabular}{@{}c@{}}{\tiny 54}\\[1pt]0\end{tabular} & \begin{tabular}{@{}c@{}}{\tiny 55}\\[1pt]0\end{tabular} \\[5pt] \hline
6 & \begin{tabular}{@{}c@{}}{\tiny 40}\\[1pt]0\end{tabular} & \begin{tabular}{@{}c@{}}{\tiny 41}\\[1pt]0\end{tabular} & \begin{tabular}{@{}c@{}}{\tiny 42}\\[1pt]0\end{tabular} & \begin{tabular}{@{}c@{}}{\tiny 43}\\[1pt]\textbf{1}\end{tabular} & \begin{tabular}{@{}c@{}}{\tiny 44}\\[1pt]0\end{tabular} & \begin{tabular}{@{}c@{}}{\tiny 45}\\[1pt]0\end{tabular} & \begin{tabular}{@{}c@{}}{\tiny 46}\\[1pt]\textbf{1}\end{tabular} & \begin{tabular}{@{}c@{}}{\tiny 47}\\[1pt]0\end{tabular} \\[5pt] \hline
5 & \begin{tabular}{@{}c@{}}{\tiny 32}\\[1pt]0\end{tabular} & \begin{tabular}{@{}c@{}}{\tiny 33}\\[1pt]\textbf{1}\end{tabular} & \begin{tabular}{@{}c@{}}{\tiny 34}\\[1pt]0\end{tabular} & \begin{tabular}{@{}c@{}}{\tiny 35}\\[1pt]\textbf{1}\end{tabular} & \begin{tabular}{@{}c@{}}{\tiny 36}\\[1pt]0\end{tabular} & \begin{tabular}{@{}c@{}}{\tiny 37}\\[1pt]\textbf{1}\end{tabular} & \begin{tabular}{@{}c@{}}{\tiny 38}\\[1pt]0\end{tabular} & \begin{tabular}{@{}c@{}}{\tiny 39}\\[1pt]0\end{tabular} \\[5pt] \hline
4 & \begin{tabular}{@{}c@{}}{\tiny 24}\\[1pt]0\end{tabular} & \begin{tabular}{@{}c@{}}{\tiny 25}\\[1pt]0\end{tabular} & \begin{tabular}{@{}c@{}}{\tiny 26}\\[1pt]\textbf{1}\end{tabular} & \begin{tabular}{@{}c@{}}{\tiny 27}\\[1pt]\textbf{1}\end{tabular} & \begin{tabular}{@{}c@{}}{\tiny 28}\\[1pt]\textbf{1}\end{tabular} & \begin{tabular}{@{}c@{}}{\tiny 29}\\[1pt]0\end{tabular} & \begin{tabular}{@{}c@{}}{\tiny 30}\\[1pt]0\end{tabular} & \begin{tabular}{@{}c@{}}{\tiny 31}\\[1pt]0\end{tabular} \\[5pt] \hline
3 & \begin{tabular}{@{}c@{}}{\tiny 16}\\[1pt]0\end{tabular} & \begin{tabular}{@{}c@{}}{\tiny 17}\\[1pt]\textbf{1}\end{tabular} & \begin{tabular}{@{}c@{}}{\tiny 18}\\[1pt]\textbf{1}\end{tabular} & \cellcolor{lightgray}\begin{tabular}{@{}c@{}}{\tiny 19}\\[1pt]\textbf{0}\end{tabular} & \begin{tabular}{@{}c@{}}{\tiny 20}\\[1pt]\textbf{1}\end{tabular} & \begin{tabular}{@{}c@{}}{\tiny 21}\\[1pt]\textbf{1}\end{tabular} & \begin{tabular}{@{}c@{}}{\tiny 22}\\[1pt]\textbf{1}\end{tabular} & \begin{tabular}{@{}c@{}}{\tiny 23}\\[1pt]0\end{tabular} \\[5pt] \hline
2 & \begin{tabular}{@{}c@{}}{\tiny 8}\\[1pt]0\end{tabular} & \begin{tabular}{@{}c@{}}{\tiny 9}\\[1pt]0\end{tabular} & \begin{tabular}{@{}c@{}}{\tiny 10}\\[1pt]\textbf{1}\end{tabular} & \begin{tabular}{@{}c@{}}{\tiny 11}\\[1pt]\textbf{1}\end{tabular} & \begin{tabular}{@{}c@{}}{\tiny 12}\\[1pt]\textbf{1}\end{tabular} & \begin{tabular}{@{}c@{}}{\tiny 13}\\[1pt]0\end{tabular} & \begin{tabular}{@{}c@{}}{\tiny 14}\\[1pt]0\end{tabular} & \begin{tabular}{@{}c@{}}{\tiny 15}\\[1pt]0\end{tabular} \\[5pt] \hline
1 & \begin{tabular}{@{}c@{}}{\tiny 0}\\[1pt]0\end{tabular} & \begin{tabular}{@{}c@{}}{\tiny 1}\\[1pt]0\end{tabular} & \begin{tabular}{@{}c@{}}{\tiny 2}\\[1pt]0\end{tabular} & \begin{tabular}{@{}c@{}}{\tiny 3}\\[1pt]0\end{tabular} & \begin{tabular}{@{}c@{}}{\tiny 4}\\[1pt]0\end{tabular} & \begin{tabular}{@{}c@{}}{\tiny 5}\\[1pt]0\end{tabular} & \begin{tabular}{@{}c@{}}{\tiny 6}\\[1pt]0\end{tabular} & \begin{tabular}{@{}c@{}}{\tiny 7}\\[1pt]0\end{tabular} \\[5pt] \hline
\multicolumn{1}{c|}{} & a & b & c & d & e & f & g & h \\
\cline{2-9}
\end{tabular}
\end{table}

En effet, on peut scinder les mouvements d'une pièce en mouvements diagonaux et orthogonaux. Nous obtiendrons deux tableaux~: \mintinline{c++}{MOVES_D3_DIAG} et \mintinline{c++}{MOVES_D3_ORTHO}. Nous aurons à consulter deux tableaux au lieu d'un seul, pour obtenir les coups possibles en {\ttfamily d3} (ou n'importe quelle autre case), mais nous allons gagner beaucoup de place mémoire. Nous verrons comment combiner les valeurs de ces deux tableaux plus loin dans ce chapitre. Au lieu du masque de la table \ref{tab:mask_for_d3_without_edge}, nous aurons maintenant les deux masques de la figure \ref{fig:diagonal_orthogonal_masks_d3}.
Pour le masque orthogonal de la figure \ref{fig:orthogonal_mask_d3}, on a $2^{10} = 1024$ sous-ensembles et pour le masque diagonal de la figure \ref{fig:diagonal_mask_d3}, on a $2^{7} = 128$ sous-ensembles, on passe donc de $2^{17} = 131072$ à $2^{10} + 2^{7} = 1152$ configurations d'obstacles à considérer pour la pièce en {\ttfamily d3}~!\\

\begin{figure}[htpb]
\centering
\begin{subfigure}[b]{0.42\textwidth}
\centering
\small
\begin{tabular}{|c|*{8}{>{\centering\arraybackslash}p{0.4cm}|}}
\hline
8 & \begin{tabular}{@{}c@{}}{\tiny 56}\\[0pt]0\end{tabular} & \begin{tabular}{@{}c@{}}{\tiny 57}\\[0pt]0\end{tabular} & \begin{tabular}{@{}c@{}}{\tiny 58}\\[0pt]0\end{tabular} & \begin{tabular}{@{}c@{}}{\tiny 59}\\[0pt]0\end{tabular} & \begin{tabular}{@{}c@{}}{\tiny 60}\\[0pt]0\end{tabular} & \begin{tabular}{@{}c@{}}{\tiny 61}\\[0pt]0\end{tabular} & \begin{tabular}{@{}c@{}}{\tiny 62}\\[0pt]0\end{tabular} & \begin{tabular}{@{}c@{}}{\tiny 63}\\[0pt]0\end{tabular} \\[3pt] \hline
7 & \begin{tabular}{@{}c@{}}{\tiny 48}\\[0pt]0\end{tabular} & \begin{tabular}{@{}c@{}}{\tiny 49}\\[0pt]0\end{tabular} & \begin{tabular}{@{}c@{}}{\tiny 50}\\[0pt]0\end{tabular} & \begin{tabular}{@{}c@{}}{\tiny 51}\\[0pt]\textbf{1}\end{tabular} & \begin{tabular}{@{}c@{}}{\tiny 52}\\[0pt]0\end{tabular} & \begin{tabular}{@{}c@{}}{\tiny 53}\\[0pt]0\end{tabular} & \begin{tabular}{@{}c@{}}{\tiny 54}\\[0pt]0\end{tabular} & \begin{tabular}{@{}c@{}}{\tiny 55}\\[0pt]0\end{tabular} \\[3pt] \hline
6 & \begin{tabular}{@{}c@{}}{\tiny 40}\\[0pt]0\end{tabular} & \begin{tabular}{@{}c@{}}{\tiny 41}\\[0pt]0\end{tabular} & \begin{tabular}{@{}c@{}}{\tiny 42}\\[0pt]0\end{tabular} & \begin{tabular}{@{}c@{}}{\tiny 43}\\[0pt]\textbf{1}\end{tabular} & \begin{tabular}{@{}c@{}}{\tiny 44}\\[0pt]0\end{tabular} & \begin{tabular}{@{}c@{}}{\tiny 45}\\[0pt]0\end{tabular} & \begin{tabular}{@{}c@{}}{\tiny 46}\\[0pt]0\end{tabular} & \begin{tabular}{@{}c@{}}{\tiny 47}\\[0pt]0\end{tabular} \\[3pt] \hline
5 & \begin{tabular}{@{}c@{}}{\tiny 32}\\[0pt]0\end{tabular} & \begin{tabular}{@{}c@{}}{\tiny 33}\\[0pt]0\end{tabular} & \begin{tabular}{@{}c@{}}{\tiny 34}\\[0pt]0\end{tabular} & \begin{tabular}{@{}c@{}}{\tiny 35}\\[0pt]\textbf{1}\end{tabular} & \begin{tabular}{@{}c@{}}{\tiny 36}\\[0pt]0\end{tabular} & \begin{tabular}{@{}c@{}}{\tiny 37}\\[0pt]0\end{tabular} & \begin{tabular}{@{}c@{}}{\tiny 38}\\[0pt]0\end{tabular} & \begin{tabular}{@{}c@{}}{\tiny 39}\\[0pt]0\end{tabular} \\[3pt] \hline
4 & \begin{tabular}{@{}c@{}}{\tiny 24}\\[0pt]0\end{tabular} & \begin{tabular}{@{}c@{}}{\tiny 25}\\[0pt]0\end{tabular} & \begin{tabular}{@{}c@{}}{\tiny 26}\\[0pt]0\end{tabular} & \begin{tabular}{@{}c@{}}{\tiny 27}\\[0pt]\textbf{1}\end{tabular} & \begin{tabular}{@{}c@{}}{\tiny 28}\\[0pt]0\end{tabular} & \begin{tabular}{@{}c@{}}{\tiny 29}\\[0pt]0\end{tabular} & \begin{tabular}{@{}c@{}}{\tiny 30}\\[0pt]0\end{tabular} & \begin{tabular}{@{}c@{}}{\tiny 31}\\[0pt]0\end{tabular} \\[3pt] \hline
3 & \begin{tabular}{@{}c@{}}{\tiny 16}\\[0pt]0\end{tabular} & \begin{tabular}{@{}c@{}}{\tiny 17}\\[0pt]\textbf{1}\end{tabular} & \begin{tabular}{@{}c@{}}{\tiny 18}\\[0pt]\textbf{1}\end{tabular} & \cellcolor{lightgray}\begin{tabular}{@{}c@{}}{\tiny 19}\\[0pt]\textbf{0}\end{tabular} & \begin{tabular}{@{}c@{}}{\tiny 20}\\[0pt]\textbf{1}\end{tabular} & \begin{tabular}{@{}c@{}}{\tiny 21}\\[0pt]\textbf{1}\end{tabular} & \begin{tabular}{@{}c@{}}{\tiny 22}\\[0pt]\textbf{1}\end{tabular} & \begin{tabular}{@{}c@{}}{\tiny 23}\\[0pt]0\end{tabular} \\[3pt] \hline
2 & \begin{tabular}{@{}c@{}}{\tiny 8}\\[0pt]0\end{tabular} & \begin{tabular}{@{}c@{}}{\tiny 9}\\[0pt]0\end{tabular} & \begin{tabular}{@{}c@{}}{\tiny 10}\\[0pt]0\end{tabular} & \begin{tabular}{@{}c@{}}{\tiny 11}\\[0pt]\textbf{1}\end{tabular} & \begin{tabular}{@{}c@{}}{\tiny 12}\\[0pt]0\end{tabular} & \begin{tabular}{@{}c@{}}{\tiny 13}\\[0pt]0\end{tabular} & \begin{tabular}{@{}c@{}}{\tiny 14}\\[0pt]0\end{tabular} & \begin{tabular}{@{}c@{}}{\tiny 15}\\[0pt]0\end{tabular} \\[3pt] \hline
1 & \begin{tabular}{@{}c@{}}{\tiny 0}\\[0pt]0\end{tabular} & \begin{tabular}{@{}c@{}}{\tiny 1}\\[0pt]0\end{tabular} & \begin{tabular}{@{}c@{}}{\tiny 2}\\[0pt]0\end{tabular} & \begin{tabular}{@{}c@{}}{\tiny 3}\\[0pt]0\end{tabular} & \begin{tabular}{@{}c@{}}{\tiny 4}\\[0pt]0\end{tabular} & \begin{tabular}{@{}c@{}}{\tiny 5}\\[0pt]0\end{tabular} & \begin{tabular}{@{}c@{}}{\tiny 6}\\[0pt]0\end{tabular} & \begin{tabular}{@{}c@{}}{\tiny 7}\\[0pt]0\end{tabular} \\[3pt] \hline
\multicolumn{1}{c|}{} & a & b & c & d & e & f & g & h \\
\cline{2-9}
\end{tabular}
\caption{Orthogonal mask}
\label{fig:orthogonal_mask_d3}
\end{subfigure}
\hfill
\begin{subfigure}[b]{0.42\textwidth}
\centering
\small
\begin{tabular}{|c|*{8}{>{\centering\arraybackslash}p{0.4cm}|}}
\hline
8 & \begin{tabular}{@{}c@{}}{\tiny 56}\\[0pt]0\end{tabular} & \begin{tabular}{@{}c@{}}{\tiny 57}\\[0pt]0\end{tabular} & \begin{tabular}{@{}c@{}}{\tiny 58}\\[0pt]0\end{tabular} & \begin{tabular}{@{}c@{}}{\tiny 59}\\[0pt]0\end{tabular} & \begin{tabular}{@{}c@{}}{\tiny 60}\\[0pt]0\end{tabular} & \begin{tabular}{@{}c@{}}{\tiny 61}\\[0pt]0\end{tabular} & \begin{tabular}{@{}c@{}}{\tiny 62}\\[0pt]0\end{tabular} & \begin{tabular}{@{}c@{}}{\tiny 63}\\[0pt]0\end{tabular} \\[3pt] \hline
7 & \begin{tabular}{@{}c@{}}{\tiny 48}\\[0pt]0\end{tabular} & \begin{tabular}{@{}c@{}}{\tiny 49}\\[0pt]0\end{tabular} & \begin{tabular}{@{}c@{}}{\tiny 50}\\[0pt]0\end{tabular} & \begin{tabular}{@{}c@{}}{\tiny 51}\\[0pt]0\end{tabular} & \begin{tabular}{@{}c@{}}{\tiny 52}\\[0pt]0\end{tabular} & \begin{tabular}{@{}c@{}}{\tiny 53}\\[0pt]0\end{tabular} & \begin{tabular}{@{}c@{}}{\tiny 54}\\[0pt]0\end{tabular} & \begin{tabular}{@{}c@{}}{\tiny 55}\\[0pt]0\end{tabular} \\[3pt] \hline
6 & \begin{tabular}{@{}c@{}}{\tiny 40}\\[0pt]0\end{tabular} & \begin{tabular}{@{}c@{}}{\tiny 41}\\[0pt]0\end{tabular} & \begin{tabular}{@{}c@{}}{\tiny 42}\\[0pt]0\end{tabular} & \begin{tabular}{@{}c@{}}{\tiny 43}\\[0pt]0\end{tabular} & \begin{tabular}{@{}c@{}}{\tiny 44}\\[0pt]0\end{tabular} & \begin{tabular}{@{}c@{}}{\tiny 45}\\[0pt]0\end{tabular} & \begin{tabular}{@{}c@{}}{\tiny 46}\\[0pt]\textbf{1}\end{tabular} & \begin{tabular}{@{}c@{}}{\tiny 47}\\[0pt]0\end{tabular} \\[3pt] \hline
5 & \begin{tabular}{@{}c@{}}{\tiny 32}\\[0pt]0\end{tabular} & \begin{tabular}{@{}c@{}}{\tiny 33}\\[0pt]\textbf{1}\end{tabular} & \begin{tabular}{@{}c@{}}{\tiny 34}\\[0pt]0\end{tabular} & \begin{tabular}{@{}c@{}}{\tiny 35}\\[0pt]0\end{tabular} & \begin{tabular}{@{}c@{}}{\tiny 36}\\[0pt]0\end{tabular} & \begin{tabular}{@{}c@{}}{\tiny 37}\\[0pt]\textbf{1}\end{tabular} & \begin{tabular}{@{}c@{}}{\tiny 38}\\[0pt]0\end{tabular} & \begin{tabular}{@{}c@{}}{\tiny 39}\\[0pt]0\end{tabular} \\[3pt] \hline
4 & \begin{tabular}{@{}c@{}}{\tiny 24}\\[0pt]0\end{tabular} & \begin{tabular}{@{}c@{}}{\tiny 25}\\[0pt]0\end{tabular} & \begin{tabular}{@{}c@{}}{\tiny 26}\\[0pt]\textbf{1}\end{tabular} & \begin{tabular}{@{}c@{}}{\tiny 27}\\[0pt]0\end{tabular} & \begin{tabular}{@{}c@{}}{\tiny 28}\\[0pt]\textbf{1}\end{tabular} & \begin{tabular}{@{}c@{}}{\tiny 29}\\[0pt]0\end{tabular} & \begin{tabular}{@{}c@{}}{\tiny 30}\\[0pt]0\end{tabular} & \begin{tabular}{@{}c@{}}{\tiny 31}\\[0pt]0\end{tabular} \\[3pt] \hline
3 & \begin{tabular}{@{}c@{}}{\tiny 16}\\[0pt]0\end{tabular} & \begin{tabular}{@{}c@{}}{\tiny 17}\\[0pt]0\end{tabular} & \begin{tabular}{@{}c@{}}{\tiny 18}\\[0pt]0\end{tabular} & \cellcolor{lightgray}\begin{tabular}{@{}c@{}}{\tiny 19}\\[0pt]\textbf{0}\end{tabular} & \begin{tabular}{@{}c@{}}{\tiny 20}\\[0pt]0\end{tabular} & \begin{tabular}{@{}c@{}}{\tiny 21}\\[0pt]0\end{tabular} & \begin{tabular}{@{}c@{}}{\tiny 22}\\[0pt]0\end{tabular} & \begin{tabular}{@{}c@{}}{\tiny 23}\\[0pt]0\end{tabular} \\[3pt] \hline
2 & \begin{tabular}{@{}c@{}}{\tiny 8}\\[0pt]0\end{tabular} & \begin{tabular}{@{}c@{}}{\tiny 9}\\[0pt]0\end{tabular} & \begin{tabular}{@{}c@{}}{\tiny 10}\\[0pt]\textbf{1}\end{tabular} & \begin{tabular}{@{}c@{}}{\tiny 11}\\[0pt]0\end{tabular} & \begin{tabular}{@{}c@{}}{\tiny 12}\\[0pt]\textbf{1}\end{tabular} & \begin{tabular}{@{}c@{}}{\tiny 13}\\[0pt]0\end{tabular} & \begin{tabular}{@{}c@{}}{\tiny 14}\\[0pt]0\end{tabular} & \begin{tabular}{@{}c@{}}{\tiny 15}\\[0pt]0\end{tabular} \\[3pt] \hline
1 & \begin{tabular}{@{}c@{}}{\tiny 0}\\[0pt]0\end{tabular} & \begin{tabular}{@{}c@{}}{\tiny 1}\\[0pt]0\end{tabular} & \begin{tabular}{@{}c@{}}{\tiny 2}\\[0pt]0\end{tabular} & \begin{tabular}{@{}c@{}}{\tiny 3}\\[0pt]0\end{tabular} & \begin{tabular}{@{}c@{}}{\tiny 4}\\[0pt]0\end{tabular} & \begin{tabular}{@{}c@{}}{\tiny 5}\\[0pt]0\end{tabular} & \begin{tabular}{@{}c@{}}{\tiny 6}\\[0pt]0\end{tabular} & \begin{tabular}{@{}c@{}}{\tiny 7}\\[0pt]0\end{tabular} \\[3pt] \hline
\multicolumn{1}{c|}{} & a & b & c & d & e & f & g & h \\
\cline{2-9}
\end{tabular}
\caption{Diagonal mask}
\label{fig:diagonal_mask_d3}
\end{subfigure}

\caption{Diagonal and orthogonal masks for d3 without edge squares}
\label{fig:diagonal_orthogonal_masks_d3}
\end{figure}

Nous pouvons maintenant étudier comment trouver les constantes \mintinline{c++}{MAGIC} et \mintinline{c++}{K} pour les coups diagonaux et orthogonaux de chacune des cases du plateau. Le code est donné dans le listing \ref{lst:find_magic}\footnote{J'ai essayé d'utiliser le solveur \gls{z3} pour obtenir ces constantes mais sans succès.}. Une exécution de ce programme\footnote{Au vu du générateur aléatoire et de son initialisation, le programme pourra trouver des constantes différentes pour des exécutions différentes.} est donnée dans le listing \ref{lst:find_magic_exec}\footnote{La sortie est reformatée pour être plus lisible.}.

\begin{mdframed}[skipabove=\baselineskip,hidealllines=true]
\begin{minted}[linenos, breaklines,fontsize=\small]{cpp}
constexpr uint64_t FileABB = 0x0101010101010101;
constexpr uint64_t FileHBB = FileABB << 7;
constexpr uint64_t Rank1BB = 0xFF;
constexpr uint64_t Rank8BB = Rank1BB << (8 * 7);

constexpr bool is_ok(Square s) { return s >= SQ_A1 && s <= SQ_H8; }

constexpr File file_of(Square s) { return File(s & 7); }

constexpr Rank rank_of(Square s) { return Rank(s >> 3); }

constexpr uint64_t rank_bb(Rank r) { return Rank1BB << (8 * r); }

constexpr uint64_t rank_bb(Square s) { return rank_bb(rank_of(s)); }

constexpr uint64_t file_bb(File f) { return FileABB << f; }

constexpr uint64_t file_bb(Square s) { return file_bb(file_of(s)); }

constexpr uint64_t square_bb(Square s) {
    return uint64_t(1) << s;
}

constexpr Square operator+(Square s, Direction d) { 
    return Square(int(s) + int(d)); 
}

int manhattan_distance(Square sq1, Square sq2) {
    int d_rank = std::abs(rank_of(sq1) - rank_of(sq2));
    int d_file = std::abs(file_of(sq1) - file_of(sq2));
    return d_rank + d_file;
}

enum MoveType {
    ORTHOGONAL,
    DIAGONAL
};

uint64_t reachable_squares(MoveType mt, Square sq, uint64_t occupied) {
    uint64_t  moves    = 0;
    Direction o_dir[4] = {NORTH, SOUTH, EAST, WEST};
    Direction d_dir[4] = {NORTH_EAST,SOUTH_EAST,SOUTH_WEST,NORTH_WEST};
    for (Direction d : (mt == ORTHOGONAL ? o_dir : d_dir)) {
        Square s = sq;
        while (true) {            
            Square to = s + d;
            if (!is_ok(to) || manhattan_distance(s, to) > 2) break;
            uint64_t bb = square_bb(to);            
            if ((square_bb(to) & occupied) != 0) break;
            s = to;
            moves |= bb;
        } 
    }
    return moves;
}

std::pair<int, uint64_t> magic_for_square(MoveType mt, Square sq) {    
    using namespace std;
    uint64_t edges = ((Rank1BB | Rank8BB) & ~rank_bb(sq)) | 
                        ((FileABB | FileHBB) & ~file_bb(sq));
    uint64_t moves_bb = reachable_squares(mt, sq, 0) & ~edges;
    vector<uint64_t> occupancies;
    vector<uint64_t> possible_moves;
    uint64_t b = 0;
    int size = 0;
    do {
        occupancies.push_back(b);
        possible_moves.push_back(sliding_moves(mt, sq, b));
        size++;
        b = (b - moves_bb) & moves_bb;
    } while (b);
    int k = popcount(moves_bb);
    int shift = 64 - k;                
    random_device rd;
    mt19937_64 twister(rd());
    uniform_int_distribution<uint64_t> d;
    vector<uint32_t> seen(1 << k);
    vector<uint64_t> moves(1 << k);                    
    for (uint32_t cnt = 0;; cnt++) {
        uint64_t magic = d(twister) & d(twister) & d(twister);
        bool found = true;                    
        for (size_t j = 0; j < occupancies.size(); j++) {
            uint64_t occ = occupancies[j];
            int index = magic * occ >> shift;
            if (seen[index] == cnt && moves[index] != possible_moves[j]) 
            {
                found = false;
                break;
            }
            seen[index] = cnt;
            moves[index] = possible_moves[j];
        }
        if (found) {
            return {k, magic};
        }
    }
    unreachable();
}

int main() {
    using namespace std;
    for (MoveType mt : {ORTHOGONAL, DIAGONAL}) {
        stringstream ss_k, ss_magic;
        ss_k << format("int {}_K[64] = {{",
                        mt == ORTHOGONAL ? "H" : "D");
        ss_magic << format("uint64_t {}_MAGIC[64] = {{",
                            mt == ORTHOGONAL ? "H" : "D");
        for (int sq = SQ_A1; sq <= SQ_H8; sq++) {
            const auto [k, magic] = magic_for_square(mt, Square(sq));
            ss_k << dec << k << ',';
            ss_magic << showbase << hex << magic << ',';
        }
        cout << ss_k.str() << "};\n";
        cout << ss_magic.str() << "};\n\n";
    }
}
\end{minted}
\captionof{listing}{Programme permettant de trouver les constantes \mintinline{c++}{K} et \mintinline{c++}{MAGIC} du listing \ref{lst:magic_perfect_hashing}}
\label{lst:find_magic}
\end{mdframed} 

Dans le listing \ref{lst:find_magic},

\begin{itemize}

\item À la ligne 39, la fonction \mintinline{c++}{uint64_t reachable_squares(MoveType mt, Square sq, uint64_t occupied)} va nous permettre de créer le masque des cases accessibles à partir de la case \mintinline{c++}{sq} et pour le type de mouvement \mintinline{c++}{mt}, qui sera soit orthogonal (\mintinline{c++}{ORTHOGONAL} à la ligne 35), soit diagonal (\mintinline{c++}{DIAGONAL} à la ligne 36). Le masque créé sera similaire aux masques de la figure \ref{fig:diagonal_orthogonal_masks_d3}.
\begin{itemize}
\item Aux lignes 43 à 53, nous parcourons tour à tour une des directions selon le type de mouvement, en partant de la case de départ (\mintinline{c++}{sq})jusqu'à rencontrer un obstacle.
\item À la ligne on test si on ne sort pas du plateau ou si on ne fait pas le tour, par exemple, si l'on se trouve dans la case \mintinline{c++}{SQ_A8} et que la direction est \mintinline{c++}{EAST}, \mintinline{c++}{SQ_H8 + EAST == SQ_B1}, le test avec la distance de Manhattan va nous permettre de d'exclure ce genre de cas (voir à la ligne 28 la définition de la distance de Manhattan et le listing \ref{lst:files_and_ranks} pour les définitions de \mintinline{c++}{File} et \mintinline{c++}{Rank}).
\item À la ligne 48, on transforme la case où l'on se trouve en un bitboard qui contient un seul $1$, sur le bit correspondant à la case.
\item À la ligne 49, on test s'il n'y a pas un obstacle sur la case où l'on se trouve grâce au bitboard que l'on vient de créer et du bitboard \mintinline{c++}{occupied} contenant les obstacles du plateau.
\item À la ligne 50, on se déplace d'une case dans la direction \mintinline{c++}{d}.
\item À la ligne 51, on ajoute la case d'où l'on vient comme coup possible.
\end{itemize}

\item À la ligne 57, la fonction \mintinline{c++}{pair<int, uint64_t> magic_for_square(MoveType mt, Square sq)} va renvoyer une paire dont le premier élément sera la valeur de \mintinline{c++}{K}, et le deuxième élément sera la constante \mintinline{c++}{MAGIC} (voir listing \ref{lst:magic_perfect_hashing}), pour le type de mouvement \mintinline{c++}{mt} (\mintinline{c++}{ORTHOGONAL} ou \mintinline{c++}{DIAGONAL}) et la case \mintinline{c++}{sq}.
\begin{itemize}
\item Aux lignes 59 à 60, on crée un masque qui va couvrir les bords du plateaux mais en faisant attention de ne pas exclure des cases qui doivent rester accessibles par la pièce. Par exemple, si \mintinline{c++}{sq == SQ_A1}, on obtient le bitboard de la figure \ref{fig:edges_mask_a1} et si \mintinline{c++}{sq == SQ_C2}, on obtient celui de la figure \ref{fig:edges_mask_c2}.

\begin{figure}[htpb]
\centering
\begin{subfigure}[b]{0.42\textwidth}
\centering
\small
\begin{tabular}{|c|*{8}{>{\centering\arraybackslash}p{0.4cm}|}}
\hline
8 & \begin{tabular}{@{}c@{}}{\tiny 56}\\[0pt]\textbf{1}\end{tabular} & \begin{tabular}{@{}c@{}}{\tiny 57}\\[0pt]\textbf{1}\end{tabular} & \begin{tabular}{@{}c@{}}{\tiny 58}\\[0pt]\textbf{1}\end{tabular} & \begin{tabular}{@{}c@{}}{\tiny 59}\\[0pt]\textbf{1}\end{tabular} & \begin{tabular}{@{}c@{}}{\tiny 60}\\[0pt]\textbf{1}\end{tabular} & \begin{tabular}{@{}c@{}}{\tiny 61}\\[0pt]\textbf{1}\end{tabular} & \begin{tabular}{@{}c@{}}{\tiny 62}\\[0pt]\textbf{1}\end{tabular} & \begin{tabular}{@{}c@{}}{\tiny 63}\\[0pt]\textbf{1}\end{tabular} \\[3pt] \hline
7 & \begin{tabular}{@{}c@{}}{\tiny 48}\\[0pt]0\end{tabular} & \begin{tabular}{@{}c@{}}{\tiny 49}\\[0pt]0\end{tabular} & \begin{tabular}{@{}c@{}}{\tiny 50}\\[0pt]0\end{tabular} & \begin{tabular}{@{}c@{}}{\tiny 51}\\[0pt]0\end{tabular} & \begin{tabular}{@{}c@{}}{\tiny 52}\\[0pt]0\end{tabular} & \begin{tabular}{@{}c@{}}{\tiny 53}\\[0pt]0\end{tabular} & \begin{tabular}{@{}c@{}}{\tiny 54}\\[0pt]0\end{tabular} & \begin{tabular}{@{}c@{}}{\tiny 55}\\[0pt]\textbf{1}\end{tabular} \\[3pt] \hline
6 & \begin{tabular}{@{}c@{}}{\tiny 40}\\[0pt]0\end{tabular} & \begin{tabular}{@{}c@{}}{\tiny 41}\\[0pt]0\end{tabular} & \begin{tabular}{@{}c@{}}{\tiny 42}\\[0pt]0\end{tabular} & \begin{tabular}{@{}c@{}}{\tiny 43}\\[0pt]0\end{tabular} & \begin{tabular}{@{}c@{}}{\tiny 44}\\[0pt]0\end{tabular} & \begin{tabular}{@{}c@{}}{\tiny 45}\\[0pt]0\end{tabular} & \begin{tabular}{@{}c@{}}{\tiny 46}\\[0pt]0\end{tabular} & \begin{tabular}{@{}c@{}}{\tiny 47}\\[0pt]\textbf{1}\end{tabular} \\[3pt] \hline
5 & \begin{tabular}{@{}c@{}}{\tiny 32}\\[0pt]0\end{tabular} & \begin{tabular}{@{}c@{}}{\tiny 33}\\[0pt]0\end{tabular} & \begin{tabular}{@{}c@{}}{\tiny 34}\\[0pt]0\end{tabular} & \begin{tabular}{@{}c@{}}{\tiny 35}\\[0pt]0\end{tabular} & \begin{tabular}{@{}c@{}}{\tiny 36}\\[0pt]0\end{tabular} & \begin{tabular}{@{}c@{}}{\tiny 37}\\[0pt]0\end{tabular} & \begin{tabular}{@{}c@{}}{\tiny 38}\\[0pt]0\end{tabular} & \begin{tabular}{@{}c@{}}{\tiny 39}\\[0pt]\textbf{1}\end{tabular} \\[3pt] \hline
4 & \begin{tabular}{@{}c@{}}{\tiny 24}\\[0pt]0\end{tabular} & \begin{tabular}{@{}c@{}}{\tiny 25}\\[0pt]0\end{tabular} & \begin{tabular}{@{}c@{}}{\tiny 26}\\[0pt]0\end{tabular} & \begin{tabular}{@{}c@{}}{\tiny 27}\\[0pt]0\end{tabular} & \begin{tabular}{@{}c@{}}{\tiny 28}\\[0pt]0\end{tabular} & \begin{tabular}{@{}c@{}}{\tiny 29}\\[0pt]0\end{tabular} & \begin{tabular}{@{}c@{}}{\tiny 30}\\[0pt]0\end{tabular} & \begin{tabular}{@{}c@{}}{\tiny 31}\\[0pt]\textbf{1}\end{tabular} \\[3pt] \hline
3 & \begin{tabular}{@{}c@{}}{\tiny 16}\\[0pt]0\end{tabular} & \begin{tabular}{@{}c@{}}{\tiny 17}\\[0pt]0\end{tabular} & \begin{tabular}{@{}c@{}}{\tiny 18}\\[0pt]0\end{tabular} & \begin{tabular}{@{}c@{}}{\tiny 19}\\[0pt]0\end{tabular} & \begin{tabular}{@{}c@{}}{\tiny 20}\\[0pt]0\end{tabular} & \begin{tabular}{@{}c@{}}{\tiny 21}\\[0pt]0\end{tabular} & \begin{tabular}{@{}c@{}}{\tiny 22}\\[0pt]0\end{tabular} & \begin{tabular}{@{}c@{}}{\tiny 23}\\[0pt]\textbf{1}\end{tabular} \\[3pt] \hline
2 & \begin{tabular}{@{}c@{}}{\tiny 8}\\[0pt]0\end{tabular} & \begin{tabular}{@{}c@{}}{\tiny 9}\\[0pt]0\end{tabular} & \begin{tabular}{@{}c@{}}{\tiny 10}\\[0pt]0\end{tabular} & \begin{tabular}{@{}c@{}}{\tiny 11}\\[0pt]0\end{tabular} & \begin{tabular}{@{}c@{}}{\tiny 12}\\[0pt]0\end{tabular} & \begin{tabular}{@{}c@{}}{\tiny 13}\\[0pt]0\end{tabular} & \begin{tabular}{@{}c@{}}{\tiny 14}\\[0pt]0\end{tabular} & \begin{tabular}{@{}c@{}}{\tiny 15}\\[0pt]\textbf{1}\end{tabular} \\[3pt] \hline
1 & \cellcolor{lightgray}\begin{tabular}{@{}c@{}}{\tiny 0}\\[0pt]\textbf{0}\end{tabular} & \begin{tabular}{@{}c@{}}{\tiny 1}\\[0pt]0\end{tabular} & \begin{tabular}{@{}c@{}}{\tiny 2}\\[0pt]0\end{tabular} & \begin{tabular}{@{}c@{}}{\tiny 3}\\[0pt]0\end{tabular} & \begin{tabular}{@{}c@{}}{\tiny 4}\\[0pt]0\end{tabular} & \begin{tabular}{@{}c@{}}{\tiny 5}\\[0pt]0\end{tabular} & \begin{tabular}{@{}c@{}}{\tiny 6}\\[0pt]0\end{tabular} & \begin{tabular}{@{}c@{}}{\tiny 7}\\[0pt]\textbf{1}\end{tabular} \\[3pt] \hline
\multicolumn{1}{c|}{} & a & b & c & d & e & f & g & h \\
\cline{2-9}
\end{tabular}
\caption{Edge mask for square a1}
\label{fig:edges_mask_a1}
\end{subfigure}
\hfill
\begin{subfigure}[b]{0.42\textwidth}
\centering
\small
\begin{tabular}{|c|*{8}{>{\centering\arraybackslash}p{0.4cm}|}}
\hline
8 & \begin{tabular}{@{}c@{}}{\tiny 56}\\[0pt]\textbf{1}\end{tabular} & \begin{tabular}{@{}c@{}}{\tiny 57}\\[0pt]\textbf{1}\end{tabular} & \begin{tabular}{@{}c@{}}{\tiny 58}\\[0pt]\textbf{1}\end{tabular} & \begin{tabular}{@{}c@{}}{\tiny 59}\\[0pt]\textbf{1}\end{tabular} & \begin{tabular}{@{}c@{}}{\tiny 60}\\[0pt]\textbf{1}\end{tabular} & \begin{tabular}{@{}c@{}}{\tiny 61}\\[0pt]\textbf{1}\end{tabular} & \begin{tabular}{@{}c@{}}{\tiny 62}\\[0pt]\textbf{1}\end{tabular} & \begin{tabular}{@{}c@{}}{\tiny 63}\\[0pt]\textbf{1}\end{tabular} \\[3pt] \hline
7 & \begin{tabular}{@{}c@{}}{\tiny 48}\\[0pt]\textbf{1}\end{tabular} & \begin{tabular}{@{}c@{}}{\tiny 49}\\[0pt]0\end{tabular} & \begin{tabular}{@{}c@{}}{\tiny 50}\\[0pt]0\end{tabular} & \begin{tabular}{@{}c@{}}{\tiny 51}\\[0pt]0\end{tabular} & \begin{tabular}{@{}c@{}}{\tiny 52}\\[0pt]0\end{tabular} & \begin{tabular}{@{}c@{}}{\tiny 53}\\[0pt]0\end{tabular} & \begin{tabular}{@{}c@{}}{\tiny 54}\\[0pt]0\end{tabular} & \begin{tabular}{@{}c@{}}{\tiny 55}\\[0pt]\textbf{1}\end{tabular} \\[3pt] \hline
6 & \begin{tabular}{@{}c@{}}{\tiny 40}\\[0pt]\textbf{1}\end{tabular} & \begin{tabular}{@{}c@{}}{\tiny 41}\\[0pt]0\end{tabular} & \begin{tabular}{@{}c@{}}{\tiny 42}\\[0pt]0\end{tabular} & \begin{tabular}{@{}c@{}}{\tiny 43}\\[0pt]0\end{tabular} & \begin{tabular}{@{}c@{}}{\tiny 44}\\[0pt]0\end{tabular} & \begin{tabular}{@{}c@{}}{\tiny 45}\\[0pt]0\end{tabular} & \begin{tabular}{@{}c@{}}{\tiny 46}\\[0pt]0\end{tabular} & \begin{tabular}{@{}c@{}}{\tiny 47}\\[0pt]\textbf{1}\end{tabular} \\[3pt] \hline
5 & \begin{tabular}{@{}c@{}}{\tiny 32}\\[0pt]\textbf{1}\end{tabular} & \begin{tabular}{@{}c@{}}{\tiny 33}\\[0pt]0\end{tabular} & \begin{tabular}{@{}c@{}}{\tiny 34}\\[0pt]0\end{tabular} & \begin{tabular}{@{}c@{}}{\tiny 35}\\[0pt]0\end{tabular} & \begin{tabular}{@{}c@{}}{\tiny 36}\\[0pt]0\end{tabular} & \begin{tabular}{@{}c@{}}{\tiny 37}\\[0pt]0\end{tabular} & \begin{tabular}{@{}c@{}}{\tiny 38}\\[0pt]0\end{tabular} & \begin{tabular}{@{}c@{}}{\tiny 39}\\[0pt]\textbf{1}\end{tabular} \\[3pt] \hline
4 & \begin{tabular}{@{}c@{}}{\tiny 24}\\[0pt]\textbf{1}\end{tabular} & \begin{tabular}{@{}c@{}}{\tiny 25}\\[0pt]0\end{tabular} & \begin{tabular}{@{}c@{}}{\tiny 26}\\[0pt]0\end{tabular} & \begin{tabular}{@{}c@{}}{\tiny 27}\\[0pt]0\end{tabular} & \begin{tabular}{@{}c@{}}{\tiny 28}\\[0pt]0\end{tabular} & \begin{tabular}{@{}c@{}}{\tiny 29}\\[0pt]0\end{tabular} & \begin{tabular}{@{}c@{}}{\tiny 30}\\[0pt]0\end{tabular} & \begin{tabular}{@{}c@{}}{\tiny 31}\\[0pt]\textbf{1}\end{tabular} \\[3pt] \hline
3 & \begin{tabular}{@{}c@{}}{\tiny 16}\\[0pt]\textbf{1}\end{tabular} & \begin{tabular}{@{}c@{}}{\tiny 17}\\[0pt]0\end{tabular} & \begin{tabular}{@{}c@{}}{\tiny 18}\\[0pt]0\end{tabular} & \begin{tabular}{@{}c@{}}{\tiny 19}\\[0pt]0\end{tabular} & \begin{tabular}{@{}c@{}}{\tiny 20}\\[0pt]0\end{tabular} & \begin{tabular}{@{}c@{}}{\tiny 21}\\[0pt]0\end{tabular} & \begin{tabular}{@{}c@{}}{\tiny 22}\\[0pt]0\end{tabular} & \begin{tabular}{@{}c@{}}{\tiny 23}\\[0pt]\textbf{1}\end{tabular} \\[3pt] \hline
2 & \begin{tabular}{@{}c@{}}{\tiny 8}\\[0pt]\textbf{1}\end{tabular} & \begin{tabular}{@{}c@{}}{\tiny 9}\\[0pt]0\end{tabular} & \cellcolor{lightgray}\begin{tabular}{@{}c@{}}{\tiny 10}\\[0pt]\textbf{0}\end{tabular} & \begin{tabular}{@{}c@{}}{\tiny 11}\\[0pt]0\end{tabular} & \begin{tabular}{@{}c@{}}{\tiny 12}\\[0pt]0\end{tabular} & \begin{tabular}{@{}c@{}}{\tiny 13}\\[0pt]0\end{tabular} & \begin{tabular}{@{}c@{}}{\tiny 14}\\[0pt]0\end{tabular} & \begin{tabular}{@{}c@{}}{\tiny 15}\\[0pt]\textbf{1}\end{tabular} \\[3pt] \hline
1 & \begin{tabular}{@{}c@{}}{\tiny 0}\\[0pt]\textbf{1}\end{tabular} & \begin{tabular}{@{}c@{}}{\tiny 1}\\[0pt]\textbf{1}\end{tabular} & \begin{tabular}{@{}c@{}}{\tiny 2}\\[0pt]\textbf{1}\end{tabular} & \begin{tabular}{@{}c@{}}{\tiny 3}\\[0pt]\textbf{1}\end{tabular} & \begin{tabular}{@{}c@{}}{\tiny 4}\\[0pt]\textbf{1}\end{tabular} & \begin{tabular}{@{}c@{}}{\tiny 5}\\[0pt]\textbf{1}\end{tabular} & \begin{tabular}{@{}c@{}}{\tiny 6}\\[0pt]\textbf{1}\end{tabular} & \begin{tabular}{@{}c@{}}{\tiny 7}\\[0pt]\textbf{1}\end{tabular} \\[3pt] \hline
\multicolumn{1}{c|}{} & a & b & c & d & e & f & g & h \\
\cline{2-9}
\end{tabular}
\caption{Edge mask for square c2}
\label{fig:edges_mask_c2}
\end{subfigure}

\caption{Edge masks excluding the rank and file of the piece}
\label{fig:edge_masks}
\end{figure}

\item À la ligne 61, on crée le bitboard \mintinline{c++}{moves_bb} des cases accessibles par la pièce à partir de la case \mintinline{c++}{sq}. Le paramètre \mintinline{c++}{occupied} est égal à zéro pour indiquer qu'il n'y a aucun obstacle. On retire les bords avec \mintinline{c++}{& ~edges} comme nous l'avions vu au paragraphe \ref{par:edge_exclusion}.

\item Aux lignes 62 à 71, on va créer toutes les configurations d'obstacles sur les cases accessibles par la pièce en \mintinline{c++}{sq}, sans considérer les cases dans \mintinline{c++}{edge}, en énumérant tous les sous-ensembles du bitboard \mintinline{c++}{moves_bb}.

\begin{itemize}
\item Aux lignes 62 et 63, le vecteur \mintinline{c++}{occupancies} va contenir toutes les configurations d'obstacles sur la trajectoire de la pièce, sans considérer les cases dans \mintinline{c++}{edge}, et pour chacune de ces configurations, le vecteur \mintinline{c++}{possible_moves} va contenir le bitboard des coups possibles pour cette configuration. Par exemple, on pourrait avoir comme configuration d'obstacles le bitboard de la figure \ref{fig:occupancy_d3_23} et les coups possibles associés seraient représentés par le bitboard de la figure \ref{fig:possible_moves_d3_23}.

\begin{figure}[H]
\centering
\begin{subfigure}[b]{0.42\textwidth}
\centering
\small
\begin{tabular}{|c|*{8}{>{\centering\arraybackslash}p{0.4cm}|}}
\hline
8 & \begin{tabular}{@{}c@{}}{\tiny 56}\\[0pt]0\end{tabular} & \begin{tabular}{@{}c@{}}{\tiny 57}\\[0pt]0\end{tabular} & \begin{tabular}{@{}c@{}}{\tiny 58}\\[0pt]0\end{tabular} & \begin{tabular}{@{}c@{}}{\tiny 59}\\[0pt]0\end{tabular} & \begin{tabular}{@{}c@{}}{\tiny 60}\\[0pt]0\end{tabular} & \begin{tabular}{@{}c@{}}{\tiny 61}\\[0pt]0\end{tabular} & \begin{tabular}{@{}c@{}}{\tiny 62}\\[0pt]0\end{tabular} & \begin{tabular}{@{}c@{}}{\tiny 63}\\[0pt]0\end{tabular} \\[3pt] \hline
7 & \begin{tabular}{@{}c@{}}{\tiny 48}\\[0pt]0\end{tabular} & \begin{tabular}{@{}c@{}}{\tiny 49}\\[0pt]0\end{tabular} & \begin{tabular}{@{}c@{}}{\tiny 50}\\[0pt]0\end{tabular} & \begin{tabular}{@{}c@{}}{\tiny 51}\\[0pt]0\end{tabular} & \begin{tabular}{@{}c@{}}{\tiny 52}\\[0pt]0\end{tabular} & \begin{tabular}{@{}c@{}}{\tiny 53}\\[0pt]0\end{tabular} & \begin{tabular}{@{}c@{}}{\tiny 54}\\[0pt]0\end{tabular} & \begin{tabular}{@{}c@{}}{\tiny 55}\\[0pt]0\end{tabular} \\[3pt] \hline
6 & \begin{tabular}{@{}c@{}}{\tiny 40}\\[0pt]0\end{tabular} & \begin{tabular}{@{}c@{}}{\tiny 41}\\[0pt]0\end{tabular} & \begin{tabular}{@{}c@{}}{\tiny 42}\\[0pt]0\end{tabular} & \begin{tabular}{@{}c@{}}{\tiny 43}\\[0pt]\textbf{1}\end{tabular} & \begin{tabular}{@{}c@{}}{\tiny 44}\\[0pt]0\end{tabular} & \begin{tabular}{@{}c@{}}{\tiny 45}\\[0pt]0\end{tabular} & \begin{tabular}{@{}c@{}}{\tiny 46}\\[0pt]0\end{tabular} & \begin{tabular}{@{}c@{}}{\tiny 47}\\[0pt]0\end{tabular} \\[3pt] \hline
5 & \begin{tabular}{@{}c@{}}{\tiny 32}\\[0pt]0\end{tabular} & \begin{tabular}{@{}c@{}}{\tiny 33}\\[0pt]0\end{tabular} & \begin{tabular}{@{}c@{}}{\tiny 34}\\[0pt]0\end{tabular} & \begin{tabular}{@{}c@{}}{\tiny 35}\\[0pt]\textbf{1}\end{tabular} & \begin{tabular}{@{}c@{}}{\tiny 36}\\[0pt]0\end{tabular} & \begin{tabular}{@{}c@{}}{\tiny 37}\\[0pt]\textbf{1}\end{tabular} & \begin{tabular}{@{}c@{}}{\tiny 38}\\[0pt]0\end{tabular} & \begin{tabular}{@{}c@{}}{\tiny 39}\\[0pt]0\end{tabular} \\[3pt] \hline
4 & \begin{tabular}{@{}c@{}}{\tiny 24}\\[0pt]0\end{tabular} & \begin{tabular}{@{}c@{}}{\tiny 25}\\[0pt]0\end{tabular} & \begin{tabular}{@{}c@{}}{\tiny 26}\\[0pt]0\end{tabular} & \begin{tabular}{@{}c@{}}{\tiny 27}\\[0pt]\textbf{1}\end{tabular} & \begin{tabular}{@{}c@{}}{\tiny 28}\\[0pt]\textbf{1}\end{tabular} & \begin{tabular}{@{}c@{}}{\tiny 29}\\[0pt]0\end{tabular} & \begin{tabular}{@{}c@{}}{\tiny 30}\\[0pt]0\end{tabular} & \begin{tabular}{@{}c@{}}{\tiny 31}\\[0pt]0\end{tabular} \\[3pt] \hline
3 & \begin{tabular}{@{}c@{}}{\tiny 16}\\[0pt]0\end{tabular} & \begin{tabular}{@{}c@{}}{\tiny 17}\\[0pt]0\end{tabular} & \begin{tabular}{@{}c@{}}{\tiny 18}\\[0pt]0\end{tabular} & \cellcolor{lightgray}\begin{tabular}{@{}c@{}}{\tiny 19}\\[0pt]\textbf{0}\end{tabular} & \begin{tabular}{@{}c@{}}{\tiny 20}\\[0pt]0\end{tabular} & \begin{tabular}{@{}c@{}}{\tiny 21}\\[0pt]\textbf{1}\end{tabular} & \begin{tabular}{@{}c@{}}{\tiny 22}\\[0pt]0\end{tabular} & \begin{tabular}{@{}c@{}}{\tiny 23}\\[0pt]0\end{tabular} \\[3pt] \hline
2 & \begin{tabular}{@{}c@{}}{\tiny 8}\\[0pt]0\end{tabular} & \begin{tabular}{@{}c@{}}{\tiny 9}\\[0pt]0\end{tabular} & \begin{tabular}{@{}c@{}}{\tiny 10}\\[0pt]0\end{tabular} & \begin{tabular}{@{}c@{}}{\tiny 11}\\[0pt]\textbf{1}\end{tabular} & \begin{tabular}{@{}c@{}}{\tiny 12}\\[0pt]0\end{tabular} & \begin{tabular}{@{}c@{}}{\tiny 13}\\[0pt]0\end{tabular} & \begin{tabular}{@{}c@{}}{\tiny 14}\\[0pt]0\end{tabular} & \begin{tabular}{@{}c@{}}{\tiny 15}\\[0pt]0\end{tabular} \\[3pt] \hline
1 & \begin{tabular}{@{}c@{}}{\tiny 0}\\[0pt]0\end{tabular} & \begin{tabular}{@{}c@{}}{\tiny 1}\\[0pt]0\end{tabular} & \begin{tabular}{@{}c@{}}{\tiny 2}\\[0pt]0\end{tabular} & \begin{tabular}{@{}c@{}}{\tiny 3}\\[0pt]0\end{tabular} & \begin{tabular}{@{}c@{}}{\tiny 4}\\[0pt]0\end{tabular} & \begin{tabular}{@{}c@{}}{\tiny 5}\\[0pt]0\end{tabular} & \begin{tabular}{@{}c@{}}{\tiny 6}\\[0pt]0\end{tabular} & \begin{tabular}{@{}c@{}}{\tiny 7}\\[0pt]0\end{tabular} \\[3pt] \hline
\multicolumn{1}{c|}{} & a & b & c & d & e & f & g & h \\
\cline{2-9}
\end{tabular}
\caption{Occupancy configuration example}
\label{fig:occupancy_d3_23}
\end{subfigure}
\hfill
\begin{subfigure}[b]{0.42\textwidth}
\centering
\small
\begin{tabular}{|c|*{8}{>{\centering\arraybackslash}p{0.4cm}|}}
\hline
8 & \begin{tabular}{@{}c@{}}{\tiny 56}\\[0pt]0\end{tabular} & \begin{tabular}{@{}c@{}}{\tiny 57}\\[0pt]0\end{tabular} & \begin{tabular}{@{}c@{}}{\tiny 58}\\[0pt]0\end{tabular} & \begin{tabular}{@{}c@{}}{\tiny 59}\\[0pt]0\end{tabular} & \begin{tabular}{@{}c@{}}{\tiny 60}\\[0pt]0\end{tabular} & \begin{tabular}{@{}c@{}}{\tiny 61}\\[0pt]0\end{tabular} & \begin{tabular}{@{}c@{}}{\tiny 62}\\[0pt]0\end{tabular} & \begin{tabular}{@{}c@{}}{\tiny 63}\\[0pt]0\end{tabular} \\[3pt] \hline
7 & \begin{tabular}{@{}c@{}}{\tiny 48}\\[0pt]0\end{tabular} & \begin{tabular}{@{}c@{}}{\tiny 49}\\[0pt]0\end{tabular} & \begin{tabular}{@{}c@{}}{\tiny 50}\\[0pt]0\end{tabular} & \begin{tabular}{@{}c@{}}{\tiny 51}\\[0pt]0\end{tabular} & \begin{tabular}{@{}c@{}}{\tiny 52}\\[0pt]0\end{tabular} & \begin{tabular}{@{}c@{}}{\tiny 53}\\[0pt]0\end{tabular} & \begin{tabular}{@{}c@{}}{\tiny 54}\\[0pt]0\end{tabular} & \begin{tabular}{@{}c@{}}{\tiny 55}\\[0pt]0\end{tabular} \\[3pt] \hline
6 & \begin{tabular}{@{}c@{}}{\tiny 40}\\[0pt]\textbf{1}\end{tabular} & \begin{tabular}{@{}c@{}}{\tiny 41}\\[0pt]0\end{tabular} & \begin{tabular}{@{}c@{}}{\tiny 42}\\[0pt]0\end{tabular} & \begin{tabular}{@{}c@{}}{\tiny 43}\\[0pt]0\end{tabular} & \begin{tabular}{@{}c@{}}{\tiny 44}\\[0pt]0\end{tabular} & \begin{tabular}{@{}c@{}}{\tiny 45}\\[0pt]0\end{tabular} & \begin{tabular}{@{}c@{}}{\tiny 46}\\[0pt]0\end{tabular} & \begin{tabular}{@{}c@{}}{\tiny 47}\\[0pt]0\end{tabular} \\[3pt] \hline
5 & \begin{tabular}{@{}c@{}}{\tiny 32}\\[0pt]0\end{tabular} & \begin{tabular}{@{}c@{}}{\tiny 33}\\[0pt]\textbf{1}\end{tabular} & \begin{tabular}{@{}c@{}}{\tiny 34}\\[0pt]0\end{tabular} & \begin{tabular}{@{}c@{}}{\tiny 35}\\[0pt]0\end{tabular} & \begin{tabular}{@{}c@{}}{\tiny 36}\\[0pt]0\end{tabular} & \begin{tabular}{@{}c@{}}{\tiny 37}\\[0pt]0\end{tabular} & \begin{tabular}{@{}c@{}}{\tiny 38}\\[0pt]0\end{tabular} & \begin{tabular}{@{}c@{}}{\tiny 39}\\[0pt]0\end{tabular} \\[3pt] \hline
4 & \begin{tabular}{@{}c@{}}{\tiny 24}\\[0pt]0\end{tabular} & \begin{tabular}{@{}c@{}}{\tiny 25}\\[0pt]0\end{tabular} & \begin{tabular}{@{}c@{}}{\tiny 26}\\[0pt]\textbf{1}\end{tabular} & \begin{tabular}{@{}c@{}}{\tiny 27}\\[0pt]0\end{tabular} & \begin{tabular}{@{}c@{}}{\tiny 28}\\[0pt]0\end{tabular} & \begin{tabular}{@{}c@{}}{\tiny 29}\\[0pt]0\end{tabular} & \begin{tabular}{@{}c@{}}{\tiny 30}\\[0pt]0\end{tabular} & \begin{tabular}{@{}c@{}}{\tiny 31}\\[0pt]0\end{tabular} \\[3pt] \hline
3 & \begin{tabular}{@{}c@{}}{\tiny 16}\\[0pt]\textbf{1}\end{tabular} & \begin{tabular}{@{}c@{}}{\tiny 17}\\[0pt]\textbf{1}\end{tabular} & \begin{tabular}{@{}c@{}}{\tiny 18}\\[0pt]\textbf{1}\end{tabular} & \cellcolor{lightgray}\begin{tabular}{@{}c@{}}{\tiny 19}\\[0pt]\textbf{0}\end{tabular} & \begin{tabular}{@{}c@{}}{\tiny 20}\\[0pt]\textbf{1}\end{tabular} & \begin{tabular}{@{}c@{}}{\tiny 21}\\[0pt]0\end{tabular} & \begin{tabular}{@{}c@{}}{\tiny 22}\\[0pt]0\end{tabular} & \begin{tabular}{@{}c@{}}{\tiny 23}\\[0pt]0\end{tabular} \\[3pt] \hline
2 & \begin{tabular}{@{}c@{}}{\tiny 8}\\[0pt]0\end{tabular} & \begin{tabular}{@{}c@{}}{\tiny 9}\\[0pt]0\end{tabular} & \begin{tabular}{@{}c@{}}{\tiny 10}\\[0pt]\textbf{1}\end{tabular} & \begin{tabular}{@{}c@{}}{\tiny 11}\\[0pt]0\end{tabular} & \begin{tabular}{@{}c@{}}{\tiny 12}\\[0pt]\textbf{1}\end{tabular} & \begin{tabular}{@{}c@{}}{\tiny 13}\\[0pt]0\end{tabular} & \begin{tabular}{@{}c@{}}{\tiny 14}\\[0pt]0\end{tabular} & \begin{tabular}{@{}c@{}}{\tiny 15}\\[0pt]0\end{tabular} \\[3pt] \hline
1 & \begin{tabular}{@{}c@{}}{\tiny 0}\\[0pt]0\end{tabular} & \begin{tabular}{@{}c@{}}{\tiny 1}\\[0pt]\textbf{1}\end{tabular} & \begin{tabular}{@{}c@{}}{\tiny 2}\\[0pt]0\end{tabular} & \begin{tabular}{@{}c@{}}{\tiny 3}\\[0pt]0\end{tabular} & \begin{tabular}{@{}c@{}}{\tiny 4}\\[0pt]0\end{tabular} & \begin{tabular}{@{}c@{}}{\tiny 5}\\[0pt]\textbf{1}\end{tabular} & \begin{tabular}{@{}c@{}}{\tiny 6}\\[0pt]0\end{tabular} & \begin{tabular}{@{}c@{}}{\tiny 7}\\[0pt]0\end{tabular} \\[3pt] \hline
\multicolumn{1}{c|}{} & a & b & c & d & e & f & g & h \\
\cline{2-9}
\end{tabular}
\caption{Possible moves for this occupancy}
\label{fig:possible_moves_d3_23}
\end{subfigure}

\caption{Example of occupancy configuration and corresponding possible moves}
\label{fig:occupancy_example}
\end{figure}

\item La boucle entre les lignes 66 et 71 va permettre d'énumérer tous les sous-ensembles du bitboard \mintinline{c++}{moves_bb}. On utilise pour se faire la technique appelée \emph{Carry-Rippler}\footnote{Si vous êtes intéressés par ce genre d'astuces de manipulation bit-à-bit, vous aimerez l'ouvrage \emph{Hacker's Delight}\cite{warren2012hackersdelight}.} à la ligne 70. Prenons un exemple fictif avec \mintinline{c++}{moves_bb = 0b1011}. On obtient alors
\begin{minted}[linenos, breaklines,fontsize=\small]{c++}
uint64_t b = 0;                 // b == 0
b = (0 - 0b1011) & 0b1011;      // b == 0b0001
b = (0b0001 - 0b1011) & 0b1011; // b == 0b0010
b = (0b0010 - 0b1011) & 0b1011; // b == 0b0011
b = (0b0011 - 0b1011) & 0b1011; // b == 0b1000
b = (0b1000 - 0b1011) & 0b1011; // b == 0b1001
b = (0b1001 - 0b1011) & 0b1011; // b == 0b1010
b = (0b1010 - 0b1011) & 0b1011; // b == 0b1011
b = (0b1011 - 0b1011) & 0b1011; // b == 0
\end{minted}

\vspace{0.3cm}

\item À la ligne 68, on crée les coups possibles pour la configuration d'obstacles \mintinline{c++}{b}. Notons que comme \mintinline{c++}{b} ne couvre pas les cases de \mintinline{c++}{edge}, si aucun obstacle dans \mintinline{c++}{b} ne bloque les déplacements de la pièce jusqu'à une case dans \mintinline{c++}{edge}, celle-ci apparaîtra dans les coups possibles. 

\end{itemize}

\item À la ligne 72, on compte le nombre de bits à $1$ dans le bitboard \mintinline{c++}{moves_bb}, ce qui revient à compter le nombre de sous-ensembles de configurations d'obstacles. Nous fixons \mintinline{c++}{k} à cette valeur, ce qui veut dire que pour la case \mintinline{c++}{sq} et le type de mouvement \mintinline{c++}{mt}, la place occupée par notre table de hachage parfait sera de $2^k$\footnote{C'est la façon de faire dans le logiciel Stockfish\cite{stockfish2025}.}. Notons que certaines configurations d'obstacles produisent les mêmes coups possibles, il est donc peut-être possible d'obtenir une valeur de \mintinline{c++}{k} plus petite. Mais cette façon de faire permet de trouver les différentes constantes \mintinline{c++}{MAGIC} très rapidement et la place mémoire consommée est relativement faible ($105$Kio en tout).

\item Aux lignes 74 à 76, on met en place le générateur aléatoire qui va nous permettre de créer les constantes \mintinline{c++}{MAGIC}.

\item À la ligne 77, le tableau \mintinline{c++}{seen} va nous permettre de mémoriser les indices, obtenues par la formule de la ligne 84, que l'on a déjà rencontrés.

\item À la ligne 78, le tableau \mintinline{c++}{moves} va nous permettre de mémoriser les coups possibles pour un indice donné. Il se peut que deux configurations d'obstacles différentes donnent lieu aux mêmes coups possibles, ce tableau nous permettra de ne pas considérer comme des collisions ces deux configurations différentes menant au même indice, mais ayant les mêmes coups possibles.

\item La boucle de la ligne 79 à 96 va rechercher la constante \mintinline{c++}{MAGIC} permettant de créer le hachage parfait pour la case \mintinline{c++}{sq} et la valeur \mintinline{c++}{k}. Notons que cette boucle est potentiellement infinie. Nous informons le compilateur, à la ligne 97, que cette partie du code ne sera jamais atteinte grâce à la fonction \mintinline{c++}{std::unreachable}.

\item À la ligne 80, on génère un candidat \mintinline{c++}{MAGIC}. On génère trois nombres aléatoires et on fait un ET bit-à-bit entre eux pour obtenir un nombre aléatoire avec moins de bits à $1$. Cette astuce est très importante car sans celle-ci nous n'arrivions pas à trouver un en temps raisonnable les constantes \mintinline{c++}{MAGIC}~!

\item Aux lignes 82 à 92, on vérifie qu'il n'y a pas de collisions pour la constante considérée. Si c'est le cas, on retourne la valeur \mintinline{c++}{k} et la constante magique trouvée.

\begin{itemize}
\item À la ligne 84, pour une configuration d'obstacles donnée, on calcule son indice \mintinline{c++}{index} grâce à la fonction de hachage parfait \mintinline{c++}{magic * occ >> (64 - k)}.

\item À la ligne 85, on test s'il y a une collision. On utilise une petite astuce classique pour ne pas avoir à réinitialiser \mintinline{c++}{seen} à zéro à chaque fois que l'on va tester un nouveau candidat \mintinline{c++}{magic}. La variable \mintinline{c++}{cnt} nous permet de savoir si la valeur contenue dans seen pour la case index est une valeur mise à jour pour un ancien candidat \mintinline{c++}{magic} ou est pour le candidat actuel. En effet, si \mintinline{c++}{seen[index]} est inférieur à \mintinline{c++}{cnt}, la valeur n'est plus d'actualité. Maintenant, la partie \mintinline{c++}{moves[index] != possible_moves[j]} nous permet d'ignorer une collision qui produit les mêmes coups possibles. 

\end{itemize}

\end{itemize}

\item Aux lignes 100 à 116, le \mintinline{c++}{main} va produire et afficher sur la sortie standard toutes les valeurs de \mintinline{c++}{k} et toutes les constantes \mintinline{c++}{MAGIC} pour chacune des cases du plateau et pour les deux types de mouvement. Ces constantes vont nous servir bientôt pour initialiser la table de tous les coups possibles pour chacune des configurations d'obstacles. On peut voir une sortie possible dans le listing \ref{lst:find_magic_exec}.

\end{itemize}


\begin{mdframed}[skipabove=\baselineskip,hidealllines=true]
\begin{minted}[linenos, breaklines,fontsize=\small]{shell}
int O_K[64] = {
    12, 11, 11, 11, 11, 11, 11, 12,
    11, 10, 10, 10, 10, 10, 10, 11,
    11, 10, 10, 10, 10, 10, 10, 11,
    11, 10, 10, 10, 10, 10, 10, 11,
    11, 10, 10, 10, 10, 10, 10, 11,
    11, 10, 10, 10, 10, 10, 10, 11,
    11, 10, 10, 10, 10, 10, 10, 11,
    12, 11, 11, 11, 11, 11, 11, 12,
};
uint64_t O_MAGIC[64] = {
    0x80011040002082,     0x40022002100040,     0x1880200081181000,
    0x2080240800100080,   0x8080024400800800,   0x4100080400024100,
    0xc080028001000a00,   0x80146043000080,
    0x8120802080034004,   0x8401000200240,      0x202001282002044,
    0x81010021000b1000,   0x808044000800,       0x300080800c000200,
    0x8c000268411004,     0x810080058020c100,
    0xc248608010400080,   0x30024040002000,     0x9001010042102000,
    0x210009001002,       0xa0061d0018001100,   0x2410808004000600,
    0x6400240008025001,   0xc10600010340a4,
    0x628080044011,       0x4810014040002000,   0x380200080801000,
    0x10018580080010,     0x101040080180180,    0x9208020080040080,
    0x10400a21008,        0x6800104200010484,
    0x21400280800020,     0x9400402008401001,   0x8430006800200400,
    0x8104411202000820,   0x8010171000408,      0x1202000402001008,
    0x881100904002208,    0x15a0800a49802100,
    0x224001808004,       0x4420201002424000,   0xc04500020008080,
    0x2503009004210008,   0x42801010010,        0x2000400090100,
    0x8080011810040002,   0x44401c008046000d,
    0x4000800521104100,   0x82000b080400080,    0x10821022420200,
    0x9488a82104100100,   0x1004800041100,      0x81600a0034008080,
    0xa00056210280400,    0x5124088200,
    0x4210410010228202,   0x1802230840001081,   0x1002102000400901,
    0x1100c46010000901,   0x281000408001003,    0xc001001c00028809,
    0x10020008008c4102,   0x280005008c014222,
};

int D_K[64] = {
    6, 5, 5, 5, 5, 5, 5, 6,
    5, 5, 5, 5, 5, 5, 5, 5,
    5, 5, 7, 7, 7, 7, 5, 5,
    5, 5, 7, 9, 9, 7, 5, 5,
    5, 5, 7, 9, 9, 7, 5, 5,
    5, 5, 7, 7, 7, 7, 5, 5,
    5, 5, 5, 5, 5, 5, 5, 5,
    6, 5, 5, 5, 5, 5, 5, 6,
};
uint64_t D_MAGIC[64] = {
    0x811100100408200,    0x412100401044020,    0x404044c00408002,
    0xa0c070200010102,    0x104042001400008,    0x8802013008080000,
    0x1001008860080080,   0x20220044202800,
    0x2002610802080160,   0x4080800808610,      0x91c2800a10a0132,
    0x400242401822000,    0x8530040420040001,   0x142010c210048,
    0x8841820801241004,   0x804212084108801,
    0x2032402094100484,   0x40202110010210a2,   0x8010000800202020,
    0x800240421a800,      0x62200401a00444,     0x224082200820845,
    0x106021492012000,    0x8481020082849000,
    0x40a110c59602800,    0x10020108020400,     0x208c020844080010,
    0x2000480004012020,   0x8001004004044000,   0xa044104128080200,
    0x1108008015cc1400,   0x8284004801844400,
    0x8180a020c2004,      0x9101004080100,      0x8840264108800c0,
    0xc004200900200900,   0x8040008020020020,   0x20010802e1920200,
    0x80204000480a0,      0xc0a80a100008400,
    0x4018808114000,      0x90092200b9000,      0x80020c0048000400,
    0x6018005500,         0x80a0204110a00,      0x4018808407201,
    0x6050040806500280,   0x108208400c40180,
    0x803081210840480,    0x201210402200200,    0x200010400920042,
    0x902000a884110010,   0x851002021004,       0x43c08020120,
    0x6140500501010044,   0x200a04440400c028,
    0x14a002084046000,    0x10002409041040,     0x100022020500880b,
    0x1000000000460802,   0x21084104410,        0x8000001053300104,
    0x4000182008c20048,   0x112088105020200,
};
\end{minted}
\captionof{listing}{Exemple de la sortie pour une exécution du programme du listing \ref{lst:find_magic}}
\label{lst:find_magic_exec}
\end{mdframed} 

Maintenant que nous détenons les constantes \mintinline{c++}{MAGIC}, nous allons pouvoir initialiser grâce à elles des tableaux, décrits dans le listing \ref{lst:magic_tables}, qui vont stocker les informations nécessaires nous permettant de générer les coups possibles à partir d'une case donnée et d'une configuration d'obstacles.
\begin{itemize}
\item À la ligne 1, le tableau \mintinline{c++}{orthogonalTable} contiendra les bitboards des coups possibles pour tous les coups orthogonaux, pour toutes les cases et toutes les configurations d'obstacles. La valeur $102400$ pour la taille de ce tableau s'obtient en faisant
\begin{minted}[breaklines,fontsize=\small]{c++}
size_t size = 0;
for (int i = 0; i < 64; i++) {
    size += 1 << O_K[i];
}
\end{minted}
\item À la ligne 2, le tableau \mintinline{c++}{diagonalTable} fera de même mais pour les coups diagonaux.
\item À la ligne 3, le tableau \mintinline{c++}{orthogonalMagics} va définir pour chaque case du plateau les différentes informations nous permettant d'appliquer le hachage parfait et retrouver les coups possibles dans le tableau \mintinline{c++}{orthogonalTable}. Nous allons décrire juste après la structure \mintinline{c++}{Magic} présentée dans le listing \ref{lst:magic_struct}.
\item À la ligne 4, le tableau \mintinline{c++}{diagonalMagics} fera de même pour les coups diagonaux.
\end{itemize}


\begin{mdframed}[skipabove=\baselineskip,hidealllines=true]
\begin{minted}[linenos, breaklines,fontsize=\small]{c++}
uint64_t orthogonalTable[102400];
uint64_t diagonalTable[5248];
Magic orthogonalMagics[SQUARE_NB];
Magic diagonalMagics[SQUARE_NB];
\end{minted}
\captionof{listing}{Les tableaux nous permettant de générer les coups possibles à partir d'une case donnée et d'une configuration d'obstacles}
\label{lst:magic_tables}
\end{mdframed}

Le listing \ref{lst:magic_struct} décrit la structure permettant de stocker les informations permettant d'appliquer le hachage parfait pour une case donnée.
\begin{itemize}
\item À la ligne 2, l'attribut \mintinline{c++}{mask} est le bitboard permettant d'isoler la partie du plateau qui contient les cases concernées par le déplacement de la pièce pour la case et le type de déplacement considérés. Ce masque correspond à la variable \mintinline{c++}{moves_bb} à la ligne 61 du listing \ref{lst:find_magic}.

\item À la ligne 3, l'attribut \mintinline{c++}{magic} est la constante \mintinline{c++}{MAGIC} que l'on a trouvé pour la case et le déplacement considérés. Ce attribut prendra la valeur d'une des constantes des tables \mintinline{c++}{O_MAGIC} ou \mintinline{c++}{D_MAGIC} du listing \ref{lst:find_magic_exec}.

\item À la ligne 4, l'attribut \mintinline{c++}{moves} est un pointeur sur la partie de la table \mintinline{c++}{orthogonalTable} ou \mintinline{c++}{diagonalTable}, selon le type de déplacement, qui contiendra les coups possibles pour la case considérée.

\item À la ligne 5, l'attribut \mintinline{c++}{shift} est le déplacement \mintinline{c++}{(64 - K)} dans la formule du listing \ref{lst:magic_perfect_hashing}.

\item Aux lignes 7 à 9, la fonction \mintinline{c++}{uint32_t index(uint64_t occupied) const}, calcule la position dans \mintinline{c++}{moves} où trouver le bitboard des coups possibles. On retrouve la formule du listing \ref{lst:magic_perfect_hashing} avec le masque appliquée aux obstacles pour ne considérer que ceux dans la trajectoire de la pièce.

\end{itemize}

\begin{mdframed}[skipabove=\baselineskip,hidealllines=true]
\begin{minted}[linenos, breaklines,fontsize=\small]{c++}
struct Magic {
    uint64_t  mask;
    uint64_t  magic;
    uint64_t* moves;
    uint32_t  shift;

    uint32_t index(uint64_t occupied) const {
        return uint32_t( ((occupied & mask) * magic) >> shift );
    }
};
\end{minted}
\captionof{listing}{Structure de donnée nous permettant de stocker les informations utiles pour trouver les coups possibles pour une case et une configuration d'obstacles données}
\label{lst:magic_struct}
\end{mdframed} 

La fonction \mintinline{c++}{uint64_t moves_bb(Square sq, uint64_t occupied)} du listing \ref{lst:possible_moves} va se servir des tables du listing \ref{lst:magic_tables} pour calculer le bitboard des coups possibles, orthogonaux et diagonaux, pour la case \mintinline{c++}{sq} et la configuration d'obstacles \mintinline{c++}{occupied}. C'est cette fonction qui sera utilisée pour la génération des coups dans la partie \ref{subsection:making_and_unmaking_moves}. Pour ce faire,

\begin{itemize}
\item À la ligne 2, on calcule l'index \mintinline{c++}{idx_omoves}, grâce au hachage parfait, en utilisant la structure \mintinline{c++}{Magic} pour la case \mintinline{c++}{sq} et les coups orthogonaux.

\item À la ligne 3, on calcule l'index \mintinline{c++}{idx_dmoves}, grâce au hachage parfait, en utilisant la structure \mintinline{c++}{Magic} pour la case \mintinline{c++}{sq} et les coups diagonaux.

\item Aux lignes 4 et 5, on peut récupérer le bitboard des coups possibles orthogonaux (ligne 4) et faire l'union de ceux-ci avec les coups diagonaux en combinant ce bitboard par un OU bit-à-bit avec celui des coups possibles diagonaux (ligne 5). On obtient ainsi le bitboard de tous les coups possibles pour la pièce en \mintinline{c++}{sq} en prenant en considération tous les obstacles du bitboard \mintinline{c++}{occupied}.

\end{itemize}

\begin{mdframed}[skipabove=\baselineskip,hidealllines=true]
\begin{minted}[linenos, breaklines,fontsize=\small]{c++}
uint64_t moves_bb(Square sq, uint64_t occupied) {
    uint32_t idx_omoves = orthogonalMagics[sq].index(occupied);
    uint32_t idx_dmoves = diagonalMagics[sq].index(occupied);
    return orthogonalMagics[sq].moves[idx_omoves] | 
            diagonalMagics[sq].moves[idx_dmoves];
}
\end{minted}
\captionof{listing}{La fonction \mintinline{c++}{moves_bb} utilisent les éléments des listings \ref{lst:magic_tables} et \ref{lst:magic_struct} pour calculer efficacement le bitboard des coups possibles pour une case et une configuration d'obstacles données}
\label{lst:possible_moves}
\end{mdframed}

\begin{mdframed}[skipabove=\baselineskip,hidealllines=true]
\begin{minted}[linenos, breaklines,fontsize=\small]{c++}
void init() {
    init_magics(ORTHOGONAL, orthogonalTable, orthogonalMagics);
    init_magics(DIAGONAL, diagonalTable, diagonalMagics);
}
\end{minted}
\captionof{listing}{}
\label{lst:magic_init}
\end{mdframed}


\begin{mdframed}[skipabove=\baselineskip,hidealllines=true]
\begin{minted}[linenos, breaklines,fontsize=\small]{c++}
Square& operator++(Square& d) { return d = Square(int(d) + 1); }

void init_magics(MoveType mt, uint64_t table[], Magic magics[]) {
    static constexpr uint64_t O_MAGIC[64] = { 0x80011040002082, // ...
    static constexpr uint64_t D_MAGIC[64] = { 0x811100100408200,// ...
    using namespace std;
    int32_t size = 0;
    vector<uint64_t> occupancies;
    vector<uint64_t> possible_moves;
    for (Square sq = SQ_A1; sq <= SQ_H8; ++sq) {
        occupancies.clear();
        possible_moves.clear();
        Magic& m = magics[sq];
        uint64_t edges = ((Rank1BB | Rank8BB) & ~rank_bb(sq)) | 
                            ((FileABB | FileHBB) & ~file_bb(sq));
        uint64_t moves_bb = sliding_moves(mt, sq, 0) & ~edges;
        m.mask = moves_bb;
        m.shift = 64 - popcount(m.mask);
        m.magic = (mt == ORTHOGONAL ? O_MAGIC : D_MAGIC)[sq];        
        m.moves = sq == SQ_A1 ? table : magics[sq - 1].moves + size;
        size = 0;
        uint64_t b = 0;
        do {
            occupancies.push_back(b);
            possible_moves.push_back(sliding_moves(mt, sq, b));
            b = (b - moves_bb) & moves_bb;
            size++;
        } while (b);
        for (int32_t j = 0; j < size; j++) {
            int32_t index = m.index(occupancies[j]);
            m.moves[index] = possible_moves[j];
        }
    }
}
\end{minted}
\captionof{listing}{}
\label{lst:find_magics}
\end{mdframed} 

\subsection{Making and Unmaking Moves}
\label{subsection:making_and_unmaking_moves}

\section{Testing with Random Games}


\section{What's Next}


\section{Complete Commented Game Board Code}


% \begin{importantbox}
% Efficient board representation is critical for performance. Every AI algorithm will access board state thousands or millions of times during search.
% \end{importantbox}

% \section{Move Generation}

% Move generation is the process of finding all legal moves from a given position. This is a critical component that affects both correctness and performance.

% \subsection{Algorithm}

% The move generation algorithm follows this structure:

% \begin{algorithmbox}
% \begin{algorithmic}[1]
% \Procedure{GenerateMoves}{$board$}
%     \State $moves \gets []$
%     \For{each position $(row, col)$ on $board$}
%         \If{position is empty}
%             \State $move \gets \textsc{CreateMove}(row, col)$
%             \If{$move$ is legal}
%                 \State $moves.\textsc{append}(move)$
%             \EndIf
%         \EndIf
%     \EndFor
%     \State \Return $moves$
% \EndProcedure
% \end{algorithmic}
% \end{algorithmbox}

% \section{Performance Considerations}

% When building a game engine, performance is paramount. Consider these key metrics:

% \begin{table}[H]
% \centering
% \begin{tabular}{lcc}
% \toprule
% \textbf{Operation} & \textbf{Target Time} & \textbf{Frequency} \\
% \midrule
% Move generation & $< 1$ ms & Every node \\
% Make/unmake move & $< 0.1$ ms & Every node \\
% Position evaluation & $< 5$ ms & Leaf nodes \\
% Legal move check & $< 0.01$ ms & Very frequent \\
% \bottomrule
% \end{tabular}
% \caption{Performance targets for engine operations}
% \label{tab:performance}
% \end{table}

% \begin{resultbox}
% Our optimized engine achieves over 100,000 nodes per second on modern hardware, enabling deep search within reasonable time constraints.
% \end{resultbox}


\chapter{AI Players}
\label{ch:ai}

% \section{Introduction to Minimax}

% The Minimax algorithm is the foundation of game-playing AI. It assumes both players play optimally and searches the game tree to find the best move.

% \subsection{Basic Concept}

% The algorithm alternates between:
% \begin{itemize}
%     \item \textbf{Maximizing} player tries to maximize score
%     \item \textbf{Minimizing} player tries to minimize score
% \end{itemize}

% This creates a mathematical framework for two-player zero-sum games.

% \section{Implementation}

% Here's a complete Minimax implementation:

% \begin{listing}[H]
% \begin{minted}[linenos, bgcolor=codebg]{python}
% def minimax(board, depth, maximizing_player):
%     """
%     Standard Minimax algorithm implementation.

%     Args:
%         board: Current game state
%         depth: Remaining search depth
%         maximizing_player: True if maximizing, False if minimizing

%     Returns:
%         Best evaluation score for this position
%     """
%     # Base case: terminal position or depth limit
%     if depth == 0 or board.is_game_over():
%         return evaluate(board)

%     if maximizing_player:
%         max_eval = float('-inf')
%         for move in board.generate_moves():
%             board.make_move(move)
%             eval = minimax(board, depth - 1, False)
%             board.undo_move()
%             max_eval = max(max_eval, eval)
%         return max_eval
%     else:
%         min_eval = float('inf')
%         for move in board.generate_moves():
%             board.make_move(move)
%             eval = minimax(board, depth - 1, True)
%             board.undo_move()
%             min_eval = min(min_eval, eval)
%         return min_eval
% \end{minted}
% \caption{Minimax algorithm implementation}
% \label{code:minimax}
% \end{listing}

% \section{Complexity Analysis}

% The time complexity of Minimax is:

% \begin{equation}
%     T(d) = O(b^d)
%     \label{eq:minimax_complexity}
% \end{equation}

% where $b$ is the branching factor and $d$ is the search depth.

% For Yolah with an average branching factor of 30:
% \begin{align}
%     \text{Nodes at depth 1} &= 30 \\
%     \text{Nodes at depth 2} &= 30^2 = 900 \\
%     \text{Nodes at depth 4} &= 30^4 = 810,000 \\
%     \text{Nodes at depth 6} &= 30^6 = 729,000,000
% \end{align}

% \begin{importantbox}
% The exponential growth of nodes makes deep search impractical without optimization techniques like Alpha-Beta pruning.
% \end{importantbox}

% %%%%%%%%%%%%%%%%%%%%%%%%%%%%%%%%%%%%%%%%%%%%%%%%%%%%%%%%%%%%%%%%%%%%%%%%%
% % EXAMPLE FIGURE
% %%%%%%%%%%%%%%%%%%%%%%%%%%%%%%%%%%%%%%%%%%%%%%%%%%%%%%%%%%%%%%%%%%%%%%%%%

% \section{Visualizing Minimax}

% Figure~\ref{fig:minimax_tree} shows how Minimax explores the game tree.

% \begin{figure}[H]
% \centering
% \begin{tikzpicture}[
%     level distance=1.5cm,
%     level 1/.style={sibling distance=4cm},
%     level 2/.style={sibling distance=2cm},
%     every node/.style={circle, draw, minimum size=0.8cm}
% ]
% \node {Max}
%     child {node {Min}
%         child {node {5}}
%         child {node {3}}
%     }
%     child {node {Min}
%         child {node {7}}
%         child {node {2}}
%     }
%     child {node {Min}
%         child {node {4}}
%         child {node {6}}
%     };
% \end{tikzpicture}
% \caption{Minimax game tree example. The maximizing player chooses the move leading to the highest value.}
% \label{fig:minimax_tree}
% \end{figure}

% %%%%%%%%%%%%%%%%%%%%%%%%%%%%%%%%%%%%%%%%%%%%%%%%%%%%%%%%%%%%%%%%%%%%%%%%%
% % ADDITIONAL CHAPTERS
% %%%%%%%%%%%%%%%%%%%%%%%%%%%%%%%%%%%%%%%%%%%%%%%%%%%%%%%%%%%%%%%%%%%%%%%%%

% \chapter{Alpha-Beta Pruning}
% \label{ch:alphabeta}

% Alpha-Beta pruning dramatically improves Minimax by eliminating branches that cannot affect the final decision.

% \section{The Pruning Principle}

% Your content here...

\chapter{Monte Carlo Player}
\label{ch:monte_carlo}

% MCTS revolutionized game AI by using random simulations instead of exhaustive search.

% \section{UCB1 Formula}

% The Upper Confidence Bound formula balances exploration and exploitation:

% \begin{equation}
%     UCB1 = \frac{w_i}{n_i} + c\sqrt{\frac{\ln N}{n_i}}
%     \label{eq:ucb1}
% \end{equation}

% where:
% \begin{itemize}
%     \item $w_i$ = wins for node $i$
%     \item $n_i$ = visits to node $i$
%     \item $N$ = total visits to parent
%     \item $c$ = exploration constant
% \end{itemize}

% Continue with more chapters...

\chapter{MCTS Player}
\label{ch:mcts}

% Your content here...

\chapter{Minmax Player}
\label{ch:minmax}

% Your content here...

\chapter{Minmax with Neural Network Player}
\label{ch:nnue}

% Your content here...

\chapter{AI Tournament}
\label{ch:tournament}

% Your content here...

\chapter{Conclusion}
\label{ch:conclusion}

% This book has taken you on a journey from classical game-playing algorithms to modern neural network approaches. The Yolah engine demonstrates that combining multiple techniques yields the best results.

% \section{Key Takeaways}

% \begin{enumerate}
%     \item Classical algorithms remain relevant and effective
%     \item Neural networks provide powerful evaluation functions
%     \item Hybrid approaches leverage the strengths of multiple methods
%     \item Performance optimization is critical for practical systems
% \end{enumerate}

% \section{Future Directions}

% Promising areas for future research include:
% \begin{itemize}
%     \item More efficient neural architectures
%     \item Better exploration strategies
%     \item Transfer learning across game domains
%     \item Real-time learning and adaptation
% \end{itemize}

%%%%%%%%%%%%%%%%%%%%%%%%%%%%%%%%%%%%%%%%%%%%%%%%%%%%%%%%%%%%%%%%%%%%%%%%%
% BACK MATTER
%%%%%%%%%%%%%%%%%%%%%%%%%%%%%%%%%%%%%%%%%%%%%%%%%%%%%%%%%%%%%%%%%%%%%%%%%

\backmatter

% Appendix
\appendix

% \chapter{Installation Guide}
% \label{app:installation}

% \section{Requirements}

% To run the Yolah engine, you need:
% \begin{itemize}
%     \item Python 3.8 or higher
%     \item NumPy
%     \item TensorFlow or PyTorch
%     \item Additional dependencies listed in requirements.txt
% \end{itemize}

% \section{Setup Instructions}

% \begin{minted}[bgcolor=codebg]{bash}
% git clone https://github.com/yourusername/yolah.git
% cd yolah
% pip install -r requirements.txt
% python setup.py install
% \end{minted}

% Glossary
\printglossary[type=\acronymtype,title=Acronymes]
\printglossary[title=Glossaire]

% Bibliography
\printbibliography[heading=bibintoc]

% Index (if needed)
% \printindex

\end{document}
